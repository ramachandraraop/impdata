\documentclass[11pt,a4paper]{article}

\usepackage{graphicx}

\usepackage{listings}

\lstset{language=Java,numbers=left,numberstyle=\tiny,numbersep=10pt,showstringspaces=false}

\usepackage{array}

\usepackage{enumitem}



\def\AnswerBox{\fbox{\begin{minipage}{4in}\hfill\vspace{0.5in}\end{minipage}}}

\usepackage{fancyhdr}
\pagestyle{fancy}
\renewcommand{\headrulewidth}{0pt}
\rhead{\includegraphics[scale=.5]{../Images/TS-Logo.png}}


\begin{document}



\centerline{\huge{ \textbf{Inheritance and Polymorphism}}}

\vspace{1pc}

\centerline{\huge{ \textbf{ Workbook}}}

\section*{Answer the Following}
\begin{enumerate}\itemsep10pt

\item What is inheritance? 

\AnswerBox

\item  Why multiple inheritance is not supported in java?

\AnswerBox

\item  What is \texttt{super} in java?

\AnswerBox

\item  What is constructor chaining and how it is achieved in Java ?

    \AnswerBox
    
\item  Why does Java not support operator overloading?

    \AnswerBox
    
\item   Which object oriented concept is achieved by using overloading and
overriding? \underline{\hspace{2in}}

\item  What is the difference between static and dynamic polymorphism?

    \AnswerBox



   
    \end{enumerate}



\section*{State whether the following are True/False}
\begin{enumerate}\itemsep2pt
\item Java supports multiple inheritance for classes but only single inheritance for interfaces. [ ]
\item Inheritance implies an “is-a” relationship. [   ]
\item The private members of a superclass can be accessed by a subclass.[ ]
\item A subclass inherit both member variables and methods. [ ]
\item An object can be a subclass of another object. [ ]
\item subclass can directly access protected members of a superclass. [ ]
\item If a class is declared final, then no other class can be derived from this class. [ ]  % true
\item You must always use the reserved word super to use a method from the superclass in the subclass. [ ]  % false
\item The class Object is directly or indirectly the superclass of every class in Java. [ ] % true
\item The superclass inherits all its properties from the subclass. [ ] % false

\item Ploymorphism applies to static methods. [ ]
\item Overloaded methods can be overriden. [ ]
\item Main method can be overloaded. [ ]
\end{enumerate}

\section*{Multiple choice questions}
\begin{enumerate}
\item Which of these can be overloaded?
\begin{enumerate}
\item Methods
\item Constructors
\item Both a and b
\item none
\end{enumerate}
\item What is the process of defining two or more methods with in same
class that have same name but different parameters declaration?
\begin{enumerate}
\item method overloading
\item method overriding
\item method hiding
\item none
\end{enumerate}
\item Which of these keywords can be used to prevent method overriding?
\begin{enumerate}
\item static
\item constant
\item protected
\item final
\end{enumerate}

\item Which of these keyword is used to inherit a class?
\begin{enumerate}
\item super
\item extends
\item this
\item implements
\end{enumerate}
\item Which of these keywords is used to refer to member of base class from a sub class?
\begin{enumerate}
\item upper
\item super
\item this
\item none
\end{enumerate}
\item Which of these is correct way of inheriting class A by class B?
\begin{enumerate}
\item class A extends B {}
\item class B inherits class A {}
\item class B extends A {}
\item class B extends class A {}
\end{enumerate}
\item A class member declared protected becomes member of subclass of which type?
\begin{enumerate}
\item public member 
\item private member
\item protected member
\item static member
\end{enumerate}
\item  What restriction is there on using the super reference in a constructor?
\begin{enumerate}
\item It can only be used in the parent's constructor. 
\item Only one child class can use it. 
\item It must be used in the last statement of the constructor. 
\item It must be used in the first statement of the constructor. 
\end{enumerate}
\item  Which of the following is not an advantage of using inheritance?
\begin{enumerate}
\item Code that is shared between classes needs to be written only once. 
\item Similar classes can be made to behave consistently. 
\item Enhancements to a base class will automatically be applied to derived classes. 
\item One big superclass can be used instead of many little classes. 
\end{enumerate}
\end{enumerate}


\section*{Exercises}
\begin{itemize}
    \item Write the expected output, or compiler errors if any, for each of the following programs in the box provided below each program.
 \item Then execute the programs and check your answers.
 \item Then answer the questions given below.
\end{itemize}
\begin{description}
\item [Program 1]\
    \begin{lstlisting}
class Super { 
    public int num = 0; 
    public Super(String text) { /* Line 4 */
        num = 1; 
    } 
} 
class Sub extends Super {
    public Sub(String text) {
        num = 2; 
    } 
    public static void main(String args[]) {
        Sub sub = new Sub(``Hello''); 
        System.out.println(sub.num); 
    } 
}
\end{lstlisting}

\AnswerBox

\begin{enumerate}[label=\bfseries Q\arabic*:]\itemsep10pt
        \item What will be the output of the program?
    \end{enumerate}

\item [Program 2] \
\begin{lstlisting}
class Creature {
    void grow() {
    }
}
class Bird extends Creature {
    void fly() {
    }
}
class Falcon extends Bird {
    void hunt() {
    }
}
public class Tester {
    public static void main(String[] args) {
        Creature c1 = new Bird();
        Falcon c2 = new Falcon();
        // insert code here
    }
}
\end{lstlisting}

\AnswerBox

\begin{enumerate}[label=\bfseries Q\arabic*:]\itemsep10pt
\item What inserted, independently at // insert code here , will compile?
\end{enumerate}
                                
\item [Program 3] \
\begin{lstlisting}
class A {
    A() {
        System.out.println(``Hello'');
    }
} 
class InitDemo extends A {
    A ob = new A();
    InitDemo() {
        System.out.println(``hello 1'');
    }
    public static void main(String[] args){
        System.out.println(``Hello 2'');
        new InitDemo();
    }
}
\end{lstlisting}

\AnswerBox

\begin{enumerate}[label=\bfseries Q\arabic*:]\itemsep10pt 
        \item What will be the output of the above code?
    \end{enumerate}

\item[Program 4] \
    \begin{lstlisting}
class Base {
    int value = 0;
    Base() {
        addValue();
    }
    void addValue() {
        value += 10;
    }
    int getValue() {
        return value;
    }
}
class Derived extends Base {
    Derived() {
        addValue();
    }
    void addValue() {
        value +=  20;
    }
}
public class Test {
    public static void main(String[] args) {
        Base b = new Derived();
        System.out.println(b.getValue());
    }
}
\end{lstlisting}

\AnswerBox

\begin{enumerate}[label=\bfseries Q\arabic*:]\itemsep10pt
  \item What will the above program prints?
\end{enumerate}

\item [Program 5] \

\begin{lstlisting}
class Plant {
    Plant() {
        System.out.println(``Plant created'');
    }
}
class Tree extends Plant {
    Tree() {
        System.out.println(``Tree created'');
        super();
    }
}
public class Test {
    public static void main(String args[]) {
        Tree tree = new Tree();
    }
}
\end{lstlisting}

\AnswerBox

\begin{enumerate}[label=\bfseries Q\arabic*:]\itemsep10pt
 \item What will the above program print out?
\end{enumerate}

\item [Program 6] \
\begin{lstlisting}
public class Profile {
    private Profile(int w) { // line 1
        System.out.println(w);
    }
    public static Profile() { // line 5
        System.out.println(10);
    }
    public static void main(String args[]) {
        Profile obj = new Profile(50);
    }
}
\end{lstlisting}

\AnswerBox

\begin{enumerate}[label=\bfseries Q\arabic*:]\itemsep10pt
 \item What will the above program prints?
\end{enumerate}

\item [Program 7] \
\begin{lstlisting}
class A {
    final public int GetResult(int a, int b) { 
        return 0; 
    } 
} 
class B extends A { 
    public int GetResult(int a, int b) {
        return 1; 
    } 
} 
public class Test {
    public static void main(String args[]) { 
        B b = new B(); 
        System.out.println(``x = '' + b.GetResult(0, 1));  
    } 
}
\end{lstlisting}

\AnswerBox

\begin{enumerate}[label=\bfseries Q\arabic*:]\itemsep10pt
        \item What will be the output of the program?
\end{enumerate}

\item [Program 8] \
\begin{lstlisting}
class BoxVar {
    static void call(Integer... i) {
        System.out.println(``hi'' +i);
    }
    static void call(int... i ) {
        System.out.println(``hello''+i);
    }
    public static void main(String... args) {
        call(10);
    }
} 
\end{lstlisting}

\AnswerBox

\begin{enumerate}[label=\bfseries Q\arabic*:]\itemsep10pt
\item What will be the output of the program?
\end{enumerate}


\item [Program 9] \
\begin{lstlisting}
class Test {
    public static void main(String arg[]) {
        Number n = 10;
        int i = 10;
        System.out.println(n == i);
    }
}
\end{lstlisting}

\AnswerBox

\begin{enumerate}[label=\bfseries Q\arabic*:]\itemsep10pt
\item What will be the output of the program?
    \end{enumerate}

\item [Program 10] \
\begin{lstlisting}
class Creature {
    Creature getIt() {
        return this;
    }
}
class Bird extends Creature {
    // insert code here
}
class Falcon extends Bird {
}
\end{lstlisting}

\AnswerBox

\begin{enumerate}[label=\bfseries Q\arabic*:]\itemsep10pt
\item Which statement(s), inserted independently at // insert code here, will compile?
\end{enumerate}

\end{description}

\end{document}

