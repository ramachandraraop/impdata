\documentclass[11pt,a4paper]{article}

\usepackage{graphicx}

\usepackage{listings}

\lstset{language=Java,numbers=left,numberstyle=\tiny,numbersep=10pt,showstringspaces=false}

\usepackage{array}

\usepackage{enumitem}



\def\AnswerBox{\fbox{\begin{minipage}{4in}\hfill\vspace{0.5in}\end{minipage}}}

\usepackage{fancyhdr}
\pagestyle{fancy}
\renewcommand{\headrulewidth}{0pt}
\rhead{\includegraphics[scale=.5]{../Images/TS-Logo.png}}


\begin{document}



\centerline{\huge{ \textbf{Advanced Class Design}}}

\vspace{1pc}

\centerline{\huge{ \textbf{ Workbook}}}

\section*{Answer the Following}
\begin{enumerate}\itemsep10pt

\item What is `this' in java?

\AnswerBox

\item  What is the purpose of default constructor?

\AnswerBox

\item What is a package?

    \AnswerBox

\item What is the impact of declaring a method as final?

    \AnswerBox

\item  What is the access scope of a protected method?

    \AnswerBox

\item I want to print "Hello" even before main() is executed. How will you acheive that?

    \AnswerBox

\item What is constructor?

    \AnswerBox

\item Explain the purpose of Garbage collection.

    \AnswerBox


   \end{enumerate}

  \section*{State whether the following are True/False}
  \begin{enumerate}\itemsep2pt
      \item Static variables can be serialized. [         ]
      \item You use this() and super() both in a constructor. [ ]
      \item We can override static methods. [ ]
      \item We can make a constructor as final. [ ]
      \item Constructor returns a value [ ]
      

  \end{enumerate}
  \section*{Multiple Choice Questions}
  \begin{enumerate}
    \item Which statement is true?
        \begin{enumerate}
            \item Programs will not run out of memory.
            \item Objects that will never again be used are eligible for garbage collection.
            \item Objects that are referred to by other objects will never be garbage collected.
            \item Objects that can be reached from a live thread will never be garbage collected.
        \end{enumerate}
    \item Which statement is true?
        \begin{enumerate}
            \item Calling Runtime.gc() will cause eligible objects to be garbage collected.
            \item The garbage collector uses a mark and sweep algorithm.
            \item If an object can be accessed from a live thread, it can't be garbage collected.
            \item If object 1 refers to object 2, then object 2 can't be garbage collected.
        \end{enumerate}
    \item Which operator is used by Java run time implementations to free the memory of an object when it is no longer needed?
        \begin{enumerate}
        \item delete
        \item free
        \item new
        \item none
    \end{enumerate}
\item Which of the folowing statements are incorrect?
    \begin{enumerate}
        \item Default constructor is called at the time of declaration of the object if a constructor has not been defined.
        \item Constructor can be parameterized.
        \item finalize() method is called when a object goes out of scope and is no longer needed.
        \item finalize() method must be declared protected.
    \end{enumerate}
\item Which of these is used as default for a member of a class if no access specifier is used for it?
    \begin{enumerate}
        \item private
        \item public
        \item public, within its own package
        \item protected
    \end{enumerate}
\item Which of the following is a method having same name as that of it’s class?
    \begin{enumerate}
        \item finalize
        \item delete
        \item class
        \item constructor
    \end{enumerate}

\item Which method can be defined only once in a program?
    \begin{enumerate}
        \item main method
        \item finalize method
        \item static method
        \item private method
    \end{enumerate}

    \end{enumerate}

    \section*{Exercises}
\begin{itemize}
\item Write the expected output, or compiler errors if any, for each of the following programs in the box provided below each program.
\item Then execute the programs and check your answers.
\item Then answer the questions given below.
\end{itemize}
\begin{description}
\item [Program 1] \
\begin{lstlisting}
public class Test {
    static {
        print(10);
    }
    static void print(int x) {
        System.out.println(x);
        System.exit(0);
    }
}
\end{lstlisting}

\AnswerBox

\begin{enumerate}[label=\bfseries Q\arabic*:]\itemsep10pt
        \item What will happen if you try to compile and run above code?
    \end{enumerate}

\item [Program 2] \
    \begin{lstlisting}
package source;
class Test {
    public static void main(String args[]) {
        StringBuffer sb = new StringBuffer(``Hello World'');
        sb.insert(6 , ``Good '');
        System.out.println(sb);
    }
}
\end{lstlisting}

\AnswerBox

\begin{enumerate}[label=\bfseries Q\arabic*:]\itemsep10pt
\item What will be the output if Test class is not in source package?
\end{enumerate}

\item [Program 3] \
\begin{lstlisting}
package source;
class Display {
    int x;
    void show() {
        if (x > 1)
            System.out.print(x + `` '');
    }
}
class Test {
    public static void main(String args[]) {
        Display[] arr = new Display[3];
        for(int i = 0; i < 3; i++)
            arr[i] = new Display();
        arr[0].x = 0;      
        arr[1].x = 1;
        arr[2].x = 2;
        for (int i = 0; i < 3; ++i)
            arr[i].show();
    }
}
\end{lstlisting}
                       
\AnswerBox

\begin{enumerate}[label=\bfseries Q\arabic*:]\itemsep10pt
        \item What will be the output if Test class is in source package?
    \end{enumerate}

\item [Program 4] \
    \begin{lstlisting}
void start() {  
    A a = new A(); 
    B b = new B(); 
    a.s(b);  
    b = null; /* Line 5 */
    a = null;  /* Line 6 */
    System.out.println(``start completed''); /* Line 7 */
} 
\end{lstlisting}

\AnswerBox

\begin{enumerate}[label=\bfseries Q\arabic*:]\itemsep10pt
 \item When is the B object, created in line 3, eligible for garbage collection?
\end{enumerate}


\item [Program 5] \
    \begin{lstlisting}
public class X {
    public static void main(String [] args) {
        X x = new X();
        X x2 = m1(x); /* Line 6 */
        X x4 = new X();
        x2 = x4; /* Line 8 */
        doComplexStuff();
    }
    static X m1(X mx) {
        mx = new X();
        return mx;
    }
}
\end{lstlisting}

\AnswerBox

\begin{enumerate}[label=\bfseries Q\arabic*:]\itemsep10pt
\item After line 8 runs. how many objects are eligible for garbage collection?
         \end{enumerate}


\item [Program 6] \
    \begin{lstlisting}
public class A { 
    void A() {  /* Line 3 */
        System.out.println(``Class A''); 
    } 
    public static void main(String[] args) { 
        new A(); 
    } 
}
\end{lstlisting}

\AnswerBox

\begin{enumerate}[label=\bfseries Q\arabic*:]\itemsep10pt
 \item What will be the output of the above program?
 \end{enumerate}

 \item [Program 7] \
\begin{lstlisting}
public class Profile {
    private Profile(int w) { // line 1
        System.out.println(w);
    }
    public static Profile() { // line 5
        System.out.println(10);
    }
    public static void main(String args[]) {
        Profile obj = new Profile(50);
    }
}
\end{lstlisting}

\AnswerBox

\begin{enumerate}[label=\bfseries Q\arabic*:]\itemsep10pt
\item What will be the output of the above program?
\end{enumerate}


\item [Program 8] \
\begin{lstlisting}
public class Tester {
    Tester() { } // line 1
    static void Tester() { this(); } // line 2
    public static void main(String[] args) { // line 3
        Tester(); // line 4
    }
}
\end{lstlisting}

\AnswerBox

\begin{enumerate}[label=\bfseries Q\arabic*:]\itemsep10pt
\item The above code contains one compilation error, where could it be?
\end{enumerate}


\item [Program 9] \
\begin{lstlisting}
public class Tester {
    public static void main(String[] args) {
        int x = 1;
        int y;
        while(++x<5) y++;
        System.out.println(y);
    }
}
\end{lstlisting}


\AnswerBox

\begin{enumerate}[label=\bfseries Q\arabic*:]\itemsep10pt
        \item What is the result of compiling and running the above program?
    \end{enumerate}

\item [Program 10] \
\begin{lstlisting}
public class Tester {
    public static void main(String[] args) {
        if(true) {
            int x = 5;
            System.out.print(x);
        } else {
            ++x;
            System.out.print(x);
        }
    }
}
\end{lstlisting}

\AnswerBox

\begin{enumerate}[label=\bfseries Q\arabic*:]\itemsep10pt
                    \item What is the result of compiling and running the above code?
                            \end{enumerate}


\end{description}
    \end{document}

