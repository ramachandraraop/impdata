\documentclass[11pt,a4paper]{article}

\usepackage{graphicx}

\usepackage{listings}

\lstset{language=Java,numbers=left,numberstyle=\tiny,numbersep=10pt,showstringspaces=false}

\usepackage{array}

\usepackage{enumitem}



\def\AnswerBox{\fbox{\begin{minipage}{4in}\hfill\vspace{0.5in}\end{minipage}}}
\usepackage{fancyhdr}
\pagestyle{fancy}
\renewcommand{\headrulewidth}{0pt}
\rhead{\includegraphics[scale=.5]{../Images/TS-Logo.png}}



\begin{document}



\centerline{\huge{ \textbf{Multi-Threading}}}

\vspace{1pc}

\centerline{\huge{ \textbf{ Workbook}}}

\section*{Answer the Following}
\begin{enumerate}\itemsep10pt
\item What is multithreading?

\AnswerBox

\item  What are the two ways to create the thread?

\AnswerBox

\item  What is the signature of the constructor of a thread class?

\AnswerBox

\item  What are the methods used for Inter Thread communication and in which class these methods are defined?

\AnswerBox

\item  What is difference between wait() and sleep() method?

\AnswerBox

\item What is the purpose of synchronized block?

\AnswerBox

\item What are the states of thread?

\AnswerBox

\item What is meant by daemon thread? In java runtime, what is it's role?

\AnswerBox

\item When should we interrupt a thread?

\AnswerBox

\item What is the difference between preemptive scheduling and time slicing?

\AnswerBox

\item What is deadlock?

\AnswerBox

\end{enumerate}

\section*{State whether the following are True/False}
\begin{enumerate}
\item It is possible to start a thread twice. [   ]
\item We can make the user thread as daemon thread if thread is started. [  ]
\item The word synchronized can be used with only a method. [  ]
\item The suspend() method is used to teriminate a thread. [  ]
\item Garbage collector thread belongs to high priority. [   ]
\item Among wait(), notify(), notifyall() the wait() method only throws IOException. [   ]
\item  wait(), notify(), notifyall() are defined as final and can be called only from with in a synchronized method. [   ]
\end{enumerate}
\section*{Multiple Choice Questions}
\begin{enumerate}
\item Which of these method can be used to make the main thread to be executed last among all the threads?
\begin{enumerate}
\item stop()
\item sleep()
\item join()
\item call()
\end{enumerate}
\item Which of these method is used to find out that a thread is still running or not?
\begin{enumerate}
\item run()
\item alive()
\item isAlive()
\item checkRun()
\end{enumerate} 
\item What is the default value of priority variable MIN\_PRIORITY AND MAX\_PRIORITY?
\begin{enumerate}
\item 1 and 10 
\item 0 and 1
\item 1 and 256
\item 0 and 256
\end{enumerate}

\item Which of these method waits for the thread to treminate?
\begin{enumerate}
\item stop()
\item join()
\item isAlive()
\item sleep()
\end{enumerate}
\item Which of these interface is implemented by Thread class?
\begin{enumerate}
\item Runnable
\item Connections
\item Set
\item MapConnections
\end{enumerate}
\item Which of these method is used to begin the execution of a thread?
\begin{enumerate}
\item run()
\item start()
\item runThread()
\item startThread()
\end{enumerate}
\item Which of these statement is incorrect?
\begin{enumerate}
\item A thread can be formed by implementing Runnable interface only.
\item A thread can be formed by a class that extends Thread class.
\item start() method is used to begin execution of the thread.
\item run() method is used to begin execution of a thread before start() method in special cases.
\end{enumerate}
\item Which of these method of Thread class is used to find out the priority given to a thread?
\begin{enumerate}
\item get()
\item ThreadPriority()
\item getPriority()
\item getThreadPriority()
\end{enumerate}
\item Which of these are types of multitasking?
\begin{enumerate}
\item Process based
\item Thread based
\item Process and Thread based
\item None
\end{enumerate}
\end{enumerate}
\section*{Exercises}
\begin{itemize}
    \item Write the expected output, or compiler errors if any, for each of the following programs in the box provided below each program.
    \item Then execute the programs and check your answers.
    \item Then answer the questions given below.
\end{itemize}
\begin{description}
\item [Program 1] \
\begin{lstlisting}
public class Test implements Runnable {
    public static void main(String[] args) {
        Test test = new Test();
        Thread thread = new Thread(test);
        thread.start();
        thread.join();
        System.out.print(``main'');
    }
    public void run() {
        System.out.print(``run'');
    }
}
\end{lstlisting}

\AnswerBox

\begin{enumerate}[label=\bfseries Q\arabic*:]\itemsep10pt
\item What will be the output of the program?
\end{enumerate}

\item [Program 2] \
\begin{lstlisting}
public class Test{
    public static void main(String[] args) {
        Painter painter1 = new Painter();
        painter1.start();
        Painter painter2 = new Painter();
        painter2.start();
    }
}
class Painter implements Runnable {
    public void run() {
        System.out.println(``we are painting'');
    }
}
\end{lstlisting}

\AnswerBox

\begin{enumerate}[label=\bfseries Q\arabic*:]\itemsep10pt
        \item How many times the statement ``we are painting'' would be printed in this program?
    \end{enumerate}

\item [Program 3] \
\begin{lstlisting}
public class Test implements Runnable {
    Integer id;
    public static void main(String[] args) {
        new Thread(new Test()).start();
        new Thread(new Test()).start();
    }
    public void run() {
        press(id);
    }
    synchronized void press(Integer id) {
        System.out.print(id.intValue());
        System.out.print((++id).intValue());
    }
}
\end{lstlisting}

\AnswerBox

\begin{enumerate}[label=\bfseries Q\arabic*:]\itemsep10pt
\item What is the possible result of compiling and running the code?
\end{enumerate}

\item [Program 4] \
\begin{lstlisting}
class Swimmer implements Runnable {
    String name ;
    Swimmer(String name) {
        this.name = name;
    } 
    public void run() {
        new Test().swim(name);
    }
}
public class Test {
    public void swim(String name) {
        System.out.print(name);
        System.out.print(name);
    }
    public static void main(String[] args) {
        new Thread(new Swimmer("Tom")).start();
        new Thread(new Swimmer("Hanks")).start();
    }
}
\end{lstlisting}

\AnswerBox

\begin{enumerate}[label=\bfseries Q\arabic*:]\itemsep10pt
\item What are the possible outputs of running this program once as it is, and second with marking swim() synchronized?
    \end{enumerate}

\item [Program 5] \
\begin{lstlisting}
public class Tester extends Thread {
    public void run() {
        System.out.print(``run'');
    }
    public static void main(String[] args) {
        Tester thread = new Tester();
        new Thread(thread).start();
        new Thread(thread).start();
    }
}
\end{lstlisting}

\AnswerBox

\begin{enumerate}[label=\bfseries Q\arabic*:]\itemsep10pt
\item What is the result of compiling and running the code?
\end{enumerate}

\item [Program 6] \
\begin{lstlisting}
class Test {
    public static void main(String [] args) {
        printAll(args);
    }
    public static void printAll(String[] lines) {
        for(int i = 0; i < lines.length; i++) {
            System.out.println(lines[i]);
            Thread.currentThread().sleep(1000);
        }
    }
}
\end{lstlisting}

\AnswerBox

\begin{enumerate}[label=\bfseries Q\arabic*:]\itemsep10pt
        \item What will be the output of the above code?
    \end{enumerate}
\item [Program 7] \
\begin{lstlisting}
class MyThread extends Thread {
    public static void main(String [] args) {
        MyThread t = new MyThread();
        Thread x = new Thread(t);
        x.start();
    }
    public void run() {
        for(int i = 0; i < 3; ++i) {
            System.out.print(i + "..");
        }
    }
}
\end{lstlisting}

\AnswerBox

\begin{enumerate}[label=\bfseries Q\arabic*:]\itemsep10pt
 \item What will be the output of the above code?
\end{enumerate}
\end{description}


\end{document}

