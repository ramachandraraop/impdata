\documentclass[14pt]{beamer}
\title[BPT:VCS:02]{Getting Started - Git :: Session 02}
\author[TS]{TalentSprint}
\institute[L\&D]{Licensed To Skill}
\date{Version 1.0.4}
\usetheme{Madrid}

\usebackgroundtemplate{\includegraphics[width=\paperwidth]{TS-Logo.jpg}}
%\graphicspath{{/home/tsuser/Desktop/Latex_Slides/Images/}}
\graphicspath{{../../Images/}}

\begin{document}

\begin{frame}
  \titlepage
\end{frame}

\begin{frame}{Learning Objectives}
At the end of this session you will learn
  \begin{itemize}
  \item what git is?
  \item basic commands to work on local project
  \item how to create a project
  \item how to add, commit and check the status
  \item how to find the difference between two versions
  \end{itemize}
\end{frame}

\begin{frame}{What is Git?}
Git is an open source, distributed version control system designed for speed and efficiency.
\end{frame}

\begin{frame}{Features}
\begin{itemize}
\item everything is local
\pause
\item fast
\pause
\item every clone is a backup
\pause
\item work offline
\end{itemize}
\end{frame}

\begin{frame}{Install Git in Linux}
\begin{block} {Command}
\$ sudo apt-get install git
\end{block}
\end{frame}

\begin{frame}{Git :: init}
To create a project follow the below steps in Terminal:
\begin{enumerate}
\item \$ mkdir GitRepo
\item \$ cd GitRepo
\item \$ git init MyGitProject 
\end{enumerate}
\begin{block}{}
Empty git repository is initialized in the current directory.
\end{block}
\end{frame}

\begin{frame}{Git :: add}
Now add a file to project
\begin{enumerate}
\item \$ cd MyGitProject
\item \$ vim DispEvenNums.c
\item \$ git add DispEvenNums.c
\end{enumerate}
\begin{block}{}
DispEvenNums.c file is added to the project.
\end{block}
\end{frame}

\begin{frame}{Git :: status}
check the staus of file added to the project
\begin{itemize}
\item \$ git status
\end{itemize}
\begin{block}{}
It displays a message :-\\  changes to be committed:  new file: DispEvenNums.c 
\end{block}
\end{frame}

\begin{frame}{Git configuration : user settings}
now add the user details to project like username and email ID
\begin{enumerate}
\item git config user.name ``Bob''
\item git config user.email ``bob@example.com''
\end{enumerate}
configure this globally for all your git projects:
\begin{enumerate}
\item git config --global user.name ``Bob''
\item git config --global user.email ``bob@example.com''
\end{enumerate}
\end{frame}

\begin{frame}{Git :: commit}
now commit the added file to the project
\begin{itemize}
\item \$ git commit -m "Added New File"
\end{itemize}
\begin{block}{}
It displays number of files added, number of lines inserted and number of lines deleted.
\end{block}
\begin{block}{Note }
If we check the status again, it displays\\ nothing to commit (working directory clean) - as the project stage and file stage are same.
\end{block}
\end{frame}

\begin{frame}{Follow the below steps}
\begin{itemize}
\item \$ vim DispEvenNums.c
\item add a function to check the number is even or not and call the function
\item \$ git add DispEvenNums.c
\item \$ git commit -m "Added isEven Function"
\end{itemize}
\begin{block}{Note}
git maintains two different versions of DispEvenNums.c
\end{block}
\end{frame}

\begin{frame}{Git :: log}
\begin{itemize}
\item \$ git log
\begin{itemize}
\item shows a listing of commits including the corresponding details like commit, author, date and commit messages.
\end{itemize}
\item \$ git log - -abbrev-commit
\begin{itemize}
\item show only the first few characters of SHA-1 checksum instead of all 40.
\end{itemize}
\end{itemize}
\end{frame}

\begin{frame}{Git :: diff}
\begin{itemize}
\item \$ git diff a2a1eb3 068b9b9
\end{itemize}
\begin{block}{Note}
displays the difference between versions.
\end{block}
\end{frame}

\begin{frame}{Summary}
\begin{description}
\item[init] git init creates an empty git repository.
\item[add] add files changes in your working directory to your index.
\item[status] git status shows you how the working tree is different from the index and how the index is different from the last commit.
\item[commit] git commit takes the contents of the index and creates a new commit.
\item[log] shows a listing of commits on a branch including the corresponding details.
\item[diff] shows the difference from one commit compared to its parent or another.
\end{description}
\end{frame}

\end{document}
