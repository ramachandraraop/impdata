\documentclass[11pt,a4paper]{article}

\usepackage{graphicx}

\usepackage{listings}

\lstset{language=Java,numbers=left,numberstyle=\tiny,numbersep=10pt,showstringspaces=false}

\usepackage{array}

\usepackage{enumitem}



\def\AnswerBox{\fbox{\begin{minipage}{4in}\hfill\vspace{0.5in}\end{minipage}}}

\usepackage{fancyhdr}
\pagestyle{fancy}
\renewcommand{\headrulewidth}{0pt}
\rhead{\includegraphics[scale=.5]{../Images/TS-Logo.png}}


\begin{document}



\centerline{\huge{ \textbf{Object Oriented Programming(OOP) Basics}}}

\vspace{1pc}

\centerline{\huge{ \textbf{ Workbook}}}

\section*{Answer the Following}
\begin{enumerate}\itemsep10pt

\item What is class?

    \AnswerBox 

\item What is local variable?
    
    \AnswerBox

\item What is instance variable?
    
    \AnswerBox

\item What is encapsulation?
    
    \AnswerBox

\item What is object?
    
    \AnswerBox

\item How to define a constant variable in Java?

    \AnswerBox

\item What is the return type of the main() method?  \underline{\hspace{4cm}}
\item Why is the main() method declared static? \underline{\hspace{4cm}}
\item What is the arguement of main() method? \underline{\hspace{4cm}}


\end{enumerate}
\section*{State whether the following are True/False}
\begin{enumerate}\itemsep2pt
        \item A class consists of only local variables [   ]        
        \item A method declaration must always contain the access level. [    ]
        \item Encapsulation means the same thing as information hiding. [    ]
        \item If a Java source file is compiled successfully, the compilation produces one or more files with a java extension. [    ]
        \item A main method should be compulsorily declared in all java classes. [   ]       
        \item The order of public and static declaration matters in main  method. [ ]
\end{enumerate}
\section*{Multiple Choice Questions}

\begin{enumerate}
    \item Which of these operators is used to allocate memory for an object?
        \begin{enumerate}
            \item malloc
            \item alloc
            \item new
            \item give
        \end{enumerate}

\item Which of the following package stores all the simple data types in java?
 \begin{enumerate}
            \item lang
            \item java
            \item util
            \item java.packages
        \end{enumerate}
        
\item Which of these statement is incorrect?
\begin{enumerate}
\item All object of a class are allotted memory for the all the variables defined in the class.
\item If a function is defined public it can be accessed by object of other class by inheritation.
\item main() method must be made public.
\item All object of a class are allotted memory for the methods defined in the class.
\end{enumerate}

\item Which of the following is a valid declaration of an object of class Box?
\begin{enumerate}
\item Box obj = new Box();
\item Box obj = new Box;
\item obj = new Box();
\item new Box obj;
    \end{enumerate}
    
\item  Which of these class is superclass of all other classes?
    \begin{enumerate}
        \item Math
        \item Process
        \item System
        \item Object
    \end{enumerate}
    
\item  Which of these method of Object class can generate duplicate copy of the object on which it is called?
    \begin{enumerate}
        \item clone()
        \item copy()
        \item duplicate()
        \item dito()
    \end{enumerate}
\end{enumerate}
\section*{Exercises}
\begin{itemize}
    \item Write the expected output, or compiler errors if any, for each of the following programs in the box provided below each program.
    \item Then execute the programs and check your answers.
    \item Then answer the questions given below.
\end{itemize}
\begin{description}
\item [Program 1]\
\begin{lstlisting}
public final static void main(String[] args) {
    double d = 10.0 / -0;
    if(d == Double.POSITIVE_INFINITY)
        System.out.println(``Positive infinity'');
    else
        System.out.println(``Negative infinity'');
}

\end{lstlisting}

\AnswerBox

\begin{enumerate}[label=\bfseries Q\arabic*:]\itemsep10pt
\item What is the result of trying to compile and run the above code.
\end{enumerate}

\item [Program 2]\
\begin{lstlisting}
void aMethod() {
    float f = (1 / 4) * 10;
    int i = Math.round(f);
    System.out.println(i);
}
\end{lstlisting}

\AnswerBox

\begin{enumerate}[label=\bfseries Q\arabic*:]\itemsep10pt
        \item What is the result that will be printed out ?
    \end{enumerate}


\item [Program 3]\
    \begin{lstlisting}
     int i = 0XCAFE;
     boolean b = 0;
     char c    = `A';
     byte b    = 128;
     char c    = ``A'';
     \end{lstlisting}

\AnswerBox

\begin{enumerate}[label=\bfseries Q\arabic*:]\itemsep10pt
 \item Which of the above declarations are valid?
\end{enumerate}
\item [Program 4] \
\begin{lstlisting}
class Box {
    int width;
    int height;
    int length;
} 
class MainClass {
    public static void main(String args[]) {        
        Box obj = new Box();
        obj.width = 10;
        obj.height = 2;
        obj.length = 10;
        int result = obj.width * obj.height * obj.length; 
        System.out.print(result);
    } 
}
\end{lstlisting}

\AnswerBox

\begin{enumerate}[label=\bfseries Q\arabic*:]\itemsep10pt
\item What will be the output of the program?
\end{enumerate}

\item [Program 5] \
\begin{lstlisting}
public class Test {    
    public void main(String[] args) {  
        System.out.println(``Hello'' + args[0]); 
    } 
}
\end{lstlisting}

\AnswerBox

\begin{enumerate}[label=\bfseries Q\arabic*:]\itemsep10pt
\item What will be the output of the program, if this code is executed with the command line by passing argument as world.
\end{enumerate}

\item [Program 6] \
\begin{lstlisting}
public class Test {
    public static void main(String [] args) {
        int [] [] [] x = new int [3] [] [];
        int i, j;
        x[0] = new int[4][];
        x[1] = new int[2][];
        x[2] = new int[5][];
        for (i = 0; i < x.length; i++) {
            for (j = 0; j < x[i].length; j++) {
                x[i][j] = new int [i + j + 1];
                System.out.println(``size = '' + x[i][j].length);
            }
        }
    }
}
\end{lstlisting}

\AnswerBox

\begin{enumerate}[label=\bfseries Q\arabic*:]\itemsep10pt
\item In the given program, how many lines of output will be produced?
\end{enumerate}

\item [Program 7]\
\begin{lstlisting}
public class Test {
    public static void main(String[] args) {
        Float f = new Float(32D);    
        System.out.println(f);
    }
}
\end{lstlisting}

\AnswerBox

\begin{enumerate}[label=\bfseries Q\arabic*:]\itemsep10pt
        \item  What is the result of attempting to compile and run above code?
    \end{enumerate}

\item [Program 8] \
\begin{lstlisting}
public class ObjComp {
    public static void main(String [] args) {
        int result = 0;
        ObjComp oc = new ObjComp();
        Object o = oc;
        if (o == oc)  
            result = 1;
        if (o != oc)  
            result = result + 10;
        if (o.equals(oc))  
            result = result + 100;
        if (oc.equals(o))  
            result = result + 1000;
        System.out.println(``result = '' + result);
    }
}
\end{lstlisting}

\AnswerBox

\begin{enumerate}[label=\bfseries Q\arabic*:]\itemsep10pt
\item What will be the output of the program?
\end{enumerate}


\end{description}
\end{document}


