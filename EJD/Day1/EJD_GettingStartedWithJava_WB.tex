\documentclass[11pt,a4paper]{article}

\usepackage{graphicx}

\usepackage{listings}

\lstset{language=JAVA,numbers=left, numberstyle=\tiny, numbersep=10pt, showstringspaces=false, breaklines=true}

\usepackage{array}

\usepackage{enumitem}



\def\AnswerBox{\fbox{\begin{minipage}{4in}\hfill\vspace{0.5in}\end{minipage}}}
\usepackage{fancyhdr}
\pagestyle{fancy}
\renewcommand{\headrulewidth}{0pt}
\rhead{\includegraphics[scale=.5]{../Images/TS-Logo.png}}


\begin{document}



\centerline{\huge{ \textbf{ Getting Started with Java}}}

\vspace{1pc}

\centerline{\huge{ \textbf{ Workbook}}}

\section*{Answer the Following}
\begin{enumerate}\itemsep10pt

\item What is the most importanat feature of Java? \underline{\hspace{4cm}}
\item What is the base class for all classes? \underline{\hspace{4cm}}
\item Which package is imported by default?  \underline{\hspace{4cm}}
\item Why is the main method declared static?  \underline{\hspace{4cm}}
\item Why is String class considered immutable? 

    \AnswerBox

\item What do you mean by Object?
    
    \AnswerBox

\item Expain JVM?
    
    \AnswerBox

\item Define class.
    
    \AnswerBox

\item What is the difference between StringBuffer and StringBuilder class?
    
    \AnswerBox

\item What is jagged array?
    
    \AnswerBox

\end{enumerate}
\section*{State whether the following are True/False}
\begin{enumerate}\itemsep2pt
        \item JVM is platform independent. [         ]
        \item Java Supports pointers. [ ]
        \item Arrays are primitive data types. [    ]
        \item A source file can contain more than one class declaration. [    ]
        \item Is empty .java file name a valid source file name? [    ]
        \item If I don't provide any arguments on the command line, then the String array of Main method will be empty. [   ]
        
\end{enumerate}
\section*{Multiple Choice Questions}
\begin{enumerate}
    \item What is the range of data type short in Java?
        \begin{enumerate}
            \item -128 to 127
            \item -32768 to 32767
            \item -2147483648 to 2147483647
            \item None
        \end{enumerate}
    \item Which of these is an incorrect array declaration?
        \begin{enumerate}
            \item int arr[] = new int[5]
            \item int [] arr = new int[5]
            \item int arr[]
                arr = new int[5]
            \item int arr[] = int [5] new
        \end{enumerate}
    \item Which of the following loops will execute the body of loop even when condition controlling the loop is initially false?\begin{enumerate}
            \item do-while
            \item while
            \item for
            \item None
        \end{enumerate}
    \item Which of the following statements are incorrect?
        \begin{enumerate}
            \item String is a class.
            \item Strings in java are mutable.
            \item Every string is an object of class String.
            \item Java defines a peer class of String, called StringBuffer, which allows string to be altered.
                      \end{enumerate}
    \item Which of these keywords is used to make a class?
        \begin{enumerate}
            \item class
            \item struct
            \item int
            \item None
        \end{enumerate}
\end{enumerate}
\section*{Exercises}
\begin{itemize}
\item Write the expected output, or compiler errors if any, for each of the following programs in the box provided below each program.
\item Then execute the programs and check your answers.
\item Then answer the questions given below.
\end{itemize}
\begin{description}
\item [Program 1]\
\begin{lstlisting}
int i = 10;
int n = i++ % 5;
\end{lstlisting}

\AnswerBox

\begin{enumerate}[label=\bfseries Q\arabic*:]\itemsep10pt
\item What are the values of i and n after the code is executed?
\item What are the final values of i and n if instead of using the postfix increment operator (i++), you use the prefix version (++i))?
\end{enumerate}

\item [Program 2]\
\begin{lstlisting}
public class NumberHolder {
    public int anInt;
    public float aFloat;
}
\end{lstlisting}

\AnswerBox

\begin{enumerate}[label=\bfseries Q\arabic*:]\itemsep10pt
\item Write some code that creates an instance of the class `NumberHolder'?
\item Write some code that initializes its two member variables, and then displays the value of each member variable?
\end{enumerate}

\item [Program 3]\
\begin{lstlisting}
public class IdentifyMyParts {
    public static int x = 7; 
    public int y = 3; 
}
\end{lstlisting}

\AnswerBox

\begin{enumerate}[label=\bfseries Q\arabic*:]\itemsep10pt
\item How many class variables does the \texttt{IdentifyMyParts} class contain? What are their names?

\item How many instance variables does the \texttt{IdentifyMyParts} class contain? What are their names?

\item What is the output from the following code:
\begin{verbatim}
IdentifyMyParts a = new IdentifyMyParts();
IdentifyMyParts b = new IdentifyMyParts();
a.y = 5;
b.y = 6;
a.x = 1;
b.x = 2;
System.out.println("a.y = " + a.y);
System.out.println("b.y = " + b.y);
System.out.println("a.x = " + a.x);
System.out.println("b.x = " + b.x);
\end{verbatim}
\end{enumerate}

\item [Program 4]\
\begin{lstlisting}
public class EqualsTest {
    public static void main(String[] args) {
        // code goes here
    }
}
\end{lstlisting}

\AnswerBox

\begin{enumerate}[label=\bfseries Q\arabic*:]\itemsep10pt
\item What if the \texttt{static} modifier is removed from the signature of the main method?
\item What if I do not provide the String array as the argument to the method?
\item What happend if line 2 changed as \texttt{static public void main(String args[]) \{ }
\end{enumerate}

\item [Program 5]\
\begin{lstlisting}
int key = 1;
switch (key + 1) {
    case 1:
        System.out.println("Cake");
        break;
    case 2:
        System.out.println("Pie");
        break;
    case 3:
        System.out.println("Ice cream");
    case 4:
        System.out.println("Cookies");
        break;
    default:
        System.out.println("Diet time");
}
\end{lstlisting}

\AnswerBox

\begin{enumerate}[label=\bfseries Q\arabic*:]\itemsep10pt
\item What would be the output of the above code?
\item What would be the output if the value of key is 3?
\item What would be the output if the value of key is 5?
\end{enumerate}

\item [Program 6]\
\begin{lstlisting}
for (int n = 1; n <= 5; n++) {
    if (n == 3)
        break;
    System.out.println(``Hello'');
}
System.out.println(``After the loop'');
\end{lstlisting}
\begin{enumerate}[label=\bfseries Q\arabic*:]\itemsep10pt
\item What would be the output of the above code?
\item What would be the output if line 3 is changed to \texttt{System.exit(0);}?
\end{enumerate}
\item [Program 7]\
\begin{lstlisting}
String[] skiResorts = {``Whistler Blackcomb'', ``Squaw Valley'', ``Brighton'', ``Snowmass'', ``Sun Valley'', ``Taos'' };
\end{lstlisting}
\begin{enumerate}[label=\bfseries Q\arabic*:]\itemsep10pt
\item What is the index of \texttt{Brighton} in the following array?
\item What is the value of the expression \texttt{skiResorts.length}?
\item What is the index of the last item in the array?
\item What is the value of the expression \texttt{skiResorts[4]}?
\end{enumerate}

\item [Program 8]\
\begin{lstlisting}
String hannah = ``Did Hannah see bees? Hannah did.'';
\end{lstlisting}

\AnswerBox

\begin{enumerate}[label=\bfseries Q\arabic*:]\itemsep10pt
\item What is the value displayed by the expression \texttt{hannah.length()}?
\item What is the value returned by the method call \texttt{hannah.charAt(12)}?
\item What is the value returned by the method call \texttt{hannah.substring(9, 12)}?
\end{enumerate}

\item [Program 9]\
\begin{lstlisting}
public class BasicsDemo {
    public static void main(String[] args) {
        int sum = 0;
        for (int current = 1; current <= 10; current++) {
            sum += current;
        }
        System.out.println(``Sum = '' + sum);
    }
}
\end{lstlisting}

\AnswerBox

\begin{enumerate}[label=\bfseries Q\arabic*:]\itemsep10pt
\item What is the name of each variable declared in the program? Remember that method parameters are also variables.
\item What is the data type of each variable?
\item What is the scope of each variable?
\end{enumerate}

\item [Program 10]\
\begin{lstlisting}
public class CheckSign {
    public static void main(String args[]) {
        int number = 7;
        boolean isPositive = (number > 0);
        if (number > 0);
            number = -100;
        if (isPositive)
            System.out.println("Positive.");
        else
            System.out.println("Not positive.");
        System.out.println(number);
    }
}
\end{lstlisting}

\AnswerBox

\begin{enumerate}[label=\bfseries Q\arabic*:]\itemsep10pt
\item What will be the output of the above code?
\item What will be the output if line 4 is changed to \texttt{boolean isPositive = false;}?
\item What will happen if line 3 is changed to \texttt{int number;}? 
\end{enumerate}
\end{description}
\end{document}



http://www.math.uni-hamburg.de/doc/java/tutorial/java/TOC.html#data
