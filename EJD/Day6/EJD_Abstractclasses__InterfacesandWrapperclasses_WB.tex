\documentclass[11pt,a4paper]{article}

\usepackage{graphicx}

\usepackage{listings}

\lstset{language=Java,numbers=left,numberstyle=\tiny,numbersep=10pt,showstringspaces=false}

\usepackage{array}

\usepackage{enumitem}



\def\AnswerBox{\fbox{\begin{minipage}{4in}\hfill\vspace{0.5in}\end{minipage}}}
\usepackage{fancyhdr}
\pagestyle{fancy}
\renewcommand{\headrulewidth}{0pt}
\rhead{\includegraphics[scale=.5]{../Images/TS-Logo.png}}



\begin{document}



\centerline{\huge{ \textbf{Abstract classes, Interfaces and Wrapper classes}}}

\vspace{1pc}

\centerline{\huge{ \textbf{ Workbook}}}

\section*{Answer the Following}
\begin{enumerate}\itemsep10pt

\item An abstract class\underline{\hspace{3cm}}
\item An abstract method\underline{\hspace{3cm}}
\item An abstract class can contain\underline{\hspace{3cm}}
\item How many interfaces can a class implement?\underline{\hspace{3cm}}
\item The methods of interface are\underline{\hspace{3cm}}by default.
\item An interface can be declared using\underline{\hspace{3cm}}keyword.
\item To implement an interface\underline{\hspace{3cm}}keyword is used.
\item The variables of interfaces are\underline{\hspace{3cm}}by default
\item Which keyword is used to define an abstract class?\underline{\hspace{3cm}}
\item What is autoboxing?

\AnswerBox

\item What are the different methods to parse Strings in java? \underline{\hspace{4cm}}
\end{enumerate}
\section*{State whether the following are True/False}
\begin{enumerate}\itemsep2pt
        \item A class that implements an interface must implement all the methods declared in the interface. [\hspace{1cm}]
        \item An abstract class can only contain abstract methods. [\hspace{1cm}]
        \item Interfaces are defined using the reserved word \texttt{interface} and the reserved word \texttt{class}.[\hspace{1cm}]
        \item Using the mechanism of inheritance, every public member of the class Object can be overridden and/or invoked by every object of any class type.[\hspace{1cm}]
	\item You can instantiate an object of a subclass of an abstract class, but only if the subclass gives the definitions of all the abstract methods of the superclass. [\hspace{1cm}]
	\item An abstract method is a method that has only the heading with no body. [\hspace{1cm}]
	\item An interface is a class that contains only abstract methods and/or named constants. [\hspace{1cm}]
	\item  A class can extend only one class and can implements more than one interface. [\hspace{1cm}]
\end{enumerate}
\section*{Multiple Choice Questions}
\begin{enumerate}
    \item Which of the following declares an abstract method in an abstract Java class?
        \begin{enumerate}
            \item public abstract method();
            \item public abstract void method();
            \item public void abstract Method();
            \item public void method() {}
            \item public abstract void method() {} 
        \end{enumerate}
    \item Which of the following statements regarding abstract methods are true?
        \begin{enumerate}
            \item An abstract class can have instances created using the constructor of the abstract class.
	    \item An abstract class can be extended.
	    \item A subclass of a non-abstract superclass can be abstract.
	    \item A subclass can override a concrete method in a superclass to declare it abstract.
	    \item An abstract class can be used as a data type
        \end{enumerate}
    \item Suppose A is an abstract class, B is a concrete subclass of A, and both A and B have a default constructor. Which of the following is correct?
        \begin{enumerate}
            \item A a = new A();
	    \item A a = new B();
	    \item B b = new A();
	    \item B b = new B();
        \end{enumerate}
    \item Which of the following is a correct interface?
        \begin{enumerate}
            \item interface A { void print() { }; }
	    \item abstract interface A { print(); }
	    \item abstract interface A { abstract void print() { };}
	    \item interface A { void print();}
        \end{enumerate}
    
    \item Which of these packages contains abstract keyword?
    \begin{enumerate}    
        \item java.lang
	\item java.util
	\item java.io
	\item java.system
    \end{enumerate}
    
    \item Which of these access specifiers can be used for an interface?
    \begin{enumerate}
	\item public
	\item protected
	\item private
	\item All of the mentioned
    \end{enumerate}
    \item Which of these keywords is used by a class to use an interface defined previously?
    \begin{enumerate}
	\item import
	\item Import
        \item implements
	\item Implements
    \end{enumerate}
    \item When class is declared as abstract, then
    \begin{enumerate}
	\item Its object can not be created
	\item Its subclass can not be created
	\item It can not inherit any class
	\item It can not have method
    \end{enumerate}
    \item A class can implements
    \begin{enumerate}
	\item only one
	\item one or more than one
	\item maximum two
	\item minimum two
    \end{enumerate}
    \item A method implementation of an interface must be declared as
    \begin{enumerate}	
	\item private
	\item default access
	\item public
	\item protected
    \end{enumerate}
    \item An interface contains
    \begin{enumerate}
	\item The method definitions
	\item The method declaration
   	\item Method declaration and definition
	\item none of these
    \end{enumerate}
    \item A class implements an interface but does not override all the methods of interface then
    \begin{enumerate}
    	\item It should be declared as abstract class
	\item It should be declared as final class
	\item It must override all the methods of interface
	\item none of these
    \end{enumerate}
    \item When class is declared as abstract, then
    \begin{enumerate}
    	\item Its object can not be created
	\item Its subclass can not be created
	\item can not inherit any class
	\item It can not have methods
    \end{enumerate}
\item Which of these is a super class of wrappers Long, Character and Integer?
     \begin{enumerate}
         \item Long
         \item digits
         \item Float
         \item Number
     \end{enumerate}
 \item Which of the following is method of wrapper Integer for converting the value of an object into byte?
     \begin{enumerate}
         \item bytevalue()
         \item byte bytevalue()
         \item Bytevalue()
         \item Byte Bytevalue().
     \end{enumerate}
 \item What is the function of the parseInt() method?
     \begin{enumerate}
         \item Parses a datatype and stores in an integer
         \item Parses a string and returns an integer
         \item Parses an integer and returns a string
         \item none
     \end{enumerate}
 
\end{enumerate}
\section*{Exercises}
\begin{itemize}
    \item Write the expected output, or compiler errors if any, for each of the following programs in the box provided below each program.
    \item Then execute the programs and check your answers.
    \item Then answer the questions given below.
\end{itemize}
\begin{description}
\item [Program 1] \
\begin{lstlisting}
abstract interface Bendable {
    final int x = 2009; 
    void method1() ; 
    public static class Angle {

    }
}
\end{lstlisting}

\AnswerBox

\begin{enumerate}[label=\bfseries Q\arabic*:]\itemsep10pt
\item Is the above declaration for interface Bendable correct and free of compilation error?
\end{enumerate}

\item [Program 2] \
\begin{lstlisting}
abstract class AirPlane { 
    abstract void fly(); 
    void land() {
        System.out.print(``Landing..'');
    }
}
class AirJet extends AirPlane { 
    AirJet() {
        super(); 
    }
    void fly() {
        System.out.print(``Flying..'');
    }
    abstract void land() ;
}
\end{lstlisting}

\AnswerBox

\begin{enumerate}[label=\bfseries Q\arabic*:]\itemsep10pt
\item The above code contains a compilation error , what can be done to fix this error - independently?
\end{enumerate}

\item [Program 3] \
\begin{lstlisting}
public abstract interface Bouncable {
    int num1 = 0;
    public int num2 = 1;
    public static int num3 = 2;
    public static transient int num4 = 3;
    public final int num5 = 3;
    public static final int num6 = 3;
}
\end{lstlisting}

\AnswerBox

\begin{enumerate}[label=\bfseries Q\arabic*:]\itemsep10pt
\item Which of the variables is incorrectly declared?
\end{enumerate}

\item [Program 4] \
\begin{lstlisting}
interface Movable {
    public abstract void m1(); 
    void m2();
    public void m3(); 
    abstract void m4(); 
}
class Chair implements Movable { 
    public void m1() {
    
    } 
    void m2() {
    
    } 
}
\end{lstlisting}

\AnswerBox

\begin{enumerate}[label=\bfseries Q\arabic*:]\itemsep10pt
\item To resolve the compilation error(s) in the above code, what can be done independently? 
\end{enumerate}

\item [Program 5] \
\begin{lstlisting}
abstract class AirPlane {
    abstract void fly();
    void land() { 
        System.out.print(``Landing'');
    }
}
class AirJet extends AirPlane {
    AirJet() {
        super();
    }
    void fly() {
        System.out.print(``Flying'');
    }
}
\end{lstlisting}

\AnswerBox

\begin{enumerate}[label=\bfseries Q\arabic*:]\itemsep10pt
\item Will the above code compile correctly?
    \end{enumerate}


\item [Program 6] \
\begin{lstlisting}
interface Count {
    short counter = 0;
    void countUp();
}
public class TestCount implements Count {
    public static void main(String [] args) {
        TestCount t = new TestCount();
        t.countUp();
    }
    public void countUp() {
        for (int x = 6; x > counter; x--, ++counter) {
            System.out.print(`` '' + counter);
        }
    }
}
\end{lstlisting}

\AnswerBox

\begin{enumerate}[label=\bfseries Q\arabic*:]\itemsep10pt
\item What will be the output of the above program?
\end{enumerate}


\item [Program 7] \
\begin{lstlisting}
abstract class A {
    int num1;
    abstract void display();
}    
class B extends A {
    int num2;
    void display() {
    System.out.println(num2);
    }
}
class Abstract_demo {
    public static void main(String args[]) {
        B obj = new B();
        obj.num2 = 2;
        obj.display();    
    }
}
\end{lstlisting}

\AnswerBox

\begin{enumerate}[label=\bfseries Q\arabic*:]\itemsep10pt
\item What will be the output of the above program?
\end{enumerate}

\item [Program 8] \
\begin{lstlisting}
public class Tester {
    public static void main(String[] args) {
        Number x = 12; // Line 5
        Number y = (Long) x; // Line 6
        System.out.print(x+`` ''+y); // Line 7
    }
}
\end{lstlisting}

\AnswerBox

\begin{enumerate}[label=\bfseries Q\arabic*:]\itemsep10pt
\item Given that Long and Integer extend Number, what is the result of compiling and running the code?
\end{enumerate}
\end{description}
\end{document}

