\documentclass[11pt,a4paper]{article}
\usepackage{graphicx}
\usepackage{listings}
\lstset{language=Java,numbers=left,numberstyle=\tiny,numbersep=10pt,showstringspaces=false}
\usepackage{array}
\usepackage{enumitem}
\def\AnswerBox{\fbox{\begin{minipage}{4in}\hfill\vspace{0.5in}\end{minipage}}}
\usepackage{fancyhdr}
\pagestyle{fancy}
\renewcommand{\headrulewidth}{0pt}
\rhead{\includegraphics[scale=.5]{../../TS-Logo.png}}
\begin{document}
\centerline{\huge{ \textbf{First Step Towards Programming}}}
\vspace{1pc}
\centerline{\Large{\textbf{Workbook}}}
\vspace{1pc}
\section*{Answer the following}
\begin{enumerate}
\item List four primitive data types
\begin{itemize}
\item \  
\item \ 
\item \ 
\item \ 
\end{itemize}
\item Every statement in a program must be terminated by \underline{\hspace{5cm}}
\item Write a statement to declare a variable $``p\_age"$ of \emph{int} type and assign a value 20? \underline{\hspace{5cm}}
\item What is the name of the java file containg the class name as \\ \textbf{SampleProgram}? \underline{\hspace{5cm}}
\item Predict the output for the below code?

\begin{lstlisting}
 public class MyMain {
     public static void main(String args[]) {
         int a = 10;
         int b = 20;
         System.out.println(a + b);  
    }  
}
\end{lstlisting}
\AnswerBox


\item Predict the output for the below code?

\begin{lstlisting}
 public class Test {
    public static void main(String args[]) {
       int a = 30;
       int b = 70;
       System.out.println("Result " + a + b);  
    }   
}
\end{lstlisting}
\AnswerBox
\end{enumerate}
\end{document}
