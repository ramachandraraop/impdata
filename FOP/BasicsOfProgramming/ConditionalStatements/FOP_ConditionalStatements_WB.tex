\documentclass[11pt,a4paper]{article}
\usepackage{graphicx}
\usepackage{listings}
\lstset{language=C,numbers=left,numberstyle=\tiny,numbersep=10pt,showstringspaces=false}
\usepackage{array}
\usepackage{enumitem}
\def\AnswerBox{\fbox{\begin{minipage}{4in}\hfill\vspace{0.5in}\end{minipage}}}
\usepackage{fancyhdr}
\pagestyle{fancy}
\renewcommand{\headrulewidth}{0pt}
\rhead{\includegraphics[scale=.5]{../../TS-Logo.png}}
\begin{document}
\centerline{\huge{ \textbf{Conditional Statements}}}
\vspace{1pc}
\centerline{\Large{ \textbf{Workbook}}}
\section*{Answer the following}
\begin{enumerate}
\item What would be the output of the below code?
\begin{lstlisting}
int x = 5;
if (x > 3)
    x = x - 2;
else
    x = x + 2;
\end{lstlisting}
\AnswerBox
\item What would be the output of the below code?
\begin{lstlisting}
int a = 10, b = 5;
if (a > b);
    System.out.println(``A is Greater\n'');
else
    System.out.println(``B is Greater\n");
\end{lstlisting}
\AnswerBox

\item What will be the output of the below code?
\begin{lstlisting}
int x = 5;
if (x < 1)
    printf(``value is -ve\n'');
if (x == 5)
    printf(``value is 5\n'');
else
    printf(``value is not equal to 5\n'');
\end{lstlisting}

\AnswerBox

\item What would be the output of the below code when executed\\
    java SampleCode1 40 20
\begin{lstlisting} 
class SampleCode1 {
    public static void main(String[] args) {
        int num1 = Integer.parseInt(args[0]);
        int num2 = Integer.parseInt(args[1]);
        if (num1 > num2)
            System.out.println(num1);
        else
            System.out.println(num2);
    }
}
\end{lstlisting} 
\AnswerBox

\item Write an expression in line 4, to get the correct output.
\begin{lstlisting} 
class SampleCode2 {
    public static void main(String[] args) {
        int val1 = Integer.parseInt(args[0]);
    if (_______)
        System.out.println(val1 + " is Even");
    else
        System.out.println(val1 + " is Odd");
}
\end{lstlisting} 
\AnswerBox

\item What would be the output of the below code when executed\\
    java SampleCode3 44 true
\begin{lstlisting}
class SampleCode3 {
    public static void main(String[] args) {
        int age = Integer.parseInt(args[0]);
        boolean HasVID = Boolean.parseBoolean(args[1]);
        if (age >= 18) {
            if (HasVID)
                System.out.println(``Eligible to Vote'');
            else
                System.out.println(``Eligible to Apply'');
        } else {
            System.out.println(``Not Eligible'');
        }
    }
}
\end{lstlisting}
\AnswerBox
\item What would be the output of the below code?
\begin{lstlisting}
class SampleCode4 {
     public static void main(String[] args) {
         int w = 4, q = 3;
         if (q > 5)
             if (w == 7)
                 w = 3;
             else
                 w = 2;
         else
             if (w > 3)
                 w = 1;
             else
                 w = 0;
         System.out.println(w);
     }
 }
\end{lstlisting}
\AnswerBox

\item What would be the output of the below code when executed\\
    java SampleCode5 85 77 87
\begin{lstlisting}
class SampleCode5 {
    public static void main(String[] args) {
         int m1 = Integer.parseInt(args[0]);
         int m2 = Integer.parseInt(args[1]);
         int m3 = Integer.parseInt(args[2]);
         int avg = (m1 + m2 + m3) / 3;
         if (m1 >= 35 && m2 >= 35 && m3 >= 35) {
             if (avg >= 75)
                 System.out.println(``Passed in Distinction'');
             else
                 System.out.println(``Passed'');
         } else {
              System.out.println(``Failed'');
         }
     }
 }
\end{lstlisting}
\AnswerBox

\item What would be the output when we execute below code?
\begin{lstlisting}
class SampleCode6 {
    public static void main(String[] args) {
        int a = 14, b = 0;
        if (a < 5)
            b = 4;
        else if (a < 10)
            b = 3;
        else if (a > 5)
            b = 2;
        else
            b = 1;
        System.out.println(b);
    }
}
\end{lstlisting}
\AnswerBox

\item What would be the output of the below code when executed\\
    java SampleCode7 58
\begin{lstlisting} 
class SampleCode7 {
    public static void main(String[] args) {
        int avg = Integer.parseInt(args[0]);
        if (avg < 40)
            System.out.println(``Failed'');
        else if (avg >= 40)
            System.out.println(``Passed in Third Division'');
        else if (avg >= 50)
            System.out.println(``Passed in Second Division'');
        else if (avg >= 60)
            System.out.println(``Passed in First Division'');
    }
}
\end{lstlisting} 
\AnswerBox

\item Write an expression in the space provided to find the number of days in a given month?\\
    java SampleCode8 4
\begin{lstlisting} 
class SampleCode8 {
    public static void main(String[] args) {
        int mon = Integer.parseInt(args[0]);
        if (_____________________)
            System.out.println(``31 Days'');
        else if (_____________________)
            System.out.println(``28 Days'');
        else
            System.out.println(``30 Days'');
    }
}
\end{lstlisting} 

\AnswerBox

\item What would be the output of the below code when executed\\
    java SampleCode9 22
\begin{lstlisting}
class SampleCode9 {
    public static void main(String[] args) {
        int age = Integer.parseInt(args[0]);                         
        if (age < 15) {                  
            System.out.println(``Child''); 
        } else if (age >=15 && age <= 19) {             
            System.out.println(``Teenager'');       
        } else if (age > 19 && age <= 25) {
            System.out.println(``Young'');     
        } else if (age > 25 && age <= 40) {
            System.out.println("Adult");
        } else if (age > 40 && age <= 60) {
            System.out.println("Middle Age");
        } else {
            System.out.println("Old Age");
        }
    }
}
\end{lstlisting}
\AnswerBox

\item What would be the output of the below code when executed\\
    java SampleCodeA 160
\begin{lstlisting}
class SampleCodeA {
    public static void main(String[] args) {
        int unit = Integer.parseInt(args[0]);
        int total = 0;
        if (unit <= 100) {
            total = unit * 4;
        } else if (unit > 100 && unit <= 300) {
            total = unit * 4.5;
        } else if (unit > 300 && unit <= 500) {
            total = unit * 4.75;
        } else {
            total = unit * 5;
        }
        System.out.println(``Total: '' + total);
    }
}
\end{lstlisting}
\AnswerBox

\item What would be the output for the below code when executed\\
    java SampleCodeB 150
\begin{lstlisting}
class SampleCodeB {
    public static void main(String[] args) {
        int x = Integer.parseInt(args[0]);
        if (x >= 100)
            x += 50;
        else if (x >=50)
            x += 20;
        else if (x > 0)
            x += 10;
        else 
            x = -x;
        System.out.println(x);
    }
}
\end{lstlisting}
\AnswerBox

\end{enumerate}
\end{document}
