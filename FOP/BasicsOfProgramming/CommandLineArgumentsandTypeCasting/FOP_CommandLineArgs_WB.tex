\documentclass[11pt,a4paper]{article}
\usepackage{graphicx}
\usepackage{listings}
\lstset{language=Java,numbers=left,numberstyle=\tiny,numbersep=10pt,showstringspaces=false}
\usepackage{array}
\usepackage{enumitem}
\def\AnswerBox{\fbox{\begin{minipage}{4in}\hfill\vspace{0.5in}\end{minipage}}}
\usepackage{fancyhdr}
\pagestyle{fancy}
\renewcommand{\headrulewidth}{0pt}
\rhead{\includegraphics[scale=.5]{../../TS-Logo.png}}
\begin{document}
\centerline{\huge{ \textbf{Command Line Arguments}}}
\vspace{1pc}
\centerline{\Large{ \textbf{Workbook}}}
\section*{Answer the following}

\begin{enumerate}
\item The Operator used to compare two values for equality is \underline{\hspace{5cm}}.
\item Let \lstinline!int a = 345; int r = a % 10 * a / 100;!\\
What will be the value of r? \underline{\hspace{5cm}}.
\item What is the default value of boolean variable \underline{\hspace{5cm}}.   
\item Let \lstinline!String str = ``10'';! write a statement to convert string to number and store to \lstinline!int n;!\\
\underline{\hspace{5cm}}.
\item What will be the output of the below code when executed\\
    java SampleCode1 30 50
\begin{lstlisting}
class SampleCode1 {
    public static void main(String[] args) {
        int a = Integer.parseInt(args[0]);
        int b = Integer.parseInt(args[1]);
        int c = a + b;
        System.out.println(``Result '' + c);
    }
}
\end{lstlisting}
\AnswerBox

\item Write a command to execute the below code and output?
\begin{lstlisting}
class SampleCode2 {
    public static void main(String[] args) {
        int i = Integer.parseInt(args[0]);
        int j = Integer.parseInt(arsg[1]);
        int k = Integer.parseInt(args[2]);
        int res = (i + j) * k;
        System.out.println(res);
    }
}
\end{lstlisting}
\AnswerBox
\item What will be the output of the below code when executed?\\
    java SampleCode3 ``Hi'' ``All''
\begin{lstlisting}
class SampleCode3 {
    public static void main(String[] args) {
        String x = args[0];
        String y = args[1];
        System.out.println(x + y);
    }
}
\end{lstlisting}
\AnswerBox
\end{enumerate}
\end{document}
