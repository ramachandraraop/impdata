\documentclass[11pt,a4paper]{article}
\usepackage{graphicx}
\usepackage{listings}
\lstset{language=Java,numbers=left,numberstyle=\tiny,numbersep=10pt,showstringspaces=false}
\usepackage{array}
\usepackage{enumitem}
\def\AnswerBox{\fbox{\begin{minipage}{4in}\hfill\vspace{0.5in}\end{minipage}}}
\def\Answerbox{\fbox{\begin{minipage}{4in}\hfill\vspace{1.0in}\end{minipage}}}
\usepackage{fancyhdr}
\pagestyle{fancy}
\renewcommand{\headrulewidth}{0pt}
\rhead{\includegraphics[scale=.5]{TS-Logo.png}}
\begin{document}
\centerline{\huge{ \textbf{Writing Modular Programs}}}
\vspace{1pc}
\centerline{\Large{ \textbf{Workbook}}}
\section*{Answer the following}

\begin{enumerate}
\item What would be the output of the below code?
\begin{lstlisting}
public class Method_A {
    public static int methodA(int x, int y) {
        return (x + y);
    }
    public static void main(String args[]) {
        System.out.println(methodA(10, 20); 
    }  
}
\end{lstlisting}
\AnswerBox

\item Explain your understanding about the below function?
\begin{lstlisting}
public int fun1(int num) {
    int s = 0;
    while (num > 0) {
        int x = num % 10;
        if (x % 2 == 0)
            s += x;
        num /= 10;
    }
    return s;
}
\end{lstlisting}
\AnswerBox

\item Explain your understanding about the below function, and what it return if the argument value is 56?
\begin{lstlisting}
public boolean xyz(int n) {
    if (n % 2 != 0)
        return true;
    return false;
}
\end{lstlisting}
\AnswerBox

\item What would be the output if the argument value is 220?
\begin{lstlisting}
public void displayFactors(int num) {
    System.out.println(``1'');
    for (int i = 2; i <= num / 2; i++) {
        if (num % i == 0) 
        System.out.println(i);
    }
    System.out.println(num);
}
\end{lstlisting}
\AnswerBox

\item What would be the output when called the below method by passing 67 as argument?
\begin{lstlisting}
public char getGrade(int avg) {
    if (avg >= 80)
        return 'A';
    if (avg >= 60)
        return 'B';
    if (avg >= 40)
        return 'C';
    return 'F';
}
\end{lstlisting}
\AnswerBox

\item Write the body of the method, which returns the minimum value?

\begin{lstlisting}[numbers=none]
public static int getMin(int a, int b, int c) {
\end{lstlisting}
\Answerbox
\begin{lstlisting}[numbers=none]
}
\end{lstlisting}
\end{enumerate}
\end{document}
