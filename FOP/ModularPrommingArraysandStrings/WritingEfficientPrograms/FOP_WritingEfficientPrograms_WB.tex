\documentclass[11pt,a4paper]{article}
\usepackage{graphicx}
\usepackage{listings}
\lstset{language=Java,numbers=left,numberstyle=\tiny,numbersep=10pt,showstringspaces=false, breaklines=true}
\usepackage{array}
\usepackage{enumitem}
\def\AnswerBox{\fbox{\begin{minipage}{4in}\hfill\vspace{0.5in}\end{minipage}}}
\usepackage{fancyhdr}
\pagestyle{fancy}
\renewcommand{\headrulewidth}{0pt}
\rhead{\includegraphics[scale=.5]{TS-Logo.png}}
\begin{document}
\centerline{\huge{ \textbf{Writing Efficient Programs}}}
\vspace{1pc}
\centerline{\Large{ \textbf{Workbook}}}
\section*{Answer the following}
\begin{enumerate}
\item A program to display the sum of factors of a number.
\begin{description}
\item [Solution-A] Write your understanding about the below code?
\begin{lstlisting}
class SolutionA {
    public static void main(String[] args) {
        int num = 220;
        int i = 0;
        int sumOfFacts = 0;
        while (i <= num / 2) {
            if (num % i == 0) {
                sumOfFacts += i;
            }
            i++;    
        }
        System.out.println(sumOfFacts);
    }
}
\end{lstlisting}
\AnswerBox
\item [Solution-B] Write your understanding about the below code?
\begin{lstlisting}
class SolutionB {
    public static void main(String[] args) {
        int num = 220;
        int i = 0;
        int sumOfFacts = -num;
        while (i * i < num) {
            if (num % i == 0) {
                sumOfFacts += i;
                sumOfFacts += num / i;
            }
            i++;
        }
        System.out.println(sumOfFacts);
    }
}
\end{lstlisting}
\AnswerBox
\item [Observe] List out the differences between SolutionA and SolutionB?

    
\AnswerBox
\end{description}
\item A program to display the greatest of three numbers?
\begin{description}
\item [Solution-A] Wirte your understanding about the below code?
\begin{lstlisting}
class SolutionA {
    public static void main(String[] args) {
        int a = Integer.parseInt(args[0]);
        int b = Integer.parseInt(args[1]);
        int c = Integer.parseInt(args[2]);
        if (a > b && a > c) {
            System.out.println(a);
        } else if (b > c) {
            System.out.println(b);
        } else {
            System.out.println(c);
        }
    }
}
\end{lstlisting}
\AnswerBox
\item [Solution-B] Write your understandinf about the below code?
\begin{lstlisting}
class SolutionB {
    public static void main(String[] args) {
        int a = Integer.parseInt(arsg[0]);
        int b = Integer.parseInt(args[1]);
        int c = Integer.parseInt(args[2]);
        int max = a;
        if (b > max)
            max = b;
        if (c > max)
            max = c;
        System.out.println(max);
    }
}
\end{lstlisting}
\AnswerBox

\item [Observe] List out the differences between SolutionA and SolutionB?

\AnswerBox

\end{description}
\end{enumerate}

\end{document}

