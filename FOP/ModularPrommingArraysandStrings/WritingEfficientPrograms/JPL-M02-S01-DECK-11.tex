\documentclass[14pt]{beamer}
\title[JPL:Java:02]{JPL :: Writing Efficient Programs}
\author[TS]{TalentSprint}
\institute[L\&D]{Licensed To Skill}
\date{Version 1.0.4}
\usefonttheme{serif}
\usecolortheme{orchid}
\usepackage{bookman}
\usepackage{hyperref}
\usepackage[T1]{fontenc}
\usepackage{graphicx}
\usepackage{listings}
\graphicspath{{../../Images/}}
\usepackage{tikz}
\beamertemplateballitem
\usebackgroundtemplate{\includegraphics[width=\paperwidth]{TS-XP-Logo.jpg}}
\lstset{language=Java,numbers=left, numberstyle=\tiny, basicstyle=\footnotesize, numbersep=10pt, showstringspaces=false, breaklines=true,keepspaces=true, columns=flexible}
\begin{document}

\begin{frame}
  \titlepage
\end{frame}

\begin{frame}{Learning Objectives}
The content in this presentation is aimed at teaching  learners to:
  \begin{itemize}
  \item Provide alternative solutions to the same problem
  \item Optimize solutions to problems
  \item Write elegant and structured code for problems
  \item Write programs to problems by decomposing functionality into methods and using the methods
  \end{itemize}
\end{frame}

\begin{frame}[fragile]{Writing Efficient Programs}
Let's re-look at the solution to find Prime Numbers
\begin{block}{Solution}
 \begin{lstlisting}[numbers=none]
Read the number into n.
for i from 2 to n-1, 
    if n % i = 0, then print ("n not prime").
        print("n prime");
 \end{lstlisting}
\end{block}
Alternatively, we can also Count the number of divisors of the given number. If it is more than 2, it is not a prime number. Else, it is a prime number.
\end{frame}

\begin{frame}[fragile]{Writing Efficient Programs}
  \begin{block}{Here is an alternative solution}
  \begin{lstlisting}[numbers=none]
Read the number into n.
for i from 1 to n, 
    if n % i = 0, then increment count
if count is 2, Print ("n prime");
else print ("n not prime")
  \end{lstlisting}

 \end{block}
Now, let us write Java code for the same. 
\end{frame}

\begin{frame}[fragile]{Writing Efficient Programs}
 \begin{lstlisting}
public class PrimeNumber2 {
    public static void main(String[] args) {
        int i;
        int n = Integer.parseInt(args[0]);
        int factorCount = 2;
        for (i = 2; i <= n - 1; i++) {
            if (n % i == 0) {
                factorCount++;
            }
        }
        if (factorCount == 2) {
            System.out.println(n + " is prime");
        } else {
            System.out.println(n + " is not prime.");
        }
    }
}
\end{lstlisting}
\end{frame}

\begin{frame}[fragile]{Writing Efficient Programs}
\begin{block}{Code}
\begin{lstlisting}[numbers=none]
for (i = 2; i <= n - 1; i++)  
    if (n % i == 0) 
        factorCount++; 
if (factorCount == 2)
    System.out.println(n + " is prime");
else
    System.out.println(n + " is not prime.");
\end{lstlisting}
\end{block}
\hspace{3cm}$\downarrow$

\vspace{.2pc}
\colorbox{orange}{Instead of `n-1' why not `n/2' ! }
\end{frame}

\begin{frame}[fragile]{Writing Efficient Programs}
 \begin{block}{Solution}
  \begin{lstlisting}[numbers=none]
public class PrimeNumber3 {
    public static void main(String[] args) {
        int n = Integer.parseInt(args[0]);
        int factorCount = 2;
        for (int i = 2; i <= n / 2; i++) {
            if (n % i == 0) {
                factorCount++;
            }
        }
        if (factorCount == 2) {
            System.out.println(n + " is prime");
        } else
            System.out.println(n + " is not prime.");
    }
}
\end{lstlisting}
\end{block}
\end{frame}

\begin{frame}{Writing Efficient Programs}
 \begin{center}
  \begin{minipage}{10cm}
   Do we really need to loop thru  `n/2' ?
   
   \vspace{.5pc}
   Can we do better?
   
   \vspace{.5pc}
   How about sqrt(n)? Is it sufficient? If yes, why? 
  \end{minipage}
 \end{center}
\end{frame}

\begin{frame}[fragile]{Writing Efficient Programs}
 \begin{block}{Solution}
 \begin{lstlisting}[numbers=none]
public class PrimeNumber3 {
    public static void main(String[] args) {
        int n = Integer.parseInt(args[0]);
        int factorCount = 2;
        for (int i = 2; i <= Math.sqrt(n); i++) {
            if (n % i == 0) {
                factorCount++;
            }
        }
        if (factorCount == 2) {
            System.out.println(n + " is prime");
        } else 
            System.out.println(n + " is not prime.");
    }
}
\end{lstlisting}
\end{block}
\end{frame}

\begin{frame}[fragile]{Writing Efficient Programs}
 \begin{block}{Our earlier solution for finding Perfect Square}
  \begin{lstlisting}[numbers=none]
public class PerfSquare {
    public static void main(String[] args) {
        int i = 1;
        int givenNumber = Integer.parseInt(args[0]);
        while (i < givenNumber) {
            if(i * i == givenNumber) { 
                System.out.println(givenNumber + " is perfect square.");
                return;
            }
            i++; 
        }
        System.out.println(givenNumber + " is not perfect square.");
    }
}
\end{lstlisting}

 \end{block}
\end{frame}

\begin{frame}[fragile]{Writing Efficient Programs}
 \begin{block}{A Better Solution for Finding Perfect Square:}
  \begin{lstlisting}[numbers=none]
public class PerfSquare {
    public static void main(String[] args) {
        int i = 1;
        int n = Integer.parseInt(args[0]);
        while (i * i < n) {
            i++;
        }
        if (i * i == n)
            System.out.println(n + " is perfect square.");
        else
            System.out.println(n + " is not perfect square.");
    }
} 
\end{lstlisting}

 \end{block}
\end{frame}

\begin{frame}{Writing Efficient Programs}
 \begin{minipage}{8cm}
 \begin{enumerate}
  \item Write a program for perfect square using \lstinline!sqrt()! function.

  
  \item Print all perfect squares between 1 and a given number `n'.
  \end{enumerate}
 \end{minipage}
 \quad
 \begin{minipage}{2cm}
  \includegraphics[scale=.4]{exercise.png}
 \end{minipage}
\end{frame}

\begin{frame}[fragile]{Writing Efficient Programs}
 \begin{block}{Solution for Finding Perfect Square upto `n':}
  \begin{lstlisting}[numbers=none]
public class PerfSquareRange {
    public static void main(String[] args) {
        int i , j;
        int n = Integer.parseInt(args[0]);
        for (i = 1; i <= n; i++) {
            j = 1;
            while (j * j < i) j++;
            if (j * j == i) 
                System.out.println(i + " is perfect square.");
        } 
    }
}
\end{lstlisting}

 \end{block}
\end{frame}

\begin{frame}{Writing Efficient Programs}
 \begin{figure}[H]
 \begin{center}
   \includegraphics[scale=.3]{qa.png}   
 \end{center}
  \end{figure}
\end{frame}

\end{document}
