\documentclass[11pt,a4paper]{article}
\usepackage{graphicx}
\usepackage{listings}
\lstset{language=Java,numbers=left,numberstyle=\tiny,numbersep=10pt,showstringspaces=false}
\usepackage{array}
\usepackage{enumitem}
\def\AnswerBox{\fbox{\begin{minipage}{4in}\hfill\vspace{0.5in}  \end{minipage}}}
\usepackage{fancyhdr}
\pagestyle{fancy}
\renewcommand{\headrulewidth}{0pt}
\rhead{\includegraphics[scale=.5]{TS-Logo.png}}
\def\Answerbox{\fbox{\begin{minipage}{4in}\hfill\vspace{1.0in}\end{minipage}}}
\begin{document}
\centerline{\huge{ \textbf{Arrays}}}
\vspace{1pc}
\centerline{\Large{ \textbf{Workbook}}}
\section*{Answer the following}
\begin{enumerate}
\item What would be the output of the below code?
\begin{lstlisting}
class SampleCodeA {
    public static void main(String[] args) {
        int[] a = new int[10];
        System.out.println(a.length);
        System.out.println(a[5]);
    }
}
\end{lstlisting}
\AnswerBox

\item What would be the output of the below code? 
\begin{lstlisting}
class SampleCodeB {
    public static void main(String[] args) {
        int[] a = new int[10];
        for (int i = 0; i < 10; i++) {
            a[i] = i * i;
        }
        for (int i = 0; i < 10; i++) {
            System.out.println(a[i]);
        }
    }
}
\end{lstlisting}
\AnswerBox
\item What would be the output of the below code?
\begin{lstlisting}
class SampleCodeC {
    public static void main(String[] args) {
        int[] a = {10, 20, 30, 40, 50};
        int[] b = a;
        for (int i = 0; i < 5; i++) {
            System.out.println(b[i]);
        }
    }
}
\end{lstlisting}
\AnswerBox
\item What would be the output of the below code and explain your understanding?
\begin{lstlisting}
class SampleCodeD {
    public static int xyz(int[] b) {
        int r = 0; 
        for (int i = 0; i < b.length; i++) {
            if (b[i] % 2 == 0)
                r += b[i];
        }
        return r;
    }
    public static void main(String[] args) {
        int[] a = {33, 44, 54, 65, 67, 93, 246, 84, 20, 10, 26};
        System.out.println(xyz(a));
    }
}
\end{lstlisting}
\AnswerBox
\item What would be the output of the below code?
\begin{lstlisting}
class SampleCodeE {
    public static void main(String[] args) {
        int table[][] = new int[3][3];
        for(int i = 0; i < 3; i++) {
            for(int j = 0; j < 3; j++) {
                if (j == i) 
                    table[i][j] = 1;
                else 
                    table[i][j] = 0;
            }
        }
        for (int i = 0; i < 3; i++) {
            for (int j = 0; j < 3; i++) {
                System.out.print(table[i][j] + " ");
            }
            System.out.println();
        }
    }
}
\end{lstlisting}
\AnswerBox
\item What would be the output of the below code?
\begin{lstlisting}
class SampleCodeF {
    public static void main(String[] args) {
        int[][] a = {{1,2,3},{4,5,6},{7,8,9}};
        for (int i = 0; i < a.length; i++) {
            for (int j = 0; j < a.length; j++) {
                System.out.print(a[i][j] + " ");
            }
            System.out.println();
        }
    }
}
\end{lstlisting}
\AnswerBox
\item What would be the output of the below code?
\begin{lstlisting}
class SampleCodeG {
    public static void main(String[] args) {
        int[][] m = {{1},{1,2},{1,2,3},{1,2,3,4},{1,2,3,4,5}};
        for (int i = 0; i < m.length; i++) {
            for (int j = 0; j < m[i].length; j++) {
                System.out.print(m[i][j]);
            }
            System.out.println();
        }
    }
}
\end{lstlisting}
\AnswerBox
\item What would be the output of the below code?
\begin{lstlisting}
class SampleCodeH {
    public static void main(String[] args) {
        int[][] a = {{11,12,13},{14,15,16},{17,18,19}};
        int sum = 0;
        for (int i = 0; i < a.length; i++) {
            for (int j = 0; j < a.length; j++) {
                if (i == j)
                    sum += a[i][j];
            }
        }
        System.out.println(sum);
    }
}
\end{lstlisting}
\AnswerBox
\end{enumerate}
\end{document}
