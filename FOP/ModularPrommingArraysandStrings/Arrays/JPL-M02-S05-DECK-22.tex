\documentclass[14pt]{beamer}
\title[JPL:Java:02]{JPL :: Multidimensional Arrays}
\author[TS]{TalentSprint}
\institute[L\&D]{Licensed To Skill}
\date{Version 1.0.4}
\usefonttheme{serif}
\usecolortheme{orchid}
\usepackage{bookman}
\usepackage{hyperref}
\usepackage[T1]{fontenc}
\usepackage{graphicx}
\usepackage{listings}
\graphicspath{{../../Images/}}
\beamertemplateballitem
\usebackgroundtemplate{\includegraphics[width=\paperwidth]{TS-XP-Logo.jpg}}
\lstset{language=Java,numbers=left, numberstyle=\tiny, basicstyle=\footnotesize, numbersep=10pt, showstringspaces=false, breaklines=true,keepspaces=true, columns=flexible}
\begin{document}

\begin{frame}
  \titlepage
\end{frame}

\begin{frame}{Learning Objectives}
By the end of this Presentation, you will be able to:
\begin{itemize}
\item Learn how to find Array size.
\item Understand enhanced for loop
\item Construct, Initialize and Manipulate Multidimensional  arrays
\end{itemize}
\end{frame}

\begin{frame}[fragile]{Multidimensional Arrays}
Array length

We can refer to array length by using length attribute of array object i.e. array\_name.length
\begin{lstlisting}[numbers=none]
int arr1 = new int[10];			arr1.length - 10 
int arr2 = new int[15];			arr2.length - 15 
int arr3 = new int[4];			arr3.length - 4 
for (int i = 0; i < 10; i++) {  } 
for (int i = 0; i < arr1.length; i++) {  }
\end{lstlisting}
\end{frame}

\begin{frame}[fragile]{Multidimensional Arrays}
\begin{block}{Accessing array elements one by one}
\begin{lstlisting}[numbers=none]
class ArrayDemo { 
    public static void main ( String[] args ) { 
        int[] array = { 20, 19, 12, 25} ; 
        for (int indx = 0; indx < array.length; indx++ )
            System.out.println(array[indx]);
    } 
} 
\end{lstlisting}
\end{block}
\end{frame}

\begin{frame}[fragile]{Multidimensional Arrays}
\begin{block}{Another way to access array elements one by one}
\begin{lstlisting}[numbers=none]
class Demo { 
    public static void main ( String[] args ) { 
        int[] array = { 20, 19, 12, 25} ; 
        for ( int num : array )
            System.out.println(num);
    } 
} 
\end{lstlisting}
\end{block}
\end{frame}

\begin{frame}{Multidimensional Arrays}
\begin{enumerate}
\item Write a Java program to store a student marks of 5 subjects
\item Now write java program to store more than one student marks of 5 subjects
\end{enumerate}
\end{frame}

\begin{frame}{Multidimensional Arrays}
\textbf{Multi Dimensional  arrays}

 2 - Dimensional arrays

 3 - Dimensional arrays

 Jagged arrays
\end{frame}

\begin{frame}[fragile]{Multidimensional Arrays}
Declaring Two Dimensional array variable

We can declare 2D array either of the following way.
<type>[][] variable\_name;
<type>[] variable\_name[];
<type> variable\_name[][];
\begin{block}{Example}
\begin{lstlisting}[numbers=none]
int[][] cse;
int[] cse[];
int cse[][];
\end{lstlisting}
\end{block}
Array is not created at declaration and memory not allocated
\end{frame}

\begin{frame}[fragile]{Multidimensional Arrays}
Defining  Two Dimensional array
\begin{lstlisting}[numbers=none]
variable_name = new <type>[rows][columns];

Example: 
cse = new int[3][4];

Declaring and Defining in the same statement:
int[][] cse = new int[3][4];		
cse.length = 3
cse[0].length = 4
cse[1].length = 4
cse[2].length = 4
\end{lstlisting}
In the above example, 3X4 = 12 integer variables will be created and 48 bytes of memory will be allocated.
\end{frame}

\begin{frame}[fragile]{Multidimensional Arrays}
\begin{block}{Initializing 2D array}
\begin{lstlisting}[numbers=none]
for (i = 0; i < cse.length; i++)
    for (j = 0; j < cse[i].length; j++)
        cse[i][j] = xxxxx;
\end{lstlisting}
\end{block}
\end{frame}

\begin{frame}[fragile]{Multidimensional Arrays}
Declaring, Defining and Initializing 2D array at the same time
\begin{block}{Initializing an Array}
\begin{lstlisting}[numbers=none]
int[][] cse = { {20, 19, 12, 25},
                {23, 12, 18, 21},
                {16, 18, 17, 20} };
\end{lstlisting}
\end{block}
\end{frame}

\begin{frame}{Multidimensional Arrays}
Write a Java program to declare, define, initialize and display 2D array
\end{frame}

\begin{frame}[fragile]{Multidimensional Arrays}
\begin{block}{Defining  Three dimensional array}
\begin{lstlisting}[numbers=none]
int[][][] college = new int[2][2][4];
college.length = 2
college[0].length = 2
college[1].length = 2
college[0][0].length = 4
college[0][1].length = 4
college[1][0].length = 4
college[1][1].length = 4
\end{lstlisting}
\end{block}
\end{frame}

\begin{frame}[fragile]{Multidimensional Arrays}
Defining  Jagged array
\begin{lstlisting}[numbers=none]
int[][] cse = new int[3][];

cse[0] = new int[4];
cse[1] = new int[2];
cse[2] = new int[3];

cse.length = 2
cse[0].length = 4
cse[1].length = 2
cse[2].length = 3
\end{lstlisting}
\end{frame}

\begin{frame}{Multidimensional Arrays}
Write a Java program to declare, define, initialize and display Jagged array
\end{frame}

\begin{frame}{Multidimensional Arrays}
\begin{enumerate}
\item Write a Java program to transpose a given 3 X 3 matrix.

\item Write a Java program to test a given matrix is lower triangle or not.

\item Write a Java program to test a given matrix is upper triangle or not.

\end{enumerate}
\end{frame}

\begin{frame}{Multidimensional Arrays}
\begin{figure}[H]
 \begin{center}
   \includegraphics[scale=.3]{qa.png}   
 \end{center}
  \end{figure}
\end{frame}
\end{document}
