\documentclass[11pt,a4paper]{article}
\author{TalentSprint}
\date{}
\usepackage{fancyhdr}
\usepackage{latexsym}
\usepackage{soul}
\usepackage{verbatim}
\usepackage{graphicx}
\usepackage{array}
\usepackage{enumerate}
%\usepackage{enumitem}
\usepackage{xcolor}
\usepackage[tikz]{bclogo}
\usepackage{textcomp}
\usepackage{latexsym}
\usepackage{seqsplit} 
\usepackage{setspace}
\usepackage{listings}
\lstset{language=html,breaklines=true,numbers=none,numberstyle=\tiny,numbersep=10pt,showstringspaces=false}
\usepackage{fancyhdr}
\headheight=14pt
\lhead{\nouppercase{}}
\rhead{\nouppercase{\leftmark}}

\graphicspath{{../Images/}}

%\pagestyle{fancy}

%========================================================================

% Lengths and widths
\addtolength{\textwidth}{2.5cm}
\addtolength{\hoffset}{0cm}
\setlength{\headsep}{-12pt} % Reduce space between header and content
\setlength{\headheight}{85pt} % If less, LaTeX automatically increases it
\renewcommand{\footrulewidth}{2pt} % Remove footer line
\renewcommand{\headrulewidth}{1pt} % Remove header line
\renewcommand{\seqinsert}{\ifmmode\allowbreak\else\-\fi} % Hyphens in seqsplit
% This two commands together give roughly
% the right line height in the tables
\renewcommand{\arraystretch}{1.3}
\onehalfspacing



% Commands
\newcommand{\SetRowColor}[1]{\noalign{\gdef\RowColorName{#1}}\rowcolor{\RowColorName}} % Shortcut for row colour
\newcommand{\mymulticolumn}[3]{\multicolumn{#1}{>{\columncolor{white}}#2}{#3}} % For coloured multi-cols
\newcolumntype{x}[1]{>{\raggedright}p{#1}} % New column types for ragged-right paragraph columns
\newcommand{\tn}{\tabularnewline} % Required as custom column type in use

% Font and Colours
\definecolor{HeadBackground}{HTML}{333333}
\definecolor{FootBackground}{HTML}{666666}
\definecolor{TextColor}{HTML}{333333}
\definecolor{DarkBackground}{HTML}{6B8E23} %{FD1AA8}
\definecolor{LightBackground}{HTML}{E8FED8} %D3FDC8
\definecolor{tit}{HTML}{FF6600}
\renewcommand{\familydefault}{\sfdefault}
\color{TextColor}
 \headsep = 25pt
% Header and Footer
\pagestyle{fancy}
\usepackage[headheight=110pt]{geometry}
\fancyhf{}% Clear header/footer

\fancyhead[r]{\includegraphics[width = 4cm, height = 2cm]{TS-Logo.png}\hspace{0cm}}

%=================================TITLE=====================================
\fancyhead[l]{{\bf{\textcolor{tit}{\textrm{JQuery Effects , Validations and JQuery UI}}}}}
%===========================================================================

\renewcommand{\headrulewidth}{0.4pt}% Default \headrulewidth is 0.4pt
\renewcommand{\footrulewidth}{0.4pt}% Default \footrulewidth is 0pt

\rfoot{Page \thepage}
\lfoot{COPYRIGHT \textcopyright TALENTSPRINT, 2015. ALL RIGHTS RESERVED.}




\begin{document}


%\chapter{jQuery Effects}
\section*{JQuery Effects}
jQuery provides many methods to show animation effect on page.
\section*{Hide effect}
The jquery hide() method hides specified HTML elements like DIV, paragraphs or others.

\textbf{Syntax of hide():}

\texttt{
\$(selector).hide(speed) 
}

Selector = can be an element like div, p, etc.
Speed = Optional parameter that specifies the hide speed with possible values of:
\begin{itemize}
 \item slow
 \item fast
 \item value in milliseconds
\end{itemize}
\begin{lstlisting}
<!DOCTYPE html>
<head>
    <title>jQuery Testing</title>
        <script src=``jquery-1.11.1.js''></script>
        <script>
            $(document).ready(function() {
                $(``.hidetext'').click(function () {
                    $(``.text'').hide(``slow'');
                });
                $(``.hidetext'').click(function () {
                    $(``.text'').show(2000);
                });
            });
        </script>
    </head>
    <body>
        <button class=``hidetext''>Hide yellow line</button>
        <button class=``showtext''>Show yellow line</button>
        <div class=``text'' style=``background-color:yellow;''>
            This is Yellow line!!
        </div>
     </body>
</html>
\end{lstlisting}
\section*{Show effect}
The jquery show() method shows specified HTML elements like DIV, paragraphs or others.\\
\textbf{Syntax of show():}

\texttt{\$(selector).show(speed) }

Selector = can be an element like div, p, etc.
Speed = Optional parameter that specifies the hide speed with possible values of:
\begin{itemize}
 \item slow
 \item fast
 \item value in milliseconds
\end{itemize}
\begin{lstlisting}
<!DOCTYPE html>
<head>
    <title>jQuery Testing</title>
        <script src=``jquery-1.11.1.js''></script>
        <script>
            $(document).ready(function() {
                $(``.hidetext'').click(function () {
                    $(``.text'').hide(``slow'');
                });
                $(``.showtext'').click(function () {
                    $(``.text'').show(2000);
                });
            });
        </script>
    </head>
    <body>
        <button class=``hidetext''>Hide yellow line</button>
        <button class=``showtext''>Show yellow line</button>
        <div class=``text'' style=``background-color:yellow;''>
            This is Yellow line!!
        </div>
    </body>
</html>
\end{lstlisting}

\section*{FadeOut effect}
jQuery fadeOut() method allows to fade-Out elements of website, for example $<$div$>$ or $<$p$>$ etc. that may contain text, images or other content.

It slowly changes the opacity of given element from visible to hidden.

Speed = Optional parameter that specifies the hide speed with possible values of:
\begin{itemize}
 \item slow
 \item fast
 \item value in milliseconds
\end{itemize}

\begin{lstlisting}
<!DOCTYPE html>
<head>
    <title>jQuery Testing</title>
        <script src=``jquery-1.11.1.js''></script>
        <script>
            $(document).ready(function() {
                $(``.hidetext'').click(function () {
                    $(``.text'').fadeOut(``slow'');
                });
            });  
        </script>
    </head>
    <body>
        <button class=``hidetext''>Fade Out yellow line</button>
        <div class=``text'' style=``background-color:yellow;''>
            This is Yellow line!!
        </div>
    </body>
</html>
\end{lstlisting}

\section*{FadeIn effect}
jQuery fadeIn() method allows to fade-in elements of website, for example $<$div$>$ or $<$p$>$ etc. that may contain text, images or other content.

It slowly changes the opacity of given element from hidden to visible.

Speed = Optional parameter that specifies the hide speed with possible values of:
\begin{itemize}
 \item slow
 \item fast
 \item value in milliseconds
\end{itemize}
\begin{lstlisting}
<!DOCTYPE html>
<head>
    <title>jQuery Testing</title>
        <script src=``jquery-1.11.1.js''></script>
        <script>
            $(document).ready(function() {
                $(``.showtext'').click(function () {
                    $(``.text'').fadeIn(``slow'');
                });
             </script>
    </head>
    <body>
        <button class=``showtext''>Show yellow line</button>
        <div class=``text'' style=``background-color:yellow;''>
            This is Yellow line!!
        </div>
    </body>
</html>
\end{lstlisting}
\section*{SlideDown effect}
jQuery slideDown() method allows to slide down elements of website, for example $<$div$>$ or $<$p$>$ etc. that may contain text, images or other content.

Speed = Optional parameter that specifies the hide speed with possible values of:
\begin{itemize}
 \item slow
 \item fast
 \item value in milliseconds
\end{itemize}
\begin{lstlisting}
<!DOCTYPE html>
<html>
    <head>
        <script src=``jquery-1.11.1.js''></script>
        <script>
            $(document).ready(function(){
                $(``#flip'').click(function(event){
                    $(``#para1'').css(``color'',``green'');
                    $(``#para1'').css(``font-weight'',``bold'');
                    $(``#panel'').slideDown(2000);
                });
            });
        </script>
        <style>
            #panel,#flip {
                padding:5px;
                text-align:center;
                background-color:#aabb22;
            }
            #panel {
                padding:50px;
                display:none;
            }
        </style>
    </head>
    <body>
        <p id=``para1''>The only thing that interferes with my learning is my education.
        <spanstyle=``color:red''>``Albert Einstien''</span><p>
        <div id=``panel''>Click to slideDown</div>
        <button id=``flip''>Slide Effect</button>
    </body>
</html>
\end{lstlisting}
\section*{Toggle event}
The toggle() method attaches two or more functions to toggle between for the click event for the selected elements.

When clicking on an element, the first specified function fires, when clicking again, the second function fires, and so on.

There are two Toggle methods:
\begin{itemize}
 \item fadeToggle
 \item slideToggle
\end{itemize}
\subsection*{FadeToggle}
FadeToggle attaches two functions fadeIn and fadeOut to toggle for the click event for the selected elements.

Speed = Optional parameter that specifies the hide speed with possible values of:
\begin{itemize}
 \item slow
 \item fast
 \item value in milliseconds
\end{itemize}

\begin{lstlisting}
<!DOCTYPE html>
<html>
    <head>
        <script src=``jquery-1.11.1.js''></script>
        <script>
            $(document).ready(function(){
                $(``#flip'').click(function(event){
                    $(``#para1'').css(``color'',``green'');
                    $(``#para1'').css(``font-weight'',``bold'');
                    $(``#panel'').fadeToggle(2000);
                });
            });
        </script>
        <style>
            #panel, #flip {
                padding:5px;
                text-align:center;
                background-color:#aabb22;
            }
            #panel {
                padding:50px;
                display:none;
            }
        </style>
    </head>
    <body>
        <p id=``para1''>The only thing that interferes with my learning is
        my education.
        <span style=``color:red''>``Albert Einstien''</span><p>
        <div id=``panel''>Click to slideDown</div>
        <button id=``flip''>Slide Effect</button>
    </body>
</html> 
\end{lstlisting}
\subsection*{SlideToggle}
SlideToggle attaches two functions slideDown and slideUp to toggle for the click event for the selected elements.
\begin{lstlisting}
<!DOCTYPE html>
<html>
    <head>
        <script src=``jquery-1.11.1.js''></script>
        <script>
            $(document).ready(function(){
                $(``#flip'').click(function(event){
                    $(``#para1'').css(``color'',``green'');
                    $(``#para1'').css(``font-weight'',``bold'');
                    $(``#panel'').slideToggle(2000);
                });
            });
        </script>
        <style>
            #panel, #flip {
                padding:5px;
                text-align:center;
                background-color:#aabb22;
            }
            #panel {
                padding:50px;
                display:none;
            }
        </style>
    </head>
    <body>
        <p id=``para1''>The only thing that interferes with my learning is
        my education.
        <span style=``color:red''>``Albert Einstien''</span><p>
        <div id=``panel''>Click to slideDown</div>
        <button id=``flip''>Slide Effect</button>
    </body>
</html> 
\end{lstlisting}

\section*{jQuery Validation}
jQuery Validation plugin makes it easy to validate user input while keeping your HTML markup clean from javascript code.\\
We can provide inbuilt error messages or customized messages to the user.\newline

Note:\\ 
jQuery form validation is not an alternative to server-side form validation.\newline
It will not make sure that an email address or credit card number is truly valid.\newline
It will, however, make sure that visitors fill out required form fields
and that they enter a correctly-formatted email address.
\subsection*{Prerequisite}
jquery-validator.js plugin is required to download before doing Validations.
\subsection*{How to do jQuery Validation}
There are two type of jQuery validations:\\
1. Simple form validation without any rules and customized messages.\\
2. Advanced form validation using rules and customized messages.\newline \newline \newline \newline
\subsection*{Simple Form Validation}
Step 1: Create a basic html form.
\begin{verbatim}
<form id="theform" method="post">

   <label for="firstName">Your First Name</label>
   <input class="required" name="firstname" type="text" id="firstName">

   <label for="lastName">Your Last Name</label>
   <input class="required" name="lastName" type="text" id="lastName" >

   <label for="phone">Your Phone Number</label>
   <input name="phone" type="text" id="phone">

   <label for="email">Email</label>
   <input class="required email" name="email" type="text" id="email">

   <label for="city">City</label>
   <input name="city" type="text" id="city">

   <input class="button" type="submit" value="SEND CONTACT FORM" />

</form> 
\end{verbatim}
NOTE:There are few very important things to remember.\\
\begin{lstlisting}
 Make sure to include id=``theform'' in your <form> tag.
 Any form field that is required needs to have class=``required''
                                                   in input tag.
 The email field needs to have class=``required email''. 
 Include the jQuery plugin files in the head part.
\end{lstlisting}
\subsection*{Include plugin at head tag}
$<$head$>$\\
$<$script type=``text/javascript'' src=``jquery.min.js''$>$ $<$/script$>$\\
$<$script type=``text/javascript'' src=``jQuery.Validate.min.js''$>$ $<$/script$>$\\
$<$/head$>$\\
\textbf{Call the validate() function to perform client side validation.}\newline \newline \newline \newline
\subsection*{Complete Example :: Simple Validation}
\begin{lstlisting}
<head>
<script type=``text/javascript'' src=``jquery.min.js''></script>
<script type=``text/javascript'' src=''jQuery.Validate.min.js''>
                                                       </script>
<script type="text/javascript">
      $(document).ready(function() {
          $(``#theForm'').validate();
      });
    </script> 
    <title>A Simple Contact form</title>
</head>
<body>
<form id=``theForm''>
<div>
<label for=``firstName''>Your First Name</label>
<input class=``required'' name=``firstName'' type=``text'' id=``firstName''/>
</div>
<div>
<label for=``lastName''>Your Last Name</label>
<input class=``required'' name=``lastName'' type=``text'' id=``lastName''/>
</div>
<div>
<label for=``phone''>Phone Number</label>
<input name=``phone'' type=``text'' id=``phone''/>
</div>
<div>
<label for=``email''>Email</label>
<input class=``required email`` name=``email'' type=``text'' id=``uemail''/>
</div>
<div>
<label for=``city''>City</label>
<input name=``city'' type=``text'' id=``city''/>
</div>
<div>
 <input class=``button'' type=``submit'' value=``SEND CONTACT FORM'' />
</div>
</form>
</body>
\end{lstlisting}
\subsection*{Advanced Form Validation}
We can provide customized messages to the user,instead of browser inbuilt messages.\\
Take the same form as above and we want to validate only firstname,phone,email.\\
\begin{lstlisting}
<head>
<script type=``text/javascript'' src=``jquery.min.js''></script>
<script type=``text/javascript'' src=''jQuery.validate.min.js''>
                                                      </script>
 <script type="text/javascript">
      $(document).ready(function() {
          $(``#theForm'').validate({
      rules: {
               firstName: {
                           required:true,
                           minlength:5
               },
               phone: {
                           required:true,
                           digits:true,
                           minlength:10,
                           maxlength:10
               },
               uemail: {
                          required:true,
                          email:true
               }          
                          
               },
       messages:{
          firstName:{
                    required:``Enter Name'',
                    minlength:``Minimum {0} characters required''
               },
           phone:{
                   required:``Enter Phone'',
                   digits:``Only Digits are allowed'',
                   minlength:``Minimum {0} digits required'',
                   maxlength:``Not more than {0} digits are allowed''
               },
          uemail:{
                  required:``Enter Email'',
                  email:``Enter valid email''
               }         
            }
          });
      });
    </script> 
    <title>A Simple Contact form</title>
</head>
<body>
<form id=``theForm''>
<div>
<label for=``firstName''>Your First Name</label>
<input class=``required'' name=``firstName'' type=``text'' id=``firstName''/>
</div>
<div>
<label for=``lastName''>Your Last Name</label>
<input class=``required'' name=``lastName'' type=``text'' id=``lastName''/>
</div>
<div>
<label for=``phone''>Phone Number</label>
<input name=``phone'' type=``text'' id=``phone''/>
</div>
<div>
<label for=``email''>Email</label>
<input class=``required email`` name=``email'' type=``text'' id=``uemail''/>
</div>
<div>
<label for=``city''>City</label>
<input name=``city'' type=``text'' id=``city''/>
</div>
<div>
<input class=``button'' type=``submit'' value=``SEND CONTACT FORM'' />
</div>
</form>
</body>
\end{lstlisting}
\pagebreak
\subsection*{Explanation}
\begin{itemize}
 \item Line 6 calls validate function for the form(``myForm'').
 \item Validate function brackets are not closed.
 \item This function will hold arguments rules and messages.
 \item Rules are defined for the elements which are to be validated.
 \item Messages are customized set for the rules.
 \item Messages should be in double quotes.
 \item Elements which are to be validated should have class=required.
\end{itemize}

% ================================================================
%    Jquery UI  
% ================================================================

\section*{jQuery UI}
jQuery UI, a JavaScript library built on top of jQuery and intended for desktop users, provides a set of interactions, effects, widgets, utilities, and themes for use on web pages.

To use jQuery UI, you must link to the library after first linking to the standard jQuery library. As with jQuery, you can link to the jQuery UI library either via a hosted CDN (from Google or jQuery) or by downloading and hosting the library yourself.

Note that jQuery UI is themeable: you can use the stock themes (a pre-designed set of CSS rules that govern how popup menus, buttons, and other elements look in response to user interactions) or build your own themes with jQuery UI’s ThemeRoller.


\subsection*{Interactions}
 
Query UI offers the following interactions, which are ways to add mouse-based behaviors to elements
\begin{description}
 \item [Draggable] Enable draggable functionality on any element.
 \item [Droppable] Create a target for a droppable element.
 \item [Resizable] Change the size of an element with the mouse.
 \item [Selectable] Select one or more elements with the mouse.
 \item [Sortable] Reorder elements with the mouse.

\end{description}
\pagebreak

\subsection*{Example}

\begin{lstlisting}
<html>
	<head>
<script>
	$(document).ready(function(){
		$(’#draggable’).draggable();
		$(’#droppable’).droppable({
			drop:function(event,ui){
				$(this)
				.addClass(’dropped’)
				.find(’h2’)
				.html(’Dropped’);
			}
		});
		$(’#resizable’).resizable();
		$(’#selectable’).selectable();
		$(’#sortable’).sortable();
	});
</script>
<style>
	.container{
		width:30%;
		margin:01%1%0;
		float:left;
		height:200px;
		background−color:#fbb;
		padding:1%;
	}
	.widecontainer{
		width:98%;
		margin−bottom:1%;
		clear:left;
		background−color:#fbb;
		height:100px;
	}
	.contained{
		width:90px;
		height:90px;
		padding:10px;
		background−color:#c00;
	}
	h3{
		float:right;
		font−size:14px;
		font−weight:bold;
		color:#444;
		margin:5px5px00;
	}
	.dropped{
		background−color:#fff;
	}
	ul{
		list−style:none;
		margin:0;
		padding:0;
	}
	ulli{
		float:left;
		margin:20px10px010px;
		padding:5px15px;
		background−color:#c00;
	}
	#selectable.ui−selecting{
		background−color:#600;
	}
	#selectable.ui−selected{
		background−color:#000;
		color:white;
	}
</style>
</head>
<body>
	<div id=”draggablecontainer” class=”container”>
	<h3>Draggable</h3>
	<div id=”draggable” class=”contained”>DragMe</div>
	</div>
	<div id=”droppable” class=”container”>
	<h3>Droppable</h3>
	<h2>DropHere</h2>
	</div>
	<div id=”resizablecontainer” class=”container”>
	<h3>Resizable</h3>
	<div id=”resizable” class=”contained”>ResizeMe</div>
	</div>
	<div id=”selectablecontainer” class=”widecontainer”>
	<h3>Selectable</h3>
	<ul id=”selectable”>
		<li>One</li>
		<li>Two</li>
		<li>Three</li>
		<li>Four</li>
		<li>Five</li>
		<li>Six</li>
		<li>Seven</li>
		<li>Eight</li>
		<li>Nine</li>
		<li>Ten</li>
	</ul>
	</div>
	<div id=”sortablecontainer”class=”widecontainer”>
	<h3>Sortable</h3>
	<ul id=”sortable”>
		<li>One</li>
		<li>Two</li>
		<li>Three</li>
		<li>Four</li>
		<li>Five</li>
		<li>Six</li>
		<li>Seven</li>	
		<li>Eight</li>
		<li>Nine</li>
		<li>Ten</li>
	</ul>
	</div>
</body>
</html>		
		
\end{lstlisting}


\subsection*{Widgets}
jQuery UI offers a set of powerful widgets that enable complex functionality with minimal coding, and allow precise customization with more work. All of the widgets are amenable to jQuery UI’s theming, displaying according to one of the stock themes or your custom theme, as you decide.
\begin{description}
 \item [Accordion] Collapsible content panels for displaying information in small space.
\item [Autocomplete] Select from a pre-populated list of values as user types in field.
\item [Button] Enhance standard form elements as themeable buttons with hover and active styles.
\item [Datepicker] Select a date from a popup calendar.
\item [Dialog] Open content in an interactive overlay.
\item [Menu] Themeable menu with mouse and keyboard interactions.
\item [Progressbar] Display status of a determinate or indeterminate process.
\item [Slider] Drag handle to select numeric value.
\item [Spinner] Enhance a text input for entering numeric values, with up/down buttons and arrow-key handling.
\item [Tabs] Tabbed content area with multiple panels, each associated with a header in a list.
\item [Tooltip] Customizable, themeable tooltips to replace native tooltips.
\end{description}
\subsection*{Example}
\begin{lstlisting}
<html>
    <head>
        <script>
            $(document).ready(function(){
                $(’#acc’).accordion();
                varmylist=[
                    ”one”,
                    ”two”,
                    ”three”,
                    ”four”,
                    ”five”
                ];
                $(’#search’).autocomplete({
                    source:mylist
                });
                $(’#dp’).datepicker();
            });
        </script>
        <style>
            #acc{
                width:32%;
                margin−right:1%;
                float:left;
            }
            #autocompletecontainer{
                width:32%;
                margin−right:1%;
                float:left;
            }
            #datepickercontainer{
                width:32%;
                float:left;
            }
        </style>
    </head>
    <body>
        <div id=``acc''>
            <h3>Item1</h3>
            <div>
                <p>content1 content1 content1</p>
            </div>
            <h3>Item2</h3>
            <div>
                <p>content2 content2 content2</p>
            </div>
            <h3>Item3</h3>
            <div>
                <p>content3 content3 content3</p>
            </div>
        </div>
        <div id=``autocompletecontainer''>
            <label for=``search''>Start typing(try "o" or "t"):</label><input type=``text''
        </div>
        <div id=``datepickercontainer''>
            <label for=``dp''>Select date:</label><input type=``text'' name=``dp'' id=``dp''>
        </div>
    </body>
</html>
\end{lstlisting}

\section*{Effects}
jQuery UI effects mimic core jQuery effects, with the addition of animations and control over duration, easing, and other properties. As the docs state: ``[e]ffects add support for animating colors and class transitions, as well as providing several additional easings [with] a full suite of custom effects for use when showing and hiding elements or just to add some visual appeal.''

Where the jQuery core addClass method, for example, adds a class to any element(s) matching the found set, the jQuery UI addClass method does the same (i.e. adds a class to the matched elements) but also animates the changes of background color, width, height, etc., over the specified duration. (Note that some properties cannot be animated: back-ground images, for example, are changed at the end of the duration.) Note that jQuery UI
animations differ from CSS3 transitions; jQuery UI animations are often used as fallbacks for older browsers that do not support CSS3 transitions.

jQuery UI effects include the following

\begin{itemize}
\item Add Class
\item Color Animation
\item Effect
\item Hide
\item Remove Class
\item Show
\item Switch Class
\item Toggle
\item Toggle Class

\end{itemize}


\end{document}