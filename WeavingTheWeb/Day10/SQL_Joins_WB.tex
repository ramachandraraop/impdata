\documentclass[11pt,a4paper]{article}
\usepackage{graphicx}
\usepackage{listings}
\lstset{language=C,numbers=left,numberstyle=\tiny,numbersep=10pt,showstringspaces=false}
\usepackage{array}
\usepackage{enumitem}

\def\AnswerBox{\fbox{\begin{minipage}{4in}\hfill\vspace{0.5in}\end{minipage}}}
\usepackage{fancyhdr}
\pagestyle{fancy}
\renewcommand{\headrulewidth}{0pt}
\rhead{\includegraphics[scale=.5]{../Images/TS-Logo.png}}
\begin{document}
\centerline{\huge{ \textbf{Joins}}}

\vspace{1pc}

\centerline{\Large{ \textbf{Workbook}}}
\section*{Answer the following}

\begin{enumerate}\itemsep10pt

\item The joining technique is useful when data from several relations are to be retrieved and displayed and the relationships are not necessarily nested. (True/False)

\item \underline{\hspace{3cm}} number of tables can be included with in a joining condition. 

\item Subqueries can be nested multiple times. (True/False)\underline{\hspace{3cm}}

\item Outer join is the same as equi-join, except one of the duplicate columns is eliminated in the result table. (True/False)

\item \underline{\hspace{3cm}} type of join is used to include rows that do not have matching values.

\item There should be one condition within the WHERE cluase for each pair of tables being joined. (True/False)

\item \underline{\hspace{3cm}} type of join is used to return rows that do have matching values.

\item Equi-join is same as inner-equi-join.(True/False)

\item \underline{\hspace{3cm}} type of join retrieves the unmatched rows and matched from the left table and matched rows from right table.

\item \underline{\hspace{3cm}} type of join is used in the below query.

\texttt{MYSQL> SELECT * FROM EMP E INNER JOIN DEPT ON D WHERE E.DEPTNO = D.DEPTNO;}


\item \underline{\hspace{3cm}} type of join is used in the below query.

\texttt{SELECT * FROM EMP E1, EMP E2 WHERE E1.JOB = E2.JOB;}

\item Cross join generally generate cross product between two tables.(True/False)


\end{enumerate}

\subsection*{Exercises}
\begin{itemize}
\item Write the expected output, or compiler errors if any, for each of the following programs.

\end{itemize}
\begin{description}
\item[Program 1]\
\begin{verbatim}
MYSQL> SELECT * FROM GAME;

ID        MDATE       STADIUM                   TEAM1  TEAM2
--------- ----------- ------------------------  ------ -------
1001      8-June-2012 National Stadium, Warsaw  POL    GRE
1002      8-June-2012 Stadion Miejski (Wroclaw) RUS    CZE
1003     12-June-2012 Stadion Miejski (Wroclaw) GRE    CZE
1004     12-June-2012 National Stadium, Warsaw	POL    RUS

MYSQL> SELECT * FROM GOAL;

MATCHID   TEAMID      PLAYER                    GTIME
--------- ----------- ------------------------  -----
1001      POL         Robert Lewandowski        17
1001      GRE         Dimitris Salpingidis      51
1002      RUS         Alan Dzagoev              15
1001      RUS         Roman Pavlyuchenko        82

MYSQL> SELECT * FROM ETEAM;

ID        TEAMNAME        COACH                   
--------- -----------     ------------------
POL       Poland          Franciszek Smuda
RUS       Russia          Dick Advocaat
CZE       Czech Republic  Michal Bilek
GRE       Greece          Fernando Santos      
\end{verbatim}

%\AnswerBox


\begin{enumerate}[label=\bfseries Q\arabic*:]\itemsep10pt
\item Write a query to retrieve players, their team and the amount of goals they scored against Greece(GRE).
\item What will be the output of the query given below?\newline
\texttt{SELECT teamid, mdate FROM goal JOIN game on (matchid = id) WHERE mdate = `9 June 2012';}
\item Write a query to retrieve player and their team for those who have scored against Poland(POL) in National Stadium, Warsaw.
\item What will be the output of the query given below?\newline
\texttt{MYSQL> SELECT teamname, COUNT(*) FROM eteam JOIN goal ON teamid = id GROUP BY teamname HAVING COUNT(*) < 3;}

\end{enumerate}

\item[Program 2]\
\begin{verbatim}
MYSQL> SELECT * FROM GAME;

ID        MDATE       STADIUM                   TEAM1  TEAM2
--------- ----------- ------------------------  ------ -------
1001      8-June-2012 National Stadium, Warsaw  POL    GRE
1002      8-June-2012 Stadion Miejski (Wroclaw) RUS    CZE
1003     12-June-2012 Stadion Miejski (Wroclaw) GRE    CZE
1004     12-June-2012 National Stadium, Warsaw	POL    RUS

MYSQL> SELECT * FROM GOAL;

MATCHID   TEAMID      PLAYER                    GTIME
--------- ----------- ------------------------  -----
1001      POL         Robert Lewandowski        17
1001      GRE         Dimitris Salpingidis      51
1002      RUS         Alan Dzagoev              15
1001      RUS         Roman Pavlyuchenko        82

MYSQL> SELECT * FROM ETEAM;

ID        TEAMNAME        COACH                   
--------- -----------     ------------------
POL       Poland          Franciszek Smuda
RUS       Russia          Dick Advocaat
CZE       Czech Republic  Michal Bilek
GRE       Greece          Fernando Santos
\end{verbatim}

%\AnswerBox

\begin{enumerate}[label=\bfseries Q\arabic*:]\itemsep10pt

\item What will be the output of the query given below?\newline
\texttt{SELECT teamname, player FROM eteam JOIN goal ON id = teamid ORDER BY teamname;}
\item Write a query to retrieve the stadium and the number of goals scored in each stadium.?
\item What will be the output of the query given below?\newline
\texttt{SELECT player, stadium FROM game JOIN goal ON (id = matchid);}
\end{enumerate}
\item [Program 3]\ 
\begin{verbatim}
MYSQL> DESC MOVIE;         MYSQL> DESC ACTOR;         MYSQL> DESC CASTING;
                                                                          
FIELD    TYPE              FIELD     TYPE             FIELD      TYPE   
------   -----             -------   -----            ------     -----
ID       INT               ID        INT              MOVIEID    INT
TITLE    VARCHAR(30)       NAME      VARCHAR(30)      ACTORID    INT
YR       DATE
DIRECTOR VARCHAR(30)
BUDGET   FLOAT
GROSS    FLOAT

\end{verbatim}
%\AnswerBox

\begin{enumerate}[label=\bfseries Q\arabic*:]\itemsep10pt
\item Write a query to retrieve the titles of the films with id 11768, 11955, 21191
\item Write a query to retrieve id number of the actor `Glenn Close'.
\item Write a query  that shows the list of actors called 'John' by order of number of movies in which they acted.
\item What will be the output of the query given below?\newline
\texttt{MYSQL> SELECT title FROM movie, casting, actor WHERE name =`Paul Hogan' AND movieid = movie.id AND actorid = actor.id;}
\end{enumerate}


\end{description}


\end{document}