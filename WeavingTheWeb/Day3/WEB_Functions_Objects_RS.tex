\documentclass[14pt]{beamer}
\title{WEB :: JavaScript}
\author[TS]{TalentSprint}
\institute[L\&D]{Licensed To Skill}
\usefonttheme{serif}
\usecolortheme{orchid}
\usepackage{bookman}
\usepackage{multirow}
\usepackage{hyperref}
\usepackage[T1]{fontenc}
\usepackage{graphicx}
\usepackage{listings}
\graphicspath{{./../Images/}}
\usepackage[tikz]{bclogo}
\usepackage{soul}
 \definecolor{light-gray}{gray}{0.80}
 \usepackage{color}
\beamertemplateballitem
\usebackgroundtemplate{\includegraphics[width=\paperwidth]{TS-XP-Logo.jpg}}
\lstset{language=html, numbers=left, numbers=none, basicstyle=\footnotesize, numberstyle=\tiny,  numbersep=10pt, showstringspaces=false, breaklines=true,keepspaces=true, columns=flexible}
\begin{document}

\begin{frame}
  \titlepage
\end{frame}

\begin{frame}{Learning Objectives}
The content in this presentation is aimed at teaching  learners to:
  \begin{itemize}
  \item Deploy event driven code
  \item Get the data from the form controls
  \item Dynamically change the page content
  \item Validate HTML WebPages
  \item Use JavaScript built-in objects 
  \end{itemize}
\end{frame}

\begin{frame}{Accessing HTML Elements}
We can access HTML Elements from Javascript to perform following action:
\begin{enumerate}
 \item Change content
 \item Change Style 
 \item Change CSS Classes
 \item Setting and Getting Values
 \item Changing Attribute values
\end{enumerate}
\begin{block}{}
\lstinline!var object = document.geElementById(``element_id'')!
\end{block}
\end{frame}

\begin{frame}{CSS to JavaScript}
\begin{tabular}{|p{4cm} | p{6.5cm} |}
\hline \textbf{CSS Property} & \textbf{Javascript Property} \\ \hline
color & object.style.color \\ \hline
text-align & object.style.textAlign \\ \hline
text-decoration  & object.style.textDecoration \\ \hline
white-space & object.style.whiteSpace \\ \hline
text-transform & object.style.textTransform \\ \hline
letter-spacing & object.style.letterSpacing \\ \hline
text-indent & object.style.textIntent \\ \hline
line-height & object.style.lineHeight \\ \hline
word-spacing & object.style.wordSpacing \\ \hline
\end{tabular}
Similarly we can work with all CSS Properties in Javascript
\end{frame}

\begin{frame}{CSS to JavaScript}
\begin{tabular}{|p{4cm} | p{6.5cm} |}
\hline \textbf{Element Type} & \textbf{Javascript Property} \\ \hline
Input & object.value\newline object.name\newline object.type \\ \hline
Img & object.src \\ \hline
Select & object.selectedIndex \\ \hline
Form & object.name\newline object.action\newline object.elements \\ \hline
Div, span, td etc. & object.innerHTML\newline object.className \\ \hline
\end{tabular}
\end{frame}

\begin{frame}{JavaScript}
\textbf{The String Object}

\vspace{1pc}
Javascript Automatically converts between string primitives and String objects.
\begin{block}{Syntax}
\lstinline!var val = new String(``string'');!

\lstinline!var txt = ``string'';!
\end{block}
\end{frame}

\begin{frame}{JavaScript}
\textbf{String Object Methods}

\vspace{1pc}
\begin{description}
 \item [charAt()] returns the character at the specified index (position)
 \item [charCodeAt()] returns the Unicode of the character at the specified index
 \item [concat()] joins two or more strings, and returns a copy of the joined strings
 \item [valueOf()] returns the primitive value of a String object
\end{description}
\end{frame}

\begin{frame}{JavaScript}
\textbf{String Object Methods}

\vspace{1pc}
\begin{description}
 \item [indexOf()] returns the position of the first found occurrence of a specified value in a string
 \item [lastIndexOf()] returns the position of the last found occurrence of a specified value in a string
 \item [match()] searches a string for a match against a regular expression, and returns the matches
\end{description}
\end{frame}

\begin{frame}{JavaScript}
\textbf{String Object Methods}

\vspace{1pc}
\small
\begin{description}
 \item [replace()] searches a string for a value and returns a new string with the value replaced
 \item [search()] searches a string for a value and returns the position of the match
 \item [substring()] extracts a part of a string between two specified positions
 \item [toLowerCase()] converts a string to lowercase letters
 \item [toUpperCase()] converts a string to uppercase letters
 \end{description}
\end{frame}

\begin{frame}{JavaScript}
\textbf{String HTML Wrapper Methods}

\begin{description}
 \item [anchor()] creates an HTML anchor that is used as a hypertext target.
 \item [big()] creates a string to be displayed in a big font as if it were in a <big> tag.
 \item [blink()] creates a string to blink as if it were in a <blink> tag.
 \item [bold()] creates a string to be displayed as bold as if it were in a <b> tag.
\end{description}
\end{frame}

\begin{frame}{JavaScript}
\textbf{String HTML Wrapper Methods}

\begin{description}
 \item [fixed()] causes a string to be displayed in fixed-pitch font as if it were in a <tt> tag
 \item [fontcolor()] causes a string to be displayed in the specified color as if it were in a <font color="color"> tag
 \item [Fontsize()] causes a string to be displayed in the specified font size as if it were in a <font size="size"> tag
 \item [italics()] causes a string to be italic, as if it were in an <i> tag
\end{description}
\end{frame}

\begin{frame}{JavaScript}
\textbf{String HTML Wrapper Methods}
\small
\begin{description}
 \item [link()] creates an HTML hypertext link that requests another URL
 \item [small()] causes a string to be displayed in a small font, as if it were in a <small> tag
 \item [strike()] causes a string to be displayed as struck-out text, as if it were in a <strike> tag
 \item [sub()] causes a string to be displayed as a subscript, as if it were in a <sub> tag
 \item [sup()] causes a string to be displayed as a superscript, as if it were in a <sup> tag
\end{description}
\end{frame}

\begin{frame}{JavaScript}
\textbf{The Array Object}
\begin{itemize}
 \item Store multiple values in a single variable
 
 \lstinline!var fruits = new Array(``apple'', ``orange'', ``mango'');!
 \item The Array parameter is a list of strings or integers
 \item The maximum length allowed for an array is 4,294,967,295
 \item You can create array by simply assigning values as follows:
 
 \lstinline!var fruits = [``apple'', ``orange'', ``mango''];!
\end{itemize}
\end{frame}

\begin{frame}{JavaScript}
\textbf{Array Methods}
\begin{description}
 \item [concat()] returns a new array comprised of this array joined with other array(s) and/or value(s).
 \item [every()] returns true if every element in this array satisfies the provided testing function.
 \item [filter()] creates a new array with all of the elements of this array for which the provided filtering function returns true.
\end{description}
\end{frame}

\begin{frame}{JavaScript}
\textbf{Array Methods}
\begin{description}
 \item [indexOf()] returns the first (least) index of an element within the array equal to the specified value or 1 if none is found.
 \item [join()] joins all elements of an array into a string.
 \item [lastIndexOf()] returns the last (greatest) index of an element within the array equal to the specified value or 1 if none is found.
\end{description}
\end{frame}

\begin{frame}{JavaScript}
\textbf{Array Methods}
\begin{description}
 \item [map()] creates a new array with the results of calling a provided function on every element in this array.
 \item [pop()] removes the last element from an array and returns that element.
 \item [push()] adds one or more elements to the end of an array and returns the new length of the array.
\end{description}
\end{frame}

\begin{frame}{JavaScript}
\textbf{Array Methods}
\begin{description}
 \item [reduce()] apply a function simultaneously against two values of the array (from left- to-right) as to reduce it to a single value.
 \item [reduceRight()] apply a function simultaneously against two values of the array (from right- to-left) as to reduce it to a single value.
 \item [reverse()] reverses the order of the elements of an array -- the first becomes the last, and the last becomes the first.
\end{description}
\end{frame}

\begin{frame}{JavaScript}
\textbf{Array Methods}
\begin{description}
 \item [shift()] removes the first element from an array and returns that element.
 \item [slice()] extracts a section of an array and returns a new array.
 \item [some()] returns true if at least one element in this array satisfies the provided testing function.
 \item [toSource()] represents the source code of an object
 \end{description}
\end{frame}

\begin{frame}{JavaScript}
\textbf{Array Methods}
\begin{description}
 \item [sort()] sorts the elements of an array.
 \item [splice()] adds and/or removes elements from an array.
 \item [toString()] returns a string representing the array and its elements.
 \item [unshift()] adds one or more elements to the front of an array and returns the new length of the array.
 \item [forEach()] calls a function for each element in the array.
\end{description}
\end{frame}

\begin{frame}[fragile]{JavaScript}
\textbf{The Date Object}
\begin{itemize}
 \item Built into the JavaScript language.
 \item Date objects are created with the new Date()
 \item different variant of Date() constructor:
 \begin{block}{}
  \begin{lstlisting}
   new Date()
   new Date(milliseconds)
   new Date(datestring)
   new Date(year,month,date[,hour,minute,second,millisecond])
  \end{lstlisting}
 \end{block}
\end{itemize}
\end{frame}

\begin{frame}{JavaScript}
\textbf{Getter Methods of Date}
\begin{description}
 \item [Date()] returns today's date and time
 \item [getDate()] returns the day of the month for the specified date according to local time.
 \item [getDay()] returns the day of the week for the specified date according to local time.
 \item [getFullYear()] returns the year of the specified date according to local time.
\end{description}
\end{frame}

\begin{frame}{JavaScript}
\textbf{Getter Methods of Date}
\begin{description}
 \item [getHours()] returns the hour in the specified date according to local time.
 \item [getMilliseconds()] returns the milliseconds in the specified date according to local time.
 \item [getMinutes()] returns the minutes in the specified date according to local time.
\end{description}
\end{frame}

\begin{frame}{JavaScript}
\textbf{Getter Methods of Date}
\begin{description}
 \item [getMonth()] returns the month in the specified date according to local time.
 \item [getSeconds()] returns the seconds in the specified date according to local time.
 \item [getTime()] returns the numeric value of the specified date as the number of milliseconds since January 1, 1970, 00:00:00 UTC.
\end{description}
\end{frame}

\begin{frame}{JavaScript}
\textbf{Setter Methods of Date}
\begin{description}
 \item [setDate()] sets the day of the month for a specified date according to local time.
 \item [setFullYear()] sets the full year for a specified date according to local time.
 \item [setHours()] sets the hours for a specified date according to local time.
 \item [setMilliseconds()] sets the milliseconds for a specified date according to local time.
 \end{description}
\end{frame}

\begin{frame}{JavaScript}
\textbf{Setter Methods of Date}
\begin{description}
 \item [setMinutes()] sets the minutes for a specified date according to local time.
 \item [setMonth()] sets the month for a specified date according to local time.
 \item [setSeconds()] sets the seconds for a specified date according to local time.
 \item [setTime()] sets the Date object to the time represented by a number of milliseconds since January 1, 1970, 00:00:00 UTC.
\end{description}
\end{frame}

\begin{frame}{JavaScript}
\textbf{The Math Object}
\begin{itemize}
 \item Provides you properties and methods for mathematical constants and functions.
 \item Math is not a constructor.
\end{itemize}
\end{frame}

\begin{frame}{JavaScript}
\textbf{The Math Object}
\begin{itemize}
 \item All properties and methods of Math are static and can be called by using Math as an object without creating it.
 \item Syntax to call properties and methods of Math:
  
  \lstinline!var pi_val = Math.PI;!
  
  \lstinline!var sine_val = Math.sin(30);!
\end{itemize}
\end{frame}

\begin{frame}{JavaScript}
\textbf{Math Properties}
\begin{description}
 \item [E] euler's constant and the base of natural logarithms, approximately 2.718.
 \item [LN2] Natural logarithm of 2, approximately 0.693.
 \item [LN10] Natural logarithm of 10, approximately 2.302.
 \item [LOG2E] Base 2 logarithm of E, approximately 1.442.
\end{description}
\end{frame}

\begin{frame}{JavaScript}
\textbf{Math Properties}
\begin{description}
 \item [LOG10E] Base 10 logarithm of E, approximately 0.434.
 \item [PI] Ratio of the circumference of a circle to its diameter, approximately 3.14159.
 \item [SQRT1\_2] Square root of 1/2; equivalently, 1 over the square root of 2, approximately 0.707.
 \item [SQRT2] Square root of 2, approximately 1.414.
\end{description}
\end{frame}


\begin{frame}{JavaScript}
\textbf{Math Properties}
\begin{description}
 \item [abs()] returns the absolute value of a number.
 \item [acos()] returns the arccosine (in radians) of a number.
 \item [asin()] returns the arcsine (in radians) of a number.
 \item [atan()] returns the arctangent (in radians) of a number.
 \item [atan2()] returns the arctangent of the quotient of its arguments.
\end{description}
\end{frame}

\begin{frame}{JavaScript}
\textbf{Math Properties}
\begin{description}
  \item [ceil()] returns the smallest integer greater than or equal to a number.
  \item [cos()] returns the cosine of a number.
 \item [exp()] returns EN, where N is the argument, and E is Euler's constant, the base of the natural logarithm.
 \item [floor()] returns the largest integer less than or equal to a number.
\end{description}
\end{frame}

\begin{frame}{JavaScript}
\textbf{Math Properties}
\begin{description}
 \item [log()] returns the natural logarithm (base E) of a number.
 \item [max()] returns the largest of zero or more numbers.
 \item [min()] returns the smallest of zero or more numbers.
 \item [pow()] returns base to the exponent power, that is, base exponent.
 \item [random()] returns a pseudo-random number between 0 and 1.
\end{description}
\end{frame}

\begin{frame}{JavaScript}
\textbf{Math Properties}
\begin{description}
 \item [round()] returns the value of a number rounded to the nearest integer.
 \item [sin()] returns the sine of a number.
 \item [sqrt()] returns the square root of a number.
 \item [tan()] returns the tangent of a number.
 \item [toSource()] Returns the string ``Math''.
\end{description}
\end{frame}




\begin{frame}{JavaScript}
 \begin{figure}[H]
    \includegraphics[scale=.3]{qa.png}   
   \end{figure}
\end{frame}

\end{document}
