\documentclass[11pt,a4paper]{article}
\usepackage{graphicx}
\usepackage{listings}
\lstset{language=C,numbers=left,numberstyle=\tiny,numbersep=10pt,showstringspaces=false}
\usepackage{array}
\usepackage{enumitem}

\def\AnswerBox{\fbox{\begin{minipage}{4in}\hfill\vspace{0.5in}\end{minipage}}}

\begin{document}
\section*{\center{Practice Exercise}}
\subsection*{Review Questions}
\begin{enumerate}\itemsep10pt
\item The three forms of \texttt{if} statement in JavaScript are\underline{\hspace{3cm}}.
\item The syntax of JavaScript \texttt{if} statement is\underline{\hspace{3cm}}.
\item How to write an \texttt{if} statement for executing some code if ``i'' is NOT equal to 5?\underline{\hspace{3cm}}.
\item The syntax of JavaScript \texttt{if-else} statement is\underline{\hspace{3cm}}
\item What does the following expression return?\underline{\hspace{3cm}}.
\begin{lstlisting}
4 > 1;
\end{lstlisting}
\item What does the following expression return?\underline{\hspace{3cm}}.
\begin{lstlisting}
``chatty'' = = = ``chatty'';
\end{lstlisting} 
\item What does the following expression return?\underline{\hspace{3cm}}.
\begin{lstlisting}
!false
\end{lstlisting}
\item What does the following code return?\underline{\hspace{3cm}}.
\begin{lstlisting}
var blue = [``da ba dee da ba die''];
blue = = = blue;
\end{lstlisting}
\item What does the following statement return?\underline{\hspace{3cm}}.
\begin{lstlisting}
[1, 89] = = = [1, 89];
\end{lstlisting}
\item The loops are used to iterate the piece of code.( Yes / No)\underline{\hspace{3cm}}.
\item The syntax of \texttt{for} loop is\underline{\hspace{3cm}}.
\item The JavaScript \texttt{while} loop iterates the elements for the infinite number of times mostly.(True / False)\underline{\hspace{3cm}}.
\item The syntax of \texttt{do while} loop is\underline{\hspace{3cm}}.
\item What are the three important manipulations done in a \texttt{for} loop on a loop variable?\underline{\hspace{3cm}}.
\item A ? B : C is equivalent to\underline{\hspace{3cm}}.
\end{enumerate}

\subsection*{Exercises}
\begin{itemize}
\item Write the expected output, or compiler errors if any, for each of the following programs in the box provided below each program.
\item Then execute the programs and check your answers.
\item Then answer the questions given below.
\end{itemize}
\begin{description}
\item[program 1]\
\begin{lstlisting}
if (5 > 10) {
    document.write(``Not so sure about this'');
} else {
    document.write(``walking dead'');
}
\end{lstlisting}

\AnswerBox

\begin{enumerate}[label=\bfseries Q\arabic*:]\itemsep10pt
\item What will be the output for the above code snippet?
\item What will be the output if condition in \texttt{if} is changed to \texttt{5 === 10}?
\item What will be the output if condition in \texttt{if} is changed to \texttt{5 >= 10}?
\end{enumerate}

\item[Program 2]\
\begin{lstlisting}
if (``candy'' === 8) {
    document.write(``do something with your life'');
} else if (``blah'' === ``blah'') {
    document.write(``just chill, have fun'');
} else {
    document.write(``people are strange'');
}\end{lstlisting}

\AnswerBox

\begin{enumerate}[label=\bfseries Q\arabic*:]\itemsep10pt
\item What will be the output for the above code snippet?
\item What will be the output if condition in \texttt{if} is changed to \texttt{"candy" === "8"}?
\end{enumerate}
\item [Program 3]\ 
\begin{lstlisting}
if (``'') {
    document.write(``program more'');
} else if (``cool'') {
    document.write(``party hard!'');
} else {
    document.write(``blah'');
}
\end{lstlisting}
\AnswerBox

\begin{enumerate}[label=\bfseries Q\arabic*:]\itemsep10pt
\item What will be the output for the above code snippet?
\item What happens if line 1 is changed to \texttt{if("hard") \{}?
\end{enumerate}

\item [Program 4]\
\begin{lstlisting}
var result = 0;
for (var i = 0; i < 5; i++) {
    result += i;
}
document.write(result);
\end{lstlisting}

\AnswerBox

\begin{enumerate}[label=\bfseries Q\arabic*:]\itemsep10pt
\item What will be the output for the above code snippet?
\item What happens if the expression in \texttt{for} loop is changed to \texttt{i $>$ 5}?
\item What will be the output if line 3 is changed to \texttt{result *= i;}?
\item What happens if line 2 is changed to \texttt{for (var i = 0; i < 5;) \{}?
\end{enumerate}

\item [Program 5]\

\begin{lstlisting}
while (a != 0) {
    if (a == 1) 
        continue;
    else 
        a++;
    document.write(a);
}
\end{lstlisting}

\AnswerBox

\begin{enumerate}[label=\bfseries Q\arabic*:]\itemsep10pt
\item What will be the role of the \texttt{continue} keyword in the above code snippet?
\item What will be the output if a = 2?
\item What will be the output if line 5 is changed to \texttt{a--;} for a = 3?
\end{enumerate}

\item [Program 6]\
\begin{lstlisting}
var i = 2 ;
while (i <= 10) {
    i = i + 3;
}
\end{lstlisting}

\AnswerBox

\begin{enumerate}[label=\bfseries Q\arabic*:]\itemsep10pt
\item What will be the value of i at the end of \texttt{while} loop?
\end{enumerate}

\item [Program 7]\
\begin{lstlisting}
<script>  
    var i = 21;  
    do{  
        document.write(i + ``<br/>'');  
        i++;  
    }while (i <= 25);  
</script>  
\end{lstlisting}
\AnswerBox

\begin{enumerate}[label=\bfseries Q\arabic*:]\itemsep10pt
\item What will be the output for the above code snippet?
\item What will be the output if the condition in line 6 is changed to \texttt{i >= 25}?
\item What happens if line 5 is changed to \texttt{i--;}?
\end{enumerate}

\end{description}
\subsection*{Additional Exercises}
\begin{enumerate}
\item Use a \texttt{for} loop to print out the odd numbers from 1 to 20.
\item Use a \texttt{while} loop to print out the even numbers from 1 to 20.
\item Given a number, if the number is a multiple of 3 print ``Fizz". If the number i multiple of 5 print ``Buzz". If the number is multiple of 3 and 5 print ``FizzBuzz". In all other cases, print the number it-self.
\item  Write a program to print sum of digits of a given number. 
\item Write a program to check whether a particular number is perfect square or not.
\item Write a program to print prime numbers in a given range.
\item Write a program that prints fibonacci series.
\end{enumerate}
\end{document}
