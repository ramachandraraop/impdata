\documentclass[11pt,a4paper]{article}
\author{TalentSprint}
\date{}
\usepackage{fancyhdr}
\usepackage{latexsym}
\usepackage{soul}
\usepackage{verbatim}
\usepackage{graphicx}
\usepackage{array}
\usepackage{enumerate}
%\usepackage{enumitem}
\usepackage{xcolor}
\usepackage[tikz]{bclogo}
\usepackage{textcomp}
\usepackage{latexsym}
\usepackage{seqsplit} 
\usepackage{setspace}
\usepackage{listings}
\lstset{language=html,numbers=left,numberstyle=\tiny,numbersep=10pt,showstringspaces=false}
\usepackage{fancyhdr}
\headheight=14pt
\lhead{\nouppercase{}}
\rhead{\nouppercase{\leftmark}}

\graphicspath{{../Images/}}

%\pagestyle{fancy}

%========================================================================

% Lengths and widths
\addtolength{\textwidth}{2.5cm}
\addtolength{\hoffset}{0cm}
\setlength{\headsep}{-12pt} % Reduce space between header and content
\setlength{\headheight}{85pt} % If less, LaTeX automatically increases it
\renewcommand{\footrulewidth}{2pt} % Remove footer line
\renewcommand{\headrulewidth}{1pt} % Remove header line
\renewcommand{\seqinsert}{\ifmmode\allowbreak\else\-\fi} % Hyphens in seqsplit
% This two commands together give roughly
% the right line height in the tables
\renewcommand{\arraystretch}{1.3}
\onehalfspacing



% Commands
\newcommand{\SetRowColor}[1]{\noalign{\gdef\RowColorName{#1}}\rowcolor{\RowColorName}} % Shortcut for row colour
\newcommand{\mymulticolumn}[3]{\multicolumn{#1}{>{\columncolor{white}}#2}{#3}} % For coloured multi-cols
\newcolumntype{x}[1]{>{\raggedright}p{#1}} % New column types for ragged-right paragraph columns
\newcommand{\tn}{\tabularnewline} % Required as custom column type in use

% Font and Colours
\definecolor{HeadBackground}{HTML}{333333}
\definecolor{FootBackground}{HTML}{666666}
\definecolor{TextColor}{HTML}{333333}
\definecolor{DarkBackground}{HTML}{6B8E23} %{FD1AA8}
\definecolor{LightBackground}{HTML}{E8FED8} %D3FDC8
\definecolor{tit}{HTML}{FF6600}
\renewcommand{\familydefault}{\sfdefault}
\color{TextColor}
 \headsep = 25pt
% Header and Footer
\pagestyle{fancy}
\usepackage[headheight=110pt]{geometry}
\fancyhf{}% Clear header/footer

\fancyhead[r]{\includegraphics[width = 4cm, height = 2cm]{TS-Logo.png}\hspace{0cm}}

%=================================TITLE=====================================
\fancyhead[l]{\Large{\bf{\textcolor{tit}{\textrm{Functions \& Objects}}}}}
%===========================================================================

\renewcommand{\headrulewidth}{0.4pt}% Default \headrulewidth is 0.4pt
\renewcommand{\footrulewidth}{0.4pt}% Default \footrulewidth is 0pt

\rfoot{Page \thepage}
\lfoot{COPYRIGHT \textcopyright TALENTSPRINT, 2015. ALL RIGHTS RESERVED.}




\begin{document}

\section*{Introduction}

A function is a group of reusable code which can be called anywhere in your programme. This eliminates the need of writing same code again and again. This will help programmers to write modular code. You can divide your big programme in a number of small and manageable functions.

Like any other advance programming language, JavaScript also supports all the features necessary to write modular code using functions.

JavaScript allows us to write our own functions as well. This section will explain you how to write your own functions in JavaScript.

\subsection*{Function Definition}

The most common way to define a function in JavaScript is by using the function keyword, followed by a unique function name, a list of parameters (that might be empty), and a statement block surrounded by curly braces.
\begin{description}

\item[Syntax]\

\begin{verbatim}
    <script  language = "javascript" type = "text/javascript">
    <!--
        function functionname(parameter-list)
        {
             statement;
             statement;
        }
     //-->
    </script>
\end{verbatim}

\item[Example]

A simple function that takes no parameters called sayHello is defined here:\\
\begin{verbatim}
    <script language = "javascript" type = "text/javascript">
    <!--
        function sayHello()
        {
            alert("Hello there");
        }
     //-->
    </script>
\end{verbatim}
\end{description}

\subsection*{Calling a Function}

To invoke a function somewhere later in the script, you would simple need to write the name of that function as follows:\\
\begin{verbatim}
    <script language = "javascript" type = "text/javascript">
    <!--
        sayHello();
    //-->
   </script>
\end{verbatim}

\subsection*{Function Parameters}

Till now we have seen function without a parameters. But there is a facility to pass different parameters while calling a function. These passed parameters can be captured inside the function and any manipulation can be done over those parameters. A function can take multiple parameters separated by comma.

\begin{description}
\item[Example]\

\begin{verbatim}
    <script language = "javascript" type = "text/javascript">
    <!--
        function sayHello(name, age)
        {
            alert( name + " is " + age + " years old.");
        }
     //-->
    </script>
\end{verbatim}

\begin{bclogo}[couleur=blue!5, arrondi=0.3, logo=\bctrombone]{Note}
We are using + operator to concatenate string and number all together. JavaScript does not mind in adding numbers into strings.
\end{bclogo}

Now we can call this function as follows:

\begin{verbatim}
    <script language = "javascript" type = "text/javascript">
    <!--
        sayHello('Zara', 7 );
    //-->
    </script>
\end{verbatim}
\end{description}

\subsection*{Return Statement}

A JavaScript function can have an optional return statement. This is required if you want to return a value from a function. This statement should be the last statement in a function.

\begin{description}
\item[Example]\

This function takes two parameters and concatenates them and return resultant in the calling program:

\begin{verbatim}
    <script language = "javaascript" type = "text/javascript">
    <!--
        function concatenate(first, last) {
             var full;
             full = first + last;
             return  full;
        }
    //-->
    </script>
\end{verbatim}
\end{description}
Now we can call this function as follows:


\begin{verbatim}
    <script language = "javascript"  type = "text/javascript">
    <!--
        var result;
        result = concatenate('Zara', 'Ali');
        alert(result );
    //-->
    </script>
\end{verbatim}

\end{document}
