\documentclass[11pt,a4paper]{article}
\usepackage{graphicx}
\usepackage{listings}
\lstset{language=C,numbers=left,numberstyle=\tiny,numbersep=10pt,showstringspaces=false}
\usepackage{array}
\usepackage{enumitem}

\def\AnswerBox{\fbox{\begin{minipage}{4in}\hfill\vspace{0.5in}\end{minipage}}}

\begin{document}
\section*{Practice Exercise}
\subsection*{Review Questions}
\begin{enumerate}\itemsep10pt
\item What are the possible values of the ``Position'' attribute?\underline{\hspace{3cm}}.
\item If \texttt{static} is specified, values for the top, bottom, right, and left properties will have no effect.(True / False)\underline{\hspace{3cm}}.
\item The \texttt{absolute} value of the \texttt{position} property specifies\underline{\hspace{3cm}}.
\item If no value is given to the \texttt{position} property, its value is\underline{\hspace{3cm}}.
\item If \texttt{position = relative} then the \texttt{top} property specifies\underline{\hspace{3cm}}.
\item If \texttt{position = absolute} then the \texttt{bottom} property specifies\underline{\hspace{3cm}}.
\item \texttt{left:-10px;} means that the text is 10\% (or 50px) away from the default location.(True / False)\underline{\hspace{3cm}}.
\item \texttt{right} property can be specified as a percentage, as a length, or `auto'.(Yes / No)\underline{\hspace{3cm}}.
\item Possible values of CSS \texttt{overflow} property are\underline{\hspace{3cm}}.
\item When there is an overlap between two elements, the\underline{\hspace{3cm}}value determines which one shows up on top. 
\item \texttt{Overflow:hidden;} does what?\underline{\hspace{3cm}}.
\item Which properties are used to position objects?\underline{\hspace{3cm}}.
\item  The \texttt{z-index} property can only use positive numbers, and starts on 1.(True / False)\underline{\hspace{3cm}}.
\item Applying \texttt{position:relative} to an element will cause the element to be positioned relative to\underline{\hspace{3cm}}.
\item Applying \texttt{position:absolute} to an element will cause the element to be positioned relative to\underline{\hspace{3cm}}.
\item What positioning makes an element stay in the same spot, even when the window is scrolled?\underline{\hspace{3cm}}.
\item What direction(s) is an element able to \texttt{float}?\underline{\hspace{3cm}}.
\item What is the default position for all elements on a page?\underline{\hspace{3cm}}.
\item Can you specify more than one CSS class for any HTML element?(Yes / No)\underline{\hspace{3cm}}.
\item What integer property values can we use with the `z-index' CSS property?\underline{\hspace{3cm}}.
\end{enumerate}

\subsection*{Exercises}
\begin{itemize}
\item Write the expected output, or compiler errors if any, for each of the following programs in the box provided below each program.
\item Then execute the programs and check your answers.
\item Then answer the questions given below.
\end{itemize}
\begin{description}
\item [Program 1]\
\begin{lstlisting}
<div id = ``container''>
    <div id = ``navigation''>...</div>
</div>

#navigation {
    position: absolute;
    left: 30px;
    top: 5px
}
\end{lstlisting}
\AnswerBox

\begin{enumerate}[label=\bfseries Q\arabic*:]\itemsep10pt
\item Write an equivalent HTML code for the above CSS code snippet?
\item Write an equivalent CSS code to position the navigation exactly 30px from the left and 5px from the top of the container box without using \texttt{positioning}?
\item What would be the \texttt{position} value of the container box?
\end{enumerate}
\item [Program 2]\
\begin{lstlisting}
<!DOCTYPE html>
<html>
    <head>
        <style>
            h2 {
                position: absolute;
                left: 100px;
                top: 150px;
            }
        </style>
    </head>
    <body>
        <h2>This heading has an absolute position</h2>
        <p>With absolute positioning, an element can be placed
             anywhere  on a page. The heading below is placed  
             100px from the left of the page and 150px from the 
             top of the page.
        </p>
    </body>
</html>
\end{lstlisting}

\AnswerBox

\begin{enumerate}[label=\bfseries Q\arabic*:]\itemsep10pt
\item What happens if line 6 is changed to \texttt{position:fixed}?
\item What happens if line 6 is changed to \texttt{position:relative}?
\item What happens if line 6 is changed to \texttt{position:static}?
\item What happens if line 8 is changed to \texttt{top: -100px;}?
\end{enumerate}

\item [Program 3]\
\begin{lstlisting}
<!DOCTYPE html>
<html>
    <head>
        <style>
            img {
                position: absolute;
                left: 0px;
                top: 0px;
                z-index: -1;
            }
        </style>
    </head>
    <body>
        <h1>This is a heading</h1>
        <img src=``wise.png'' width=``100'' height=``140''>
        <p>WISE module - III</p>
    </body>
</html>
\end{lstlisting}
\AnswerBox

\begin{enumerate}[label=\bfseries Q\arabic*:]\itemsep10pt
\item What does \texttt{z-index:-1} will do in the above code snippet?
\item What happens if line 9 is changed to \texttt{z-index:1}?
\item What happens if line 6 is changed to \texttt{position:static}?
\end{enumerate}
\item [Program 4]\
\begin{lstlisting}
<div class=``article''>
    <h1>Buddy In-law</h1>
    <p> A friend of a friend of yours. 
    <a href=``https://www.ts.com/?q=someone''>Someone</a>
    you have yet to meet, but is friends with your friend. 
    Someone you share a mutual friend with. 
    Someone you share a mutual buddy with. </p>
</div>
\end{lstlisting}
\AnswerBox
\begin{enumerate}[label=\bfseries Q\arabic*:]\itemsep10pt
\item How would you change the paragraph ($<$p$>$) tag font to \texttt{Verdana}?
\item How would you change the font size of the heading to a 20px?
\item How would you make the whole article 600px wide and center the heading?
\end{enumerate}
\end{description}
\subsection*{Additional Exercises}
\begin{enumerate}

\item Use your knowledge of advanced CSS positioning to build a pizza company website!

\item Use text to create a new web page. Use CSS to format text, elements, DIVs, list. Apply properties text-transform, list-style-image, text-shadow, box-shadow. Use positioned DIVs to arrange text into side-by-side columns.
\item Create a web page that includes \texttt{your name} and at least:\

\textbf{HTML elements:}
\begin{itemize}
\item title tag for the page which includes your name
\item two heading types
\item two paragraphs
\item a list - either numbered or bulleted
\item a table - either a data table or positioning table
\item a horizontal line
\item a link and an email link
\item an image
\end{itemize}
\textbf{CSS:}
\begin{itemize}
\item two different font-families and two font sizes
\item background color or background image for the page or a table or a DIV
\item an external style sheet with at least BODY style and P style
\item an internal style sheet with two styles
\item an inline style
\item one image that is floating
\item a positioned element - DIV, table, or something else
\end{itemize}
\end{enumerate}
\end{document}




