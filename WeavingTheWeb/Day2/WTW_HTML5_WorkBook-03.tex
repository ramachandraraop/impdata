\documentclass[11pt,a4paper]{article}
\usepackage{graphicx}
\usepackage{listings}
\lstset{language=C,numbers=left,numberstyle=\tiny,numbersep=10pt,showstringspaces=false}
\usepackage{array}
\usepackage{enumitem}

\def\AnswerBox{\fbox{\begin{minipage}{4in}\hfill\vspace{0.5in}\end{minipage}}}

\begin{document}
\section*{Practice Exercise}
\subsection*{Review Questions}
\begin{enumerate}\itemsep10pt
\item Which input type defines a week and year control (no time zone)? \underline{\hspace{3cm}}
\item The \underline{\hspace{2cm}} attribute value specifies that autocomplete is disabled 
\item Write two input types with which autocomplete attribute works \underline{\hspace{3cm}}
\item The \underline{\hspace{2cm}}  attribute specifies that an $<$input$>$ element should automatically get focus when the page loads.
\item The formaction attribute is only used for buttons with type \underline{\hspace{2cm}}
\item The autofocus attribute is a \underline{\hspace{2cm}} attribute.
\item The formaction attribute overrides the form's \underline{\hspace{2cm}} attribute.
\item The \underline{\hspace{2cm}} specifies where to send the form-data when a form is submitted.
\item The \underline{\hspace{2cm}} attribute defines the HTTP method for sending form-data to the action URL.
\item Write two attribute values of formmethod \underline{\hspace{3cm}}
\item The \underline{\hspace{2cm}} attribute specifies the minimum value for an $<$input$>$ element.
\item The min attribute together with the \underline{\hspace{2cm}} attribute to create a range of legal values.
%\item The \underline{\hspace{2cm}} tag defines some content aside from the content it is placed in.
\item The \underline{\hspace{2cm}} attribute specifies that an input field must be filled out before submitting the form.
\item The \underline{\hspace{2cm}} attribute specifies a name or a keyword that indicates where to display the response that is received after submitting the form.
%\item The \underline{\hspace{2cm}} element specifies independent, self-contained content.
%\item The $<$figcaption$>$ element is used to add a caption for the \underline{\hspace{2cm}} element.
%\item The \underline{\hspace{2cm}} element should be used as a container for introductory content.
%\item The HTML \underline{\hspace{2cm}} element represents highlighted text.
%\item The \underline{\hspace{2cm}} tag defines a set of navigation links.
\end{enumerate}

\subsection*{Exercises}
\begin{itemize}
\item Complete the following  exercises

\item[Exercise-1]\
\begin{enumerate}[label=\bfseries Q\arabic*:]\itemsep10pt
\item Create a Java Tutorial application with heading as ``Java Tutorial'' with a logo as a home page.Descibe about What is Java,Where it is used.
\item Give the links to ``Basics of Java'', ``Operators In Java'',``Loop Control'' on the left side of home page and explain.
\end{enumerate}

\item[Exercise-2]\
\begin{enumerate}[label=\bfseries Q\arabic*:]\itemsep10pt
\item Design a HTML page with a heading `'Structural Things used in UML'' in the center and explain.
\item List all the Structural things, give a link to every item in the list and expalin 
\end{enumerate}

\item[Exercise-3]\
\begin{enumerate}[label=\bfseries Q\arabic*:]\itemsep10pt
\item Create a HTML page to descibe about oops concepts, give the title as ``object oriented programming''.
\item List all oop's concepts as a links.
\item Explain each concept with examples.
\end{enumerate}

\end{itemize}

\subsection*{Additional Exercises}
\begin{enumerate}
\item Create a Registration Form with First Name,Last Name,Gender,Street Address,City,Country,Postal Code,Email,UserName,Password as a fileds with a submitt and Reset Buttons using html5 syntax.
\item Create a web page about your college with a tittle and an image in the center of the page.Add links Home,About Us,Placement,Research,Gallery,Contact. 
\item Create a feedback page with fields as Name, Email Address,Your age,Your State,How did you find our site,Site suggestions,Other comments, with submit and Reset buttons.
\end{enumerate}
\end{document}
