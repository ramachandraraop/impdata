\documentclass[14pt]{beamer}
\title{WEB :: CSS}
\author[TS]{TalentSprint}
\institute[L\&D]{Licensed To Skill}
\usefonttheme{serif}
\usecolortheme{orchid}
\usepackage{bookman}
\usepackage{multirow}
\usepackage{fancybox}
\usepackage[percent]{overpic}
\usepackage{hyperref}
\usepackage[T1]{fontenc}
\usepackage{graphicx}
\usepackage{listings}
\graphicspath{{../Images/}}
\usepackage{tikz}
\usepackage{soul}
\usepackage{color}
\beamertemplateballitem
\usebackgroundtemplate{\includegraphics[width=\paperwidth]{TS-XP-Logo.jpg}}
\lstset{language=html, numbers=left, numbers=none, basicstyle=\footnotesize, numberstyle=\tiny,  numbersep=10pt, showstringspaces=false, breaklines=true,keepspaces=true, columns=flexible}
\begin{document}

\begin{frame}
  \titlepage
\end{frame}

\begin{frame}{CSS}
At the end of this session, the participants will be able to:
  \begin{itemize}
  \item Understand CSS selectors
  \end{itemize}
\end{frame}

\begin{frame}[fragile]{CSS}
\textbf{Classes}
\begin{itemize}
 \item CSS class allows you to style items within the same (X)HTML element differently.
 \item A class selector begins with a (.) period
\end{itemize}
\begin{lstlisting}[xleftmargin=30pt]
p {
    font-size: small;
    color: #FF0000
}

.greenboldtext{
    font-size: small;
    color: #00FF00;
    font-weight: bold;
}
\end{lstlisting}
\end{frame}

\begin{frame}[fragile]{CSS}
\begin{lstlisting}[xleftmargin=20pt]
p {
    font-size: small;
    color: #FF0000
}
.greenboldtext {
    font-size: small;
    color: #00FF00;
    font-weight: bold;
}
<p>
To put it more simply, this <span class = ``greenboldtext''> sentence </span> you are reading is styled in my CSS file by the following.
</p>
\end{lstlisting}
\small
\textbf{Preview:}
\color{red}{To put it more simply, this} \textbf{\color{green}{sentence}} you are reading is styled in my CSS file by the following.
\end{frame}

\begin{frame}{CSS}
\textbf{IDs}
\begin{itemize}
 \item CSS IDs are similar to the classes, except that once a specific ID has been declared, it cannot be used again within the same (X)HTML file.
 \item IDs are used to style the layout elements of a page that are required only once.
 \item Classes are used to style text such that they may be declared multiple times.
\end{itemize}
\end{frame}

\begin{frame}[fragile]{CSS}
\textbf{IDs}
\begin{lstlisting}
#container {
     width: 80%;
   padding: 20px;
    border: 1px solid blue;
background: yellow;
}
<div id = ``container''>
    Everything within my document is inside this division.
</div>
\end{lstlisting}
\begin{figure}[H]
\includegraphics[scale=.25]{css-Id.png}
\end{figure}
\end{frame}

\begin{frame}[fragile]{CSS}
\textbf{Divisions}
\begin{itemize}
 \item Divisions are a block level (X)HTML element used to define sections of an (X)HTML file.
 \item It can contain all the parts that are required to build a website.
\end{itemize}
\begin{lstlisting}[xleftmargin=32pt]
<div id = ``container''>
    Site contents go here
</div>
\end{lstlisting}
\end{frame}

\begin{frame}[fragile]{CSS}
\textbf{Divisions}
\begin{block}{The CSS file contains:}
\begin{lstlisting}
#container {
    width: 70%;
    margin: auto;
    padding: 20px;
    border: 1px solid #666;
    background: #ffffff;
}
\end{lstlisting}
\end{block}
You can use both classes and IDs with a division tag to style the sections of a website.
\end{frame}

\begin{frame}[fragile]{CSS}
\textbf{Spans}
\begin{itemize}
 \item Spans are very similar to divisions except they use an inline element declaration.
 \item Span tag can also style certain areas of the text:
\end{itemize}
\begin{lstlisting}[xleftmargin=32pt]
<span class = ``italic''> This text is italic </span>

.italic {
    font-style: italic;
}
\end{lstlisting}
\textbf{Result:} \em{This text is italic}
\end{frame}

\begin{frame}{CSS}
\textbf{Borders}

\vspace{1pc}
\begin{tabular}{|p{3.5cm} | p{6cm} |}
\hline
\textbf{Property} & \textbf{values}\\ \hline
border-color & color name hexadecimal number RGB color code\\ \hline
border-style & dashed, dotted, double, groove, hidden, inset, none, outset, ridge, solid \\ \hline
border-width &  medium, thin, thick, length  \\ \hline
\end{tabular}
\end{frame}

\begin{frame}[fragile]{CSS}
\textbf{Borders}

\small
\vspace{.5pc}
We can use the following properties in any combination:
\begin{figure}[H]
\centering
\includegraphics[scale=.35]{css-borders.png}
\end{figure}
\begin{block}{For Example}
\begin{lstlisting}
border: #FF00FF 15px dotted; (For All Borders)
border-color: #FF00FF; (For all Borders color is red)
border-left: #FF00FF 15px dotted;  (Only Left Border)
border-bottom-style: dashed;
\end{lstlisting}
\end{block}
\end{frame}

\begin{frame}{CSS}
\textbf{Margins}

\begin{itemize}
 \item Clear(gap) an area around an element (outside the border)
 \item Do not have a background color, and they are completely transparent.
\end{itemize}
\end{frame}

\begin{frame}{CSS}
\textbf{Padding}
\begin{itemize}
 \item Clears(gap) an area around an element (outside the border)
 \item Does not have a background color, and is completely transparent.
\end{itemize}
\begin{figure}[H]
 \centering
 \includegraphics[scale=.5]{margins-padding.png}
\end{figure}
\end{frame}

\begin{frame}{CSS}
\textbf{Miscellaneous Tags}

\small
\begin{tabular}{|p{4cm} | p{5cm} |}
 \hline
 \textbf{Property} & \textbf{Value} \\ \hline
 Height\newline max-height\newline min-height& Number \\ \hline
 Width\newline max-width\newline min-width & Number \\ \hline
 Cursor & auto crosshair default help move pointer text url wait e-resize ne-resize nw-resize n-resize se-resize sw-resize w-resize \\ \hline
 Float & left, right and none \\ \hline
\end{tabular}
\end{frame}

\begin{frame}{CSS}
\textbf{Miscellaneous Tags}

\vspace{.5pc}
\begin{tabular}{|p{4cm} | p{5cm} |}
 \hline
 \textbf{Property} & \textbf{Value} \\ \hline
 Left & Number \\ \hline
Right & Number \\ \hline
Top & Number \\ \hline
Bottom & Number \\ \hline
Position & Absolute, Relative, Fixed and None \\ \hline
Overflow & Auto, Hidden, Visible, Scroll \\ \hline
Visibility & Hidden, Visible \\ \hline
Z-index & Number \\ \hline
 \end{tabular}
\end{frame}

\begin{frame}{CSS}
 \begin{figure}[H]
 \begin{center}
   \includegraphics[scale=.3]{qa.png}   
 \end{center}
  \end{figure}
\end{frame}
\end{document}
