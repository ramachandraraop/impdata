\documentclass[11pt,a4paper]{article}
\usepackage{graphicx}
\usepackage{listings}
\lstset{language=C,numbers=left,numberstyle=\tiny,numbersep=10pt,showstringspaces=false}
\usepackage{array}
\usepackage{enumitem}

\def\AnswerBox{\fbox{\begin{minipage}{4in}\hfill\vspace{0.5in}\end{minipage}}}

\begin{document}
\section*{Practice Exercise}
\subsection*{Review Questions}
\begin{enumerate}\itemsep10pt
\item The \underline{\hspace{2cm}} tag defines content aside from the page content.
%\item The \underline{\hspace{2cm}} attribute specifies a name or a keyword that indicates where to display the response that is received after submitting the form.
\item The \underline{\hspace{2cm}} element specifies independent, self-contained content.
\item The $<$figcaption$>$ element is used to add a caption for the \underline{\hspace{2cm}} element.
\item The \underline{\hspace{2cm}} element should be used as a container for introductory content.
\item The HTML \underline{\hspace{2cm}} element represents highlighted text.
\item The \underline{\hspace{2cm}} tag specifies additional details that the user can view or hide on demand.
\item The \underline{\hspace{2cm}} tag defines a set of navigation links.
\item The \underline{\hspace{2cm}} attribute of $<$details$>$ tag specifies that the details should be visible (open) to the user
\item The \underline{\hspace{2cm}} tag defines a footer for a document or section.
\item Which tag is used to define sections in a document?\underline{\hspace{2cm}}
\item The \underline{\hspace{2cm}} tag is used displaying diagrams, photos, code listings etc.
\end{enumerate}

\subsection*{Exercises}
\begin{itemize}
%\item Write the expected output for each of the following programs in the box provided below each program.
%\item Then execute the programs and check your answers.
%\item Then answer the questions given below.
\item Complete the following  exercises

\item[Exercise-1]\
\begin{enumerate}[label=\bfseries Q\arabic*:]\itemsep10pt
\item Create Web application with a tittle ``Education center'' with alogo and a caption as ``THE WORLD OF KNOWLEDGE'' with an image.
\item Add links to Home, Calendar,Programs,Students,Contacts in the header.
\item Describe about each link
\item Also add copyright information in the bottom of the page.
\end{enumerate}

\item[Exercise-2]\
\begin{enumerate}[label=\bfseries Q\arabic*:]\itemsep10pt
\item Design a HTML page with a heading ``UML Diagrams'' in the center of the page.
\item Explain the differnt categories of UML diagrams.
\item Highlight the imporant text.
\item Add a link to diagrams under each category.
\item Explain each diagram with an example and an image.
\end{enumerate}

\item[Exercise-3]\
\begin{enumerate}[label=\bfseries Q\arabic*:]\itemsep10pt
\item Create a webpage to explain about Software Engineering Process.
\item Explain about what is a Software development life cycle and differnt phases in SDLC.
\item Use appropriate images and examples.
\end{enumerate}

\end{itemize}

\subsection*{Additional Exercises}
\begin{enumerate}
\item Create a web page about your favourite book.Add description of a book, include the title of the book as well as its author. Names and titles should be underlined, adjectives should be italicized and bolded.
\item Create a web page about the school you have studied using all the semantic types.
\item Create a new web page with title ``my summer vacation page'' and add the heading ``my vacation to Kulu Manali'' to the body and explain.
\end{enumerate}
\end{document}
