\documentclass[11pt,a4paper]{article}
\usepackage{graphicx}
\usepackage{listings}
\lstset{language=C,numbers=left,numberstyle=\tiny,numbersep=10pt,showstringspaces=false}
\usepackage{array}
\usepackage{enumitem}

\def\AnswerBox{\fbox{\begin{minipage}{4in}\hfill\vspace{0.5in}\end{minipage}}}

\begin{document}
\section*{\center{Practice Exercises}}
\subsection*{Review Questions}
\begin{enumerate}\itemsep10pt
\item What are the built-in objects in JavaScript?\underline{\hspace{3cm}}.
\item What are the different ways to create objects in JavaScript?\underline{\hspace{3cm}}.
\item Does JavaScript has a built-in concept of inheritance?\underline{\hspace{3cm}}.
\item How do you create an object in JavaScript?\underline{\hspace{3cm}}.
\item How to you assign a value to an object?\underline{\hspace{3cm}}.
\item  What does ``1'' + 2 + 4 evaluate to?\underline{\hspace{3cm}}.
\item A string function which returns the character at the specified position is\underline{\hspace{3cm}}.
\item Given the following line of code, what would be the output?\underline{\hspace{3cm}}.
\begin{lstlisting}
var x1 = ``Hello RAM'';
document.write(``indexOf '' + x1.indexOf(``RAM''));
\end{lstlisting} 
\item Given the following line of code, what would be the output?\underline{\hspace{3cm}}.
\begin{lstlisting}
var x1 = ``Hello World''; 
document.write(x1.substring(5)); 
document.write(x1.substring(3,6));
\end{lstlisting}
\item What is the use of Math function in JavaScript?\underline{\hspace{3cm}}.
\item What's Math Constants using JavaScript?\underline{\hspace{3cm}}.
\item \texttt{Math.ceil(29.89);} returns\underline{\hspace{3cm}}.
\item A Math function which rounds a number downward to its nearest integer is\underline{\hspace{3cm}}.
\item \texttt{Math.pow(4, Math.pow(2,3));} returns\underline{\hspace{3cm}}.
\item How to create a Date object in JavaScript?\underline{\hspace{3cm}}.
\item Given the following line of code, what would be the output?\underline{\hspace{3cm}}.
\begin{lstlisting}
var d = new Date(); 
document.write (d.getDay());
\end{lstlisting}
\item How does JavaScript store dates in a date object?\underline{\hspace{3cm}}.
\end{enumerate}

\subsection*{Exercises}
\begin{itemize}
\item Write the expected output, or compiler errors if any, for each of the following programs in the box provided below each program.
\item Then execute the programs and check your answers.
\item Then answer the questions given below.
\end{itemize}
\begin{description}
\item[program 1]\
\begin{lstlisting}
<script type = ``text/JavaScript'' language = ``JavaScript''>
    var s_prim = ``foo'';
    var s_obj = new String(s_prim);  
    document.write(typeof s_prim + ``<br>''); 
    document.write(typeof s_obj);  
</script>
\end{lstlisting}

\AnswerBox

\begin{enumerate}[label=\bfseries Q\arabic*:]\itemsep10pt
\item What will be the outupt for the above code snippet?
\item What happens if line 5 is changed to \texttt{document.write(s\_obj);}?

\end{enumerate}

\item[Program 2]\
\begin{lstlisting}
<script type = ``text/JavaScript'' language = ``JavaScript''>
    document.write(String(`Hello') = = = `Hello');
    document.write(new String(`Hello') = = = `Hello');
    document.write(new String(`Hello') = = `Hello');
</script>
\end{lstlisting}

\AnswerBox

\begin{enumerate}[label=\bfseries Q\arabic*:]\itemsep10pt
\item What will be the output for the above code snippet?
\item What happens if line 1 is changed to \texttt{document.write(String(`Hello') == `Hello');}?
\end{enumerate}
\item [Program 3]\ 
\begin{lstlisting}
<script type = ``text/JavaScript'' language = ``JavaScript''>
    var num1 = ``10'', num2 = ``9'';
</script>
\end{lstlisting}
\AnswerBox

\begin{enumerate}[label=\bfseries Q\arabic*:]\itemsep10pt
\item What is the value of num1 $<$ num2?
\item What is the value of +num1 $<$ num2?
\item What is the value of num1 + num2?
\item What is the value of +num1 + num2?
\end{enumerate}

\item [Program 4]\
\begin{lstlisting}
<script type = ``text/JavaScript'' language = ``JavaScript''>
    var message = ``Hello world!'';
</script>
\end{lstlisting}

\AnswerBox

\begin{enumerate}[label=\bfseries Q\arabic*:]\itemsep10pt
\item What is the value of message.substring(1, 4)?
\item What is the value of message.substr(1, 4)?
\end{enumerate}

\item [Program 5]\

\begin{lstlisting}
<script type = ``text/JavaScript'' language = ``JavaScript''>
    var ts = ``wise module'';
    var result = ts.lastIndexOf(``s''); 
    document.write(result);
</script>
\end{lstlisting}

\AnswerBox

\begin{enumerate}[label=\bfseries Q\arabic*:]\itemsep10pt
\item What will be the output for the above code snippet?
\item What happens if line 3 is changed to \texttt{var result = ts.substring(5);}?
\item What will be the output if line 3 is changed to \texttt{var result = ts.charAt(2);}?
\end{enumerate}

\item [Program 6]\
\begin{lstlisting}
<script type = ``text/JavaScript'' language = ``JavaScript''>
    var text = prompt (``Enter Integer for var N '',``'') 
    var num1 = parseInt(text)
    var text = prompt (``Enter decimal number for var x '',``'') 
    var num2 = parseInt(text)
    result = Math.pow(num1, num2)
    document.write(``power of numbers is = '' + result);
</script>
\end{lstlisting}

\AnswerBox

\begin{enumerate}[label=\bfseries Q\arabic*:]\itemsep10pt
\item What will be the output for the above code snippet?
\item What happens if the input value for num1 = ``2'' and num2 = 3?
\item What happens if the input value for num1 = 2.5 and num2 = 3.9?
\end{enumerate}

\item [Program 7]\
\begin{lstlisting}
<script type = ``text/JavaScript'' language = ``JavaScript''>
    var ran_number = Math.random() * 5;
    document.write(ran_number); 
</script>
\end{lstlisting}
\AnswerBox

\begin{enumerate}[label=\bfseries Q\arabic*:]\itemsep10pt
\item What will be the output of the above code snippet?
\item Rewrite the code such that it prints only integer values between 0 and 4?
\end{enumerate}

\item [Program 8]\
\begin{lstlisting}
<script type = ``text/JavaScript'' language = ``JavaScript''>
    var d = new Date();
    document.write(d.getHours() + ``<br>'');
    document.write(d.getDate() + ``<br>'');
    document.write(d.getDay() + ``<br>'');
    document.write(d.getFullYear() + ``<br>'');
</script>
\end{lstlisting}
\AnswerBox

\begin{enumerate}[label=\bfseries Q\arabic*:]\itemsep10pt
\item What will be the output of the above code snippet?
\end{enumerate}
\end{description}
\subsection*{Additional Exercises}
\begin{enumerate}
\item Given a string of numbers in sequence order. Find the missing number.\
\texttt{Input: "9899100101103104105"\newline
        Output: 102}

\item A string of characters are given, find the highest occurance of the character and display it.\

\texttt{Input: MISSISIPI\newline
        Output: I}
\item Write a JavaScript that receives a string composed by words separated by spaces and returns a string where words appear in the same order but than the original string, but every word is inverted.\

\texttt{Input: the boy ran\newline
        Output: eht yob nar}
\item Write a JavaScript to reverse the words in a given sentence.
\item Write a JavaScript to search for the existence of a string(target) in another string. The program takes two strings as the input and returns the index where the second string is found. If the target string cannot be found, then return -1.
\item You have a string str = ``I love you''. Write a program to print the output as ``you love I''.
\item Write a JavaScript to remove the characters from string1 which are present in string2 and display the resultant string.
\item How to write a random number from 0 to 10, using the round and the random method.
\item Write JavaScript to find out the circumference of a circle using Math.PI .
\item Write JavaScript to display today's date including date, month, and year. Note that the getMonth method returns 0 in January, 1 in February etc. So add 1 to the getMonth method to display the correct date.
\item Write JavaScript to display the current local time including hour, minutes, and seconds. To return the GMT time use getUTCHours, getUTCMinutes etc.
\item How to write a complete date with the name of the day and the name of the month.
\end{enumerate}
\end{document}
