\documentclass[11pt,a4paper]{article}
\author{TalentSprint}
\date{}
\usepackage{fancyhdr}
\usepackage{latexsym}
\usepackage{soul}
\usepackage{verbatim}
\usepackage{graphicx}
\usepackage{array}
\usepackage{enumerate}
%\usepackage{enumitem}
\usepackage{xcolor}
\usepackage[tikz]{bclogo}
\usepackage{textcomp}
\usepackage{latexsym}
\usepackage{seqsplit} 
\usepackage{setspace}
\usepackage{listings}
\lstset{language=html,numbers=left,numberstyle=\tiny,numbersep=10pt,showstringspaces=false}
\usepackage{fancyhdr}
\headheight=14pt
\lhead{\nouppercase{}}
\rhead{\nouppercase{\leftmark}}

\graphicspath{{../Images/}}

%\pagestyle{fancy}

%========================================================================

% Lengths and widths
\addtolength{\textwidth}{2.5cm}
\addtolength{\hoffset}{0cm}
\setlength{\headsep}{-12pt} % Reduce space between header and content
\setlength{\headheight}{85pt} % If less, LaTeX automatically increases it
\renewcommand{\footrulewidth}{2pt} % Remove footer line
\renewcommand{\headrulewidth}{1pt} % Remove header line
\renewcommand{\seqinsert}{\ifmmode\allowbreak\else\-\fi} % Hyphens in seqsplit
% This two commands together give roughly
% the right line height in the tables
\renewcommand{\arraystretch}{1.3}
\onehalfspacing



% Commands
\newcommand{\SetRowColor}[1]{\noalign{\gdef\RowColorName{#1}}\rowcolor{\RowColorName}} % Shortcut for row colour
\newcommand{\mymulticolumn}[3]{\multicolumn{#1}{>{\columncolor{white}}#2}{#3}} % For coloured multi-cols
\newcolumntype{x}[1]{>{\raggedright}p{#1}} % New column types for ragged-right paragraph columns
\newcommand{\tn}{\tabularnewline} % Required as custom column type in use

% Font and Colours
\definecolor{HeadBackground}{HTML}{333333}
\definecolor{FootBackground}{HTML}{666666}
\definecolor{TextColor}{HTML}{333333}
\definecolor{DarkBackground}{HTML}{6B8E23} %{FD1AA8}
\definecolor{LightBackground}{HTML}{E8FED8} %D3FDC8
\definecolor{tit}{HTML}{FF6600}
\renewcommand{\familydefault}{\sfdefault}
\color{TextColor}
 \headsep = 25pt
% Header and Footer
\pagestyle{fancy}
\usepackage[headheight=110pt]{geometry}
\fancyhf{}% Clear header/footer

\fancyhead[r]{\includegraphics[width = 4cm, height = 2cm]{TS-Logo.png}\hspace{0cm}}

%=================================TITLE=====================================
\fancyhead[l]{\Large{\bf{\textcolor{tit}{\textrm{JQuery}}}}}
%===========================================================================

\renewcommand{\headrulewidth}{0.4pt}% Default \headrulewidth is 0.4pt
\renewcommand{\footrulewidth}{0.4pt}% Default \footrulewidth is 0pt

\rfoot{Page \thepage}
\lfoot{COPYRIGHT \textcopyright TALENTSPRINT, 2015. ALL RIGHTS RESERVED.}




\begin{document}

\section*{Introduction}
jQuery is a JavaScript based library that makes using java script much easier for your web based applications. JQuery is fast, much easy to use, small in size – around 32Kb, and contains lot of features.

You have to write lot of JavaScript code to accomplish tasks that jQuery enables you to write with much less code. Its aim is to write less do more.

jQuery is cross platform compatible, i.e. supported by many popular browsers like Chrome, Firefox, IE, Safari and others.

jQuery makes it quite simple like HTML document transversal and manipulation, handling events (like clicking on links, mouse up, mouse down etc.), animations for your applications (like slider, loader et.) and Asynchronous JavaScript and XML (AJAX) operations with an easy to use API.

You can do many things or seen, as you might noticed in different web sites like cool jQuery sliders, slidshows, menus, fadein/fadeout effects, beautiful dropdowns .. just a few what can be done with jQuery.

\section*{Setting up jQuery}
Installing or setting up jQuery to enable you working in your web applications is quite simple. This does not require any compilation etc.

You just have to place or refer library file of jQuery in your HTML files to work with it. Below is how you can do it:

You can download jQuery library from \textbf{http://jquery.com/download/.}

After downloading and placing uncompressed file to your development machine, \textbf{place it to same directory where you code files are.} 
\begin{lstlisting}
<head>
    <title>jQuery Library</title>
    <script type=``text/javascript'' src=``jquery-1.11.1.js''></script>
</head>
\end{lstlisting}
Note: jquery-1.11.1.min.js in compressed version of jquery.js,It is used when website is to be hosted on client machine.

It will decrease page load-time for webpage using jQuery.
\subsection*{Syntax of jQuery}
All jQuery programs begins with instantiating document.ready event which prevent any jQuery code from running before the document is finished loading (is ready).
Syntax for jQuery:

\begin{lstlisting}
<head>
    <title>jQuery Library</title>
    <script type=``text/javascript'' src=``jquery-1.11.1.js''></script>
    <script>
        $(document).ready(function(){
          // Jquery statements;
        )};
    </script>
</head>
\end{lstlisting}
\section*{jQuery Selectors}
jQuery selectors are one of the most important parts of the jQuery library.

jQuery selectors allow you to select and manipulate HTML element(s).

All selectors in jQuery start with the dollar sign and parentheses i.e. \texttt{\$()}.
There are three types of selectors
\begin{itemize}
 \item ID based selector.
 \item Direct tag based Selector.
 \item Class based Selector.
\end{itemize}
\subsection*{\#ID based selector}
The jQuery \#id selector uses the id attribute of an HTML tag to find the specific element.

An id should be unique within a page, so you should use the \#id selector when you want to find a single, unique element.

To find an element with a specific id, write a hash character, followed by the id of the HTML element:

The following example illustrates the flow of jQuery program.

\begin{lstlisting}
<!DOCTYPE html>
<html>
     <head>
         <script src=``jquery-1.11.1.js''>
         </script>
         <script>
             $(document).ready(function(){
                 $(``#button1'').click(function(){
                     window.close();
                });
            });
        </script>
    </head>
    <body>
        <button id=``button1''>Close Window</button>
    </body>
</html>
\end{lstlisting}
\subsubsection*{Explanation}
In the above program at Line 4 jquery library is instantiated. Line 7 is calling document.ready event to ensure page is properly loaded or not.

At Line 8 click event is called,which will invoked when button is clicked. The click event will call window.close() method.
\subsection*{Element based selector}
\begin{lstlisting}
<!DOCTYPE html>
<html>
    <head>
        <script src=``jquery-1.11.1.js''>
        </script>
        <script>
            $(document).ready(function(){
                $(``p'').click(function(){
                    window.open(``www.yahoo.com'',``_blank'');
                });
                ${``div'').click(function(){
                    window.open(``www.amazon.com'',``_blank'')
                });
            });
        </script>
    </head>
    <body>
        <p>Click here to open yahoo.com</p>
        <div>Click here to open amazon.com</div>
    </body>
</html>
\end{lstlisting}
\subsubsection*{Explanation}
In the above example p and div are the two HTML elements,when p or div are clicked yahoo.com or amazon.com in a blank window will get opened.

Both are defined under single document.ready event.
\subsection*{Class based selector}
The jQuery class selector finds elements with a specific class.

To find elements with a specific class, write a period character, followed by the name of the class.
\begin{lstlisting}
<!DOCTYPE html>
<html>
    <head>
        <script src=``jquery-1.11.1.js''>
        </script>
        <script>
            $(document).ready(function(){
                $(``.p1'').click(function(){
                    window.open(``www.yahoo.com'',``_blank'');
                });
                ${``.div1'').click(function(){
                    window.open(``www.amazon.com'',``_blank'')
                });
            });
        </script>
    </head>
    <body>
        <p class=``.p1''>Click here to open yahoo.com</p>
        <div class=``.div''>Click here to open amazon.com</div>
    </body>
</html>
\end{lstlisting}
\subsubsection*{Explanation}
In the above example .p1 and .div1 are the two classes defined for p and div elements,when .p1 or .div1 are clicked, either yahoo.com or amazon.com in a blank window will get opened.

Both are defined under single document.ready event.
\section*{Child Selector}
Selects all child elements specified by Parent element.The syntax is parent $>$ child, The $>$ sign specifies to select all child elements of parent.
\begin{lstlisting}
<!DOCTYPE html>
<html>
    <head>
        <script src=``jquery-1.11.1.js''>
        </script>
        <script>
            $(document).ready(function(){
                $(``#div1'').click(function(){
                    $(``div > p'').css(``color'',``blue'');
                });
            });
        </script>
    </head>
    <body>
        <div id=``div1''>Click here
            <p>This will be now in blue color</p>
            <p>This also will be in blue color</p>
        </div>
    </body>
</html>
\end{lstlisting}
\subsection*{Explanation}
In the above example under div1 id there are two $<$p$>$ tags. We can select all p tags of div1 using $>$ operator.

\section*{Specific Element and its childs Selector}
When a page contains similar elements, say many $<$ul$>$ elements and only one of it has to be manipulated,jQuery provides flexible way of doing it.
\begin{lstlisting}
<!DOCTYPE html>
<html>
    <head>
        <script src=``jquery-1.11.1.js''>
        </script>
        <script>
            $(document).ready(function(){
                $("#b1'').click(function(){
                    $(``ul#ul2 > li'').css(``color'',``blue'');
                });
            });
        </script>
    </head>
    <body>
        <ul id=``ul1''>
            <li>One</li>
            <li>Two</li>
        </ul>
        <ul id=``ul2''>
            <li>One</li>
            <li>Two</li>
        </ul>
        <button id=``b1''>Color Change of ul2</button>   
    </body>
</html> 
\end{lstlisting}
\subsection*{Explanation}
In the above example there are two ul tags with different id. using jQuery only ul2 and all its childs are manipulated.
Hence we can change the behaviour of any particular element easily.


\section*{Event Handling}
Events are user’s actions on a web page that web page can respond to. 

For example pressing a keyboard key, bringing mouse pointer to a link, click that link, while page is loading, page is closed or unloading and many more actions.

jQuery is tailor-made to respond to many useful events that webpage can get. As a user triggers those events, custom function or actions can be performed by catching those events in event handlers.
\section*{Types of Events}
Events in jQuery are broadly classified as:
\begin{itemize}
 \item Mouse Events
 \item Document Events
\end{itemize}
\section*{Mouse Events}
\subsection*{click event}
Invokes when mouse is clicked.
\begin{lstlisting}
(``p'').click();
\end{lstlisting}
\subsection*{dblclick event}
Invokes when mouse is double clicked.
\begin{lstlisting}
$(document).ready(function(){
    $(``p'').dblclick(function(){
        alert(``Welcome'');
     });
}); 
\end{lstlisting}
\subsection*{mouseenter event}
Invokes when mouse enters on specified element.
\begin{lstlisting}
$(document).ready(function(){
    $(``p'').mouseenter(function(){
        alert(``Welcome'');
     });
}); 
\end{lstlisting}
\subsection*{mouseleave event}
Invokes when mouse leaves the focus from the specified element.
\begin{lstlisting}
$(document).ready(function(){
    $(``p'').mouseleave(function(){
        alert(``Welcome'');
    });
}); 
\end{lstlisting}
\subsection*{hover event}
Invokes when mouse is hovered.
\begin{lstlisting}
$(document).ready(function(){
    $(``submit1'').hover(function(){
        alert(``Welcome'');
    });
}); 
\end{lstlisting}
\section*{Form Events}
\subsection*{submit event}
jQuery provided submit() method to handle submit event where a function can be attached to perform certain actions e.g. checking form fields before submitting to database.
\begin{lstlisting}
$(document).ready(function(){
    $(``submit1'').submit(function(){
        alert(``Form will be submitted to server.'');
    });
}); 
\end{lstlisting}
\subsection*{change event}
jQuery provides change() method to perform action or attach an event handler to change event.

When value of an element like textbox, dropdown, text area etc. is changed the change event occurs.

\begin{lstlisting}
<!DOCTYPE html>
<html>
    <head>
        <script src=``jquery-1.11.1.js''></script>
        <script>
            $(document).ready(function(){
                $(``#s1'').change(function(){
                    $(``#t1'').css(``color'',$(`#s1').val());
                });	
            });
        </script>
    </head>
    <body>
    <form>
        <input type=text id=``t1''>
        <select id=``s1''>
            <option value=``red''>Red</option>
            <option value=``blue''>Blue</option>
            <option value=``green''>Green</option>
        </select>
    </form>
    </body>
</html>
\end{lstlisting}
\section*{focus event}
When user click on element or when user press tab key to move the focus from one element to another.
\begin{lstlisting}
 <!DOCTYPE html>
<html>
    <head>
        <title>Insert title here</title>
        <script src=``jquery-1.11.1.js''></script>
        <script>
            $(document).ready(function(){
                $(``#t2'').focus(function(){
                    $(``#t1'').css(``color'',``magenta'');
	        });	
            });
        </script>
    </head>
    <body>
    <form>
       <input type=text id=``t1'' value=``Changed Color''> 
       <input type=text id=``t2'' value=``Change the color of text 1''>
    </form>
    </body>
</html>
\end{lstlisting}

\end{document}