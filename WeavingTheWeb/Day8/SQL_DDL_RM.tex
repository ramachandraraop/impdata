\documentclass[11pt,a4paper]{article}

\usepackage{verbatim}
\usepackage{fancyhdr}           % For header and footer
\usepackage{multicol}           % Allows multicols in tables
\usepackage{graphicx}           % For images
\usepackage{xcolor}             % For hex colours
\usepackage{colortbl}           % For coloured tables
\usepackage{setspace}           % For line height
\usepackage{seqsplit}           % Splits long words.
\usepackage{amsmath}            % Symbols

\graphicspath{{../Images/}}
% Lengths and widths
\addtolength{\textwidth}{5cm}
\addtolength{\hoffset}{-1cm}
\setlength{\headsep}{-12pt} % Reduce space between header and content
\setlength{\headheight}{85pt} % If less, LaTeX automatically increases it
\renewcommand{\footrulewidth}{2pt} % Remove footer line
\renewcommand{\headrulewidth}{1pt} % Remove header line
\renewcommand{\seqinsert}{\ifmmode\allowbreak\else\-\fi} % Hyphens in seqsplit
% This two commands together give roughly
% the right line height in the tables
\renewcommand{\arraystretch}{1.3}
\onehalfspacing



% Commands
\newcommand{\SetRowColor}[1]{\noalign{\gdef\RowColorName{#1}}\rowcolor{\RowColorName}} % Shortcut for row colour
\newcommand{\mymulticolumn}[3]{\multicolumn{#1}{>{\columncolor{white}}#2}{#3}} % For coloured multi-cols
\newcolumntype{x}[1]{>{\raggedright}p{#1}} % New column types for ragged-right paragraph columns
\newcommand{\tn}{\tabularnewline} % Required as custom column type in use

% Font and Colours
\definecolor{HeadBackground}{HTML}{333333}
\definecolor{FootBackground}{HTML}{666666}
\definecolor{TextColor}{HTML}{333333}
\definecolor{DarkBackground}{HTML}{6B8E23} %{FD1AA8}
\definecolor{LightBackground}{HTML}{E8FED8} %D3FDC8
\definecolor{tit}{HTML}{FF6600}
\renewcommand{\familydefault}{\sfdefault}
\color{TextColor}
 \headsep = 25pt
% Header and Footer
\pagestyle{fancy}
\usepackage[headheight=110pt]{geometry}
\fancyhf{}% Clear header/footer

\fancyhead[r]{\includegraphics[width = 4cm, height = 2cm]{TS-Logo.png}\hspace{0cm}}
\fancyhead[l]{\Large{\bf{\textcolor{tit}{\textrm{Data Defination Language}}}}}

\renewcommand{\headrulewidth}{0.4pt}% Default \headrulewidth is 0.4pt
\renewcommand{\footrulewidth}{0.4pt}% Default \footrulewidth is 0pt

\rfoot{Page \thepage}
\lfoot{COPYRIGHT \textcopyright TALENTSPRINT, 2015. ALL RIGHTS RESERVED.}

\begin{document}
\vspace{5in}

\subsubsection*{How to connect to MySql}
 \$  mysql u root p
      
    Enter password: root
\subsubsection*{How to display available Databases}
 mysql$>$SHOW databases;
\subsubsection*{How to create a new Database}
 mysql$>$ CREATE DATABASE demo;
\subsubsection*{How to connect to a particular Database}
mysql$>$ USE employee;\newline

mysql$>$ CONNECT employee;
\subsubsection*{How to know which Database user is connected}
 mysql$>$SELECT DATABASE() FROM DUAL;
\subsubsection*{How to display all tables available in connected Database}
 mysql$>$SHOW TABLES;
\subsubsection*{Creating a Table}
mysql$>$ CREATE TABLE employee (empid INT, ename VARCHAR (20), salary FLOAT (10, 2));
\subsubsection*{How to create a table by copying the structure and data of an existing Table}
mysql$>$ CREATE TABLE student AS SELECT * FROM employees;
\subsubsection*{How to create a table by copying the structure only not the data of an existing Table}
mysql$>$ CREATE TABLE student AS SELECT * FROM employees where 1=2;
\subsubsection*{How to describe the structure of a Table}
mysql$>$DESC employee;
\subsubsection*{Alter}
\subsubsection*{How to add a new column to existing Table}
mysql$>$ALTER TABLE employee add address varchar(20);\newline
mysql$>$ALTER TABLE employee add address varchar(20),add phone int; 
\subsubsection*{How to modify structure of exisiting column of a Table}
mysql$>$ ALTER TABLE employee MODIFY address VARCHAR (40);\\
mysql$>$ ALTER TABLE employee MODIFY ename VARCHAR (22), MODIFY city VARCHAR (22);
\subsubsection*{How to drop exisiting column from a Table}
mysql$>$ ALTER TABLE employee DROP phone;\\
mysql$>$ ALTER TABLE employee DROP ename, DROP salary;
\subsubsection*{How to rename existing column of a Table}
mysql$>$ ALTER TABLE employee CHANGE address newaddress VARCHAR(30) ;\\
mysql$>$ ALTER TABLE employee CHANGE empid id VARCHAR (100), CHANGE newaddress address VARCHAR (100);
\subsubsection*{How to rename an exisiting Table}
mysql$>$ RENAME TABLE employee TO newemployee;\\
mysql$>$ RENAME TABLE newemployee TO employees,student TO students, department TO departments;
\subsubsection*{Drop}
\subsubsection*{How to drop existing Table}
mysql$>$DROP TABLE employee;
\subsubsection*{How to delete all records of a table permanently}
mysql$>$TRUNCATE employee;
\subsubsection*{Data Types}
\begin{itemize}
\item{\textbf{SMALLINT}: A small integer. The signed range is -32768 to 32767. The unsigned range
is 0 to 65535.}

\item{\textbf{MEDIUMINT}}: A medium-sized integer. The signed range is -8388608 to 8388607. The unsigned range is 0 to 16777215.

\item{\textbf{INT}}: A normal-size integer. The signed range is -2147483648 to 2147483647. The unsigned range is 0 to 4294967295.

\item{\textbf{BIGINT}}: A large integer. The signed range is -9223372036854775808 to 9223372036854775807. The unsigned range is 0 to 18446744073709551615.

\item{\textbf{FLOAT}}: A small (single-precision) floating-point number. Permissible values are -3.402823466E+38 to -1.175494351E-38, 0, and 1.175494351E-38 to 3.402823466E+38. These are the theoretical limits, based on the IEEE standard. The actual range might be slightly smaller depending on your hardware or operating system.

\item{\textbf{DATE}}: A date. The supported range is '1000-01-01' to '9999-12-31'. MySQL displays DATE values in 'YYYY-MM-DD' format.

\item{\textbf{TIME}} : A time. The range is '-838:59:59' to '838:59:59'. MySQL displays TIME values in 'HH:MM:SS' format.

\item{\textbf{DATETIME}}: A date and time combination. The supported range is '1000-01-01 00:00:00' to '9999-12-31 23:59:59'. MySQL displays DATETIME values in 'YYYYMM-DD HH:MM:SS' format.

\item{\textbf{YEAR[(2|4)]}}: A year in two-digit or four-digit format. The default is four-digit format.In four-digit format, the permissible values are 1901 to 2155, and 0000. In two-digit format, the permissible values are 70 to 69, representing years from 1970 to 2069.MySQL displays YEAR values in YYYY format.

\item{\textbf{TIMESTAMP}}: A timestamp. The range is '1970-01-01 00:00:01' UTC to '2038-01-19 03:14:07' UTC.

\item{\textbf{CHAR}} :The length of a CHAR column is fixed to the length that you declare when you create the table. The length can be any value from 0 to 255.

\item{\textbf{VARCHAR}}: The length can be specified as a value from 0 to 255 before MySQL 5.0.3, and 0 to 65,535 in 5.0.3 and later versions.

\end{itemize}
\end{document}
