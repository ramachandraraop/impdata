\documentclass[11pt,a4paper]{article}
\usepackage{graphicx}
\usepackage{listings}
\lstset{language=Java,numbers=left,numberstyle=\tiny,numbersep=10pt,showstringspaces=false}
\usepackage{array}
\usepackage{enumitem}
\def\AnswerBox{\fbox{\begin{minipage}{4in}\hfill\vspace{0.5in}\end{minipage}}}
\usepackage{fancyhdr}
\pagestyle{fancy}
\renewcommand{\headrulewidth}{0pt}
\rhead{\includegraphics[scale=.5]{../Images/TS-Logo.png}}
\begin{document}
\centerline{\huge{\textbf{Working With DML}}}
\vspace{1pc}
\centerline{\Large{ \textbf{Workbook}}}
\subsection*{Answer the following}
\begin{enumerate}\itemsep10pt
\item What does DDL stands for\underline{\hspace{3cm}}.
\item \underline{\hspace{3cm}} are DML statements.

\item Emp table is having three column Ename, Eno and Age. Write the correct insert statements to insert a row? \underline{\hspace{3cm}}

\item Can we insert multiple values to a table at a time? (Yes / No)

\item We can update more than one value using update statement.(True /
False)


\item \underline{\hspace{3cm}} is a query to delete whole records from table Dept?


\item \underline{\hspace{3cm}} command is used to change the existing data in a table.


\item \underline{\hspace{3cm}} keyword is used to specify conditional search.


\item The FROM keyword is used to\underline{\hspace{3cm}}.


\item \underline{\hspace{3cm}}command is used to add records to a table.

\item \underline{\hspace{3cm}}command is used to remove rows from a table.

\item \underline{\hspace{3cm}}clause is used to restrict the rows returned by a query

\end{enumerate}
\subsection*{Exercises}
\begin{itemize}
\item Write the expected output, or compiler errors if any, for each of the following programs.
\end{itemize}
\begin{description}
\item[program 1]\
\begin{verbatim}
SQL> DESC employees;
 Name                    Null?    Type
 ----------------------- -------- ----------------
 EMPLOYEE_ID             NOT NULL INT(6)
 FIRST_NAME                       VARCHAR(20)
 LAST_NAME               NOT NULL VARCHAR(25)
 EMAIL                   NOT NULL VARCHAR(25)
 PHONE_NUMBER                     VARCHAR(20)
 HIRE_DATE               NOT NULL DATE
 JOB_ID                  NOT NULL VARCHAR(10)
 SALARY                           FLOAT(11)
 COMMISSION_PCT                   FLOAT(12)
 MANAGER_ID                       INT(6)
 DEPARTMENT_ID                    INT(4)

EMPLOYEE_ID         FIRST_NAME         JOB_ID 
------------------- ------------------ --------
5100                 BRUCE             CLERK
5101                 JESSICA           SALESMAN
5102                 DEBBY             SALESMAN
\end{verbatim}

\AnswerBox

\begin{enumerate}[label=\bfseries Q\arabic*:]\itemsep10pt
\item \texttt{INSERT INTO EMPLOYEES (employee\_id, first\_name, job\_id) VALUES (5100, `BRUCE', `CLERK');}\newline Assuming that there is a duplicate value check constraint on the EMPLOYEE\_ID column, what will be the outcome of the above statement?
\item \texttt{INSERT INTO EMPLOYEES (employee\_id, first\_name, job\_id) VALUES (51003, `BRUCE', `CLERK');}\newline What will be the output of above statement?
\item \texttt{INSERT INTO EMPLOYEES (employee\_id, first\_name, job\_id ) VALUES (51003, `BRUCE', NULL);}\newline What will be the output of this statement?
\end{enumerate}

\item [Program 2]\ 
\begin{verbatim}
SQL> DESC departments;
 Name                    Null?    Type
 ----------------------- -------- ----------------
 DEPARTMENT_ID           NOT NULL INT(6)
 DEPARTMENT_NAME         NOT NULL VARCHAR(30)
 MANAGER_ID                       INT(11)
 LOCATION_ID                      INT(11)
\end{verbatim}
\AnswerBox

\begin{enumerate}[label=\bfseries Q\arabic*:]\itemsep10pt
\item \texttt{INSERT INTO departments (department\_id, department\_name, manager\_id, location\_id)
VALUES (100, `Human Resources', 121, 1000); }\newline How many rows will be inserted by the above statement?
\item In which order the values will get inserted with respect to the above INSERT statement?
\item \texttt{INSERT INTO departments VALUES (100, `Human Resources', 121, 1000);}\newline What will be the outcome of this modification?
\end{enumerate}

\item[Program 3]\
\begin{verbatim}
SQL> DESC employees;
 Name                    Null?    Type
 ----------------------- -------- ----------------
 EMPLOYEE_ID             NOT NULL INT(6)
 FIRST_NAME                       VARCHAR(20)
 LAST_NAME               NOT NULL VARCHAR(25)
 EMAIL                   NOT NULL VARCHAR(25)
 PHONE_NUMBER                     VARCHAR(20)
 HIRE_DATE               NOT NULL DATE
 JOB_ID                  NOT NULL VARCHAR(10)
 SALARY                           FLOAT(11)
 COMMISSION_PCT                   FLOAT(12)
 MANAGER_ID                       INT(6)
 DEPARTMENT_ID                    INT(4)
\end{verbatim}

\AnswerBox

\begin{enumerate}[label=\bfseries Q\arabic*:]\itemsep10pt
\item \texttt{INSERT INTO EMPLOYEES (employee\_id, hire\_date) VALUES (210, ``21-JUN-2013'');}\newline What will be the outcome of the above INSERT statement?
\item \texttt{INSERT INTO EMPLOYEES (employee\_id, first\_name) VALUES (210, ``Bryan'');}\newline What will be the outcome of the above INSERT statement?
\item Suppose we need to insert the name \texttt{O'Callaghan} as the last name of the employees table. Which query will give you the required results?
\item \texttt{INSERT INTO EMPLOYEES (employee\_id, first\_name) VALUES (``210'', `Bryan');}\newline What will be the outcome of the above INSERT statement?
\end{enumerate}

\item [Program 4]\ 
\begin{verbatim}
SQL> DESC departments;
 Name                    Null?    Type
 ----------------------- -------- ----------------
 DEPARTMENT_ID           NOT NULL INT(6)
 DEPARTMENT_NAME         NOT NULL VARCHAR(30)
 MANAGER_ID                       INT(11)
 LOCATION_ID                      INT(11)
\end{verbatim}
\AnswerBox

\begin{enumerate}[label=\bfseries Q\arabic*:]\itemsep10pt
\item \texttt{INSERT INTO departments VALUES (200, `Accounts', NULL, NULL);}\newline What will be the outcome of the above INSERT statement?
\item \texttt{INSERT INTO departments VALUES (NULL, `Accounts', NULL);}\newline What will be the outcome of the above INSERT statement?
\item \texttt{INSERT INTO departments VALUES (100, `Human Resources', 121, 1000);}\newline What will be the outcome of this modification?
\end{enumerate}

\item [Program 5]\ 
\begin{verbatim}
SQL> desc countries;
 Name                    Null?    Type
 ----------------------- -------- ----------------
 COUNTRY_ID              NOT NULL INT
 COUNTRY_NAME                     VARCHAR(40)
 REGION_ID                        INT
\end{verbatim}
\AnswerBox

\begin{enumerate}[label=\bfseries Q\arabic*:]\itemsep10pt
\item \texttt{INSERT INTO COUNTRIES (1, `INDIA', 99);}\newline How many rows will be inserted by the above statement?
\item Which sql query update the column \texttt{region\_id} to 101 where country\_id = 1? 
\item Which query deletes a record from the table ``COUNTRIES'' where country\_id = 1?
\end{enumerate}
\end{description}
\end{document}

