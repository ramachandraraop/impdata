\documentclass[11pt,a4paper]{article}
\usepackage{graphicx}
\usepackage{listings}
\lstset{language=Java,numbers=left,numberstyle=\tiny,numbersep=10pt,showstringspaces=false}
\usepackage{array}
\usepackage{enumitem}
\def\AnswerBox{\fbox{\begin{minipage}{4in}\hfill\vspace{0.5in}\end{minipage}}}
\usepackage{fancyhdr}
\pagestyle{fancy}
\renewcommand{\headrulewidth}{0pt}
\rhead{\includegraphics[scale=.5]{../Images/TS-Logo.png}}
\begin{document}
\centerline{\huge{\textbf{Working With DDL}}}
\vspace{1pc}
\centerline{\Large{ \textbf{Workbook}}}
\subsection*{Answer the following}
\begin{enumerate}\itemsep10pt
\item The \underline{\hspace{3cm}} statements are used to define the database structure or schema.
\item Semicolon at the end of each SQL statement will terminate the SQL statement [True / False]
\item SQL is not case insensitive [True / False]
\item SQL is the standard language for \underline{\hspace{3cm}} database systems.
\item \underline{\hspace{3cm}} command is used to retrieve data from database.
\item \underline{\hspace{3cm}} commands that allow the user to manage database transactions.

\item \underline{\hspace{3cm}} are used to specify rules for the data in a table.
\item The \underline{\hspace{3cm}} means hiding implementation details (i.e. high leveldetails) from end user.
\item How many levels of data abstraction is there? \underline{\hspace{3cm}}
\item \underline{\hspace{3cm}} is the problem of storing the same data item in more than one place.
\item The \underline{\hspace{3cm}} level of abstraction describes only part of entire database.
\item Write any two DDL commands \underline{\hspace{3cm}}
\item The \underline{\hspace{3cm}} are used to maintain the Data Integrity
\end{enumerate}
\end{document}
