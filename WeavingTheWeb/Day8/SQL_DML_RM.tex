\documentclass[11pt,a4paper]{article}

\usepackage{verbatim}
\usepackage{fancyhdr}           % For header and footer
\usepackage{multicol}           % Allows multicols in tables
\usepackage{graphicx}           % For images
\usepackage{xcolor}             % For hex colours
\usepackage{colortbl}           % For coloured tables
\usepackage{setspace}           % For line height
\usepackage{seqsplit}           % Splits long words.
\usepackage{amsmath}            % Symbols

%\usepackage[english,german,french,spanish,italian]{babel}              % Languages


\graphicspath{{../Images/}}
% Lengths and widths
\addtolength{\textwidth}{5cm}
\addtolength{\hoffset}{-1cm}
\setlength{\headsep}{-12pt} % Reduce space between header and content
\setlength{\headheight}{85pt} % If less, LaTeX automatically increases it
\renewcommand{\footrulewidth}{2pt} % Remove footer line
\renewcommand{\headrulewidth}{1pt} % Remove header line
\renewcommand{\seqinsert}{\ifmmode\allowbreak\else\-\fi} % Hyphens in seqsplit
\renewcommand{\arraystretch}{1.3}
\onehalfspacing



% Commands
\newcommand{\SetRowColor}[1]{\noalign{\gdef\RowColorName{#1}}\rowcolor{\RowColorName}} % Shortcut for row colour
\newcommand{\mymulticolumn}[3]{\multicolumn{#1}{>{\columncolor{white}}#2}{#3}} % For coloured multi-cols
\newcolumntype{x}[1]{>{\raggedright}p{#1}} % New column types for ragged-right paragraph columns
\newcommand{\tn}{\tabularnewline} % Required as custom column type in use

% Font and Colours
\definecolor{HeadBackground}{HTML}{333333}
\definecolor{FootBackground}{HTML}{666666}
\definecolor{TextColor}{HTML}{333333}
\definecolor{DarkBackground}{HTML}{6B8E23} %{FD1AA8}
\definecolor{LightBackground}{HTML}{E8FED8} %D3FDC8
\definecolor{tit}{HTML}{FF6600}
\renewcommand{\familydefault}{\sfdefault}
\color{TextColor}
 \headsep = 25pt
% Header and Footer
\pagestyle{fancy}
\usepackage[headheight=110pt]{geometry}
\fancyhf{}% Clear header/footer

\fancyhead[r]{\includegraphics[width = 4cm, height = 2cm]{TS-Logo.png}\hspace{0cm}}
\fancyhead[l]{\Large{\bf{\textcolor{tit}{\textrm{Data Manipulation Language}}}}}
%\fancyfoot[C]{Footer}% \fancyfoot[R]{\thepage}
\renewcommand{\headrulewidth}{0.4pt}% Default \headrulewidth is 0.4pt
\renewcommand{\footrulewidth}{0.4pt}% Default \footrulewidth is 0pt

\rfoot{Page \thepage}
\lfoot{COPYRIGHT \textcopyright TALENTSPRINT, 2015. ALL RIGHTS RESERVED.}

\vspace{2in}

\begin{document}
\vspace{5in}
%\chapter*{Session 3 - Working with DML - INSERT, UPDATE, DELETE}


Data Manipulation Language (DML) statements are used for managing data within schema objects (e.g. table).\newline

It includes Inserting, Updating and deleting of data.
\subsubsection*{Insert Statement}
INSERT statement allows you to insert one or more rows to the table.It is possible to write INSERT INTO in two forms:

The first form doesn't specify the column names where the data will be inserted, only their values. It is used where users need to insert values to all available columns of a table.

\textbf{Syntax}\\
         INSERT INTO tablename VALUES (value1, value2, value3,...);

\textbf{Example}\\
mysql$>$ INSERT INTO student VALUES (1,`Ajay',`Hyderabad');

The second form specifies both the column names and the values to be inserted. It is used where users need to insert value to specific column(s) of a given table.

\textbf{Syntax}\\
         INSERT INTO tablename (column1, column2,column3,...) VALUES (value1, value2, value3,...)

\textbf{Example}\\
mysql$>$ INSERT INTO student (Rollno, Name)VALUES (2, `Rahul');

\subsubsection*{Update Statement}
SQL UPDATE statement is used to update existing data in database tables. It can be used to change values of single row, group of rows or even all rows in a table.

The SET clause determines the column(s) name and the changed value(s). The changed values could be a constant value, expression or even a subquery.

WHERE clause determines which rows of the tables will be updated. It is an optional part of SQL UPDATE statement. If WHERE clause is ignored, all rows in the tables will be
updated.

In employees table, if you want to update the email of Mary with employeeid 1 with the new email as mson@talentsprint.com, you can execute the following query.

\textbf{Syntax}
         UPDATE tablename SET columnname=value [where condition];
\textbf{Example}
mysql$>$ UPDATE employees SET email =`mson@talentsprint.com' 
                                      WHERE employeeid = 1;

The following syntax will update the email of all employees with the new email as mson@talentsprint.com.

mysql$>$ UPDATE employees SET email =`mson@talentsprint.com';

\subsubsection*{Delete Statement}

The DELETE FROM statement is used to delete records from a database table. The WHERE clause in DELETE statement specifies condition to limit which rows you want to
remove.

\textbf{Syntax}
         DELETE from tablename [where condition];\\
\textbf{Example}
         DELETE from employees where salary=10000;\\
        
The following syntax will delete all records from employees.

\textbf{Syntax}
         DELETE from tablename;\\
\textbf{Example}
         DELETE from employees;\\


\end{document}
