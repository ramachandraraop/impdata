\documentclass[11pt,a4paper]{article}
\usepackage{graphicx}
\usepackage{listings}
\lstset{language=C,numbers=left,numberstyle=\tiny,numbersep=10pt,showstringspaces=false}
\usepackage{array}
\usepackage{enumitem}

\def\AnswerBox{\fbox{\begin{minipage}{4in}\hfill\vspace{0.5in}\end{minipage}}}
\usepackage{fancyhdr}
\pagestyle{fancy}
\renewcommand{\headrulewidth}{0pt}
\rhead{\includegraphics[scale=.5]{../Images/TS-Logo.png}}
\begin{document}
\centerline{\huge{ \textbf{Working With DRL-Select}}}

\vspace{1pc}

\centerline{\Large{ \textbf{Workbook}}}
\section*{Answer the following}
\begin{enumerate}\itemsep10pt
%\underline{\hspace{3cm}}.
\item Write a query to select Emp\_Name and DeptNo from ``Emp'' table \underline{\hspace{3cm}}.

\item Write a query to retrieve records from ``Emp'' table \underline{\hspace{3cm}}.

\item Write a query to retrieve `Salary' from ``Emp'' table \underline{\hspace{3cm}}.

\item \underline{\hspace{3cm}} operator is used for SQL wildcards.

\item \underline{\hspace{3cm}} character is used to separate column names in a \texttt{SELECT} statement.

\item \underline{\hspace{3cm}} keyword is used to retrieve data from a table.

\item Where is the \texttt{GROUP BY} clause placed in a SQL \texttt{SELECT} statement?\underline{\hspace{3cm}}

\item Write a query to select Emp\_name in ascending order from ``employee'' table.



\end{enumerate}

\subsection*{Exercises}
\begin{itemize}
\item Write the expected output, or compiler errors if any, for each of the following programs.
\end{itemize}
\begin{description}
\item[Program 1]\
\begin{verbatim}
SQL> DESC employees;
 Name                    Null?    Type
 ----------------------- -------- ----------------
 EMPLOYEE_ID             NOT NULL INT(6)
 FIRST_NAME                       VARCHAR(20)
 LAST_NAME               NOT NULL VARCHAR(25)
 EMAIL                   NOT NULL VARCHAR(25)
 PHONE_NUMBER                     VARCHAR(20)
 HIRE_DATE               NOT NULL DATE
 JOB_ID                  NOT NULL VARCHAR(10)
 SALARY                           FLOAT(11)
 COMMISSION_PCT                   FLOAT(12)
 MANAGER_ID                       INT(6)
 DEPARTMENT_ID                    INT(4)
\end{verbatim}

%\AnswerBox

\begin{enumerate}[label=\bfseries Q\arabic*:]\itemsep10pt
\item What will be the output of the query given below?\newline
\texttt{SELECT distinct salary from employees where last\_name = `Adam';}
\end{enumerate}

\item[Program 2]\
\begin{verbatim}
SQL> select *from Emp;

EMPNO     SAL
--------- ------
7499      1600
7521      8500
7566      2975
7654      1250
7698      2850
7782      2450
7788      3000
7839      5000
8 rows selected.

\end{verbatim}

%\AnswerBox

\begin{enumerate}[label=\bfseries Q\arabic*:]\itemsep10pt
\item What will be the output of the query given below?\newline
\texttt{SELECT empno, sal FROM emp WHERE sal > Any(2000, 3000, 4000);}
\item What will be the output of the query given below?\newline
\texttt{SELECT empno, sal FROM   emp WHERE  all> Any (2000, 3000, 4000);}
\item What will be the output of the query given below?\newline
\texttt{SELECT empno, sal  FROM   emp WHERE  sal > Some (2000, 3000, 4000);}
\end{enumerate}
\item [Program 3]\ 
\begin{verbatim}
SQL> SELECT * FROM CHECKS;

CHECK#   PAYEE                AMOUNT   REMARKS
-------- -------------------- -------- ------------------
       1 Ma Bell                   150 Have sons next time
       2 Reading R.R.           245.34 Train to Chicago
       3 Ma Bell                200.32 Cellular Phone
       4 Local Utilities            98 Gas
       5 Joes Stale $ Dent         150 Groceries
      16 Cash                       25 Wild Night Out
      17 Joans Gas                25.1 Gas
       9 Abes Cleaners           24.35 X-Tra Starch       
      20 Abes Cleaners            10.5 All Dry Clean
       8 Cash                       60 Trip to Boston
      21 Cash                       34 Trip to Dayton
11 rows selected.
\end{verbatim}
%\AnswerBox

\begin{enumerate}[label=\bfseries Q\arabic*:]\itemsep10pt
\item What will be the output of the query given below?\newline
\texttt{SELECT * FROM CHECKS ORDER CHECK\#;}
\item What will be the output of the query given below?\newline
\texttt{SELECT PAYEE, AMOUNT FROM CHECKS ORDER BY CHECK\# ASC;}
\item Write a query to order CHECKS by PAYEE and REMARKS.
\item What will be the output of the query given below?\newline
\texttt{SELECT * FROM CHECKS ORDER BY PAYEE ASC, REMARKS DESC;}
\end{enumerate}
\end{description}


\end{document}
