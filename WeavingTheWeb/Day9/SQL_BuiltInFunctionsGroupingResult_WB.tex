\documentclass[11pt,a4paper]{article}
\usepackage{graphicx}
\usepackage{listings}
\lstset{language=C,numbers=left,numberstyle=\tiny,numbersep=10pt,showstringspaces=false}
\usepackage{array}
\usepackage{enumitem}

\def\AnswerBox{\fbox{\begin{minipage}{4in}\hfill\vspace{0.5in}\end{minipage}}}
\usepackage{fancyhdr}
\pagestyle{fancy}
\renewcommand{\headrulewidth}{0pt}
\rhead{\includegraphics[scale=.5]{../Images/TS-Logo.png}}
\begin{document}
\centerline{\huge{ \textbf{Built-In Functions And Grouping Result}}}

\vspace{1pc}

\centerline{\Large{ \textbf{Workbook}}}
\section*{Answer the following}

\begin{enumerate}\itemsep10pt

\item \underline{\hspace{3cm}} function is used to find the maximum value in a column.

\item \underline{\hspace{3cm}} function is used to compare dates.

\item \underline{\hspace{3cm}} function is used to get the current time in MYSQL.

\item \underline{\hspace{3cm}} function converts a field to Uppercase.

\item \underline{\hspace{3cm}} function returns count of rows in a table.

\item \underline{\hspace{3cm}} function is used to calculate the arc sine of a value.

\item The function SIGN(-50) gives value as \underline{\hspace{3cm}}

\item The function ABS(-50) returns \underline{\hspace{3cm}}

\item The output of day(`2008-01-12') would be same as that of day(`12.01.2008'). (True / False)

\item What will be the output of the following query? \underline{\hspace{3cm}}

\texttt{SELECT ROUND(12345, -2), ROUND(12.345,2), ROUND(12.345,3) FROM DUAL;}

\item What will be the output of the following query?\underline{\hspace{3cm}}

\texttt{SELECT SUBSTR(`Talent Sprint', -10) FROM DUAL;}

\item What will be the output of the following query?\underline{\hspace{3cm}}

\texttt{SELECT DAYOFWEEK(`2013-07-16') FROM DUAL;}

\end{enumerate}
\newpage
\subsection*{Exercises}
\begin{itemize}
\item Write the expected output, or compiler errors if any, for each of the following programs.
\end{itemize}
\begin{description}
\item[Program 1]\
\begin{verbatim}
MYSQL> SELECT * FROM TEAMSTATS;

NAME      POS AB  HITS WALKS SINGLES DOUBLES TRIPLES HR SO
--------- --- --- ---- ----- ------- ------- ------- -- --
JONES     1B  145 45   34    31       8      1       5  10
DONKNOW   3B  175 65   23    50      10      1       4  15
WORLEY    LF  157 49   15    35       8      3       3  16
DAVID     OF  187 70   24    48       4      0       17 42
HAMHOCKER 3B  50  12   10    10       2      0       0  13
CASEY     DH  1   0    0     0        0      0       0  1


6 rows selected.



\end{verbatim}

%\AnswerBox


\begin{enumerate}[label=\bfseries Q\arabic*:]\itemsep10pt
\item What will be the output  of the query given below?\newline
\texttt{SELECT SUM(SINGLES) TOTAL\_SINGLES FROM TEAMSTATS;}
\item What will be the output of the query given below?\newline
\texttt{SELECT SUM(SINGLES) TOTAL\_SINGLES, SUM(DOUBLES) TOTAL\_DOUBLES, SUM(TRIPLES) TOTAL\_TRIPLES, 
SUM(HR) TOTAL\_HR FROM TEAMSTATS;}
\item What will be the output of the query given below?\newline
\texttt{SELECT SUM(HITS)/SUM(AB) TEAM\_AVERAGE FROM TEAMSTATS;}
\item What will be the output of the query given below?
\texttt{SELECT SUM(NAME) FROM TEAMSTATS;}
\item What will be the output of the query given below?\newline
\texttt{SELECT COUNT(*) FROM TEAMSTATS WHERE HITS/AB < .35;}
\item What will be the output of the query given below?\newline
\texttt{SELECT COUNT(NAME) NUM\_BELOW\_350 FROM TEAMSTATS WHERE HITS/AB < .35;}
\end{enumerate}
\newpage
\item[Program 2]\
\begin{verbatim}
MYSQL> SELECT * FROM TEAMSTATS;

NAME      POS AB  HITS WALKS SINGLES DOUBLES TRIPLES HR SO
--------- --- --- ---- ----- ------- ------- ------- -- --
JONES     1B  145 45   34    31       8      1       5  10
DONKNOW   3B  175 65   23    50      10      1       4  15
WORLEY    LF  157 49   15    35       8      3       3  16
DAVID     OF  187 70   24    48       4      0       17 42
HAMHOCKER 3B  50  12   10    10       2      0       0  13
CASEY     DH  1   0    0     0        0      0       0  1

6 rows selected.

\end{verbatim}

%\AnswerBox

\begin{enumerate}[label=\bfseries Q\arabic*:]\itemsep10pt
\item What will be the output of the query given below?\newline
\texttt{SELECT AVG(SO) AVE\_STRIKE\_OUTS FROM TEAMSTATS;}
\item What will be the output of the query given below?\newline
\texttt{SELECT AVG(HITS)/AVG(AB) TEAM\_AVERAGE FROM TEAMSTATS;}
\item What will be the output of the query given below?\newline
\texttt{SELECT MAX(HITS) FROM TEAMSTATS;}
\item What will be the output of the query given below?\newline
\texttt{SELECT NAME FROM TEAMSTATS WHERE HITS = MAX(HITS);}
\item What will be the output of the query given below?\newline
\texttt{SELECT MIN(NAME) FROM TEAMSTATS;}
\item What will be the output of the query given below?\newline
\texttt{SELECT MIN(AB), MAX(AB) FROM TEAMSTATS;}
\end{enumerate}
\item [Program 3]\ 
\begin{verbatim}
MYSQL> SELECT * FROM PROJECT;

TASK           STARTDATE ENDDATE
-------------- --------- ---------
KICKOFF MTG    01-APR-95 01-APR-95
TECH SURVEY    02-APR-95 01-MAY-95
USER MTGS      15-MAY-95 30-MAY-95
DESIGN WIDGET  01-JUN-95 30-JUN-95
CODE WIDGET    01-JUL-95 02-SEP-95
TESTING        03-SEP-95 17-JAN-96

6 rows selected.
\end{verbatim}
%\AnswerBox

\begin{enumerate}[label=\bfseries Q\arabic*:]\itemsep10pt
\item What will be the output of the query given below?\newline
\texttt{SELECT TASK, STARTDATE, ENDDATE ORIGINAL\_END, ADD\_MONTHS(ENDDATE, 2) FROM PROJECT;}
\item What will be the output of the query given below?\newline
\texttt{SELECT ENDDATE, LAST\_DAY(ENDDATE) FROM PROJECT;}
\item What will be the output of the query given below?\newline
\texttt{SELECT DISTINCT LAST\_DAY(`1-FEB-95') NON\_LEAP, LAST\_DAY(`1-FEB-96') LEAP FROM PROJECT;}
\item What will be the output of the query given below?\newline
\texttt{SELECT TASK, STARTDATE, ENDDATE, MONTHS\_BETWEEN(STARTDATE, ENDDATE) DURATION FROM PROJECT;}
\item What will be the output of the query given below?\newline
\texttt{SELECT STARTDATE, NEXT\_DAY(STARTDATE, `FRIDAY') FROM PROJECT;}
\end{enumerate}

\item[Program 4]\
\begin{verbatim}
MYSQL> DESC SALES;


FIELD         TYPE       NULL  
------------- ---------- -------
STORE_ID      INT(11)    YES
SALES_DATE    DATE       YES
SALES_AMOUNT  FLOAT      YES

\end{verbatim}

%\AnswerBox

\begin{enumerate}[label=\bfseries Q\arabic*:]\itemsep10pt
\item Write a query to find the sales amount for each store?
\item Write a query to list all stores whose total sales amount is above 5000?
\item Write a query to find the total number of stores in the SALES table?
\item Write a query to find the total sales amount for STORE\_ID 25 and 45?
\end{enumerate}
\newpage
\item [Program 5]\ 
\begin{verbatim}
MYSQL> SELECT * FROM EXAM_RESULTS;

STUDENT_ID  FIRST_NAME    LAST_NAME EXAM_ID  EXAM_SCORE
----------- ------------- --------- -------- -----------
10          LAURA         LYNCH	    1        90 
10          LAURA         LYNCH     2        85
11          GRACE         BROWN     1        78
11          GRACE         BROWN     2        72
12          JAY           JACKSON   1        95
12          JAY           JACKSON   2        92
13          WILLIAM       BISHOP    1        70
13          WILLIAM       BISHOP    2        100
14          CHARLES       PRADA     2        85
9 rows selected.
\end{verbatim}
%\AnswerBox

\begin{enumerate}[label=\bfseries Q\arabic*:]\itemsep10pt
\item What will be the output of the query given below?\newline
\texttt{SELECT COUNT(DISTINCT STUDENT\_ID) FROM EXAM\_RESULTS;}
\item Write a query to find the average exam score for EXAM\_ID = 1?
\item Write a query to retrieve students records whose last name starts with `L'?
\item What will be the output of the query given below?\newline
\texttt{SELECT MAX(EXAM\_SCORE) FROM EXAM\_RESULTS WHERE EXAM\_ID = 1 AND FIRST\_NAME LIKE `\%E\%';}

\item How many records does the following statement returns?\newline
\texttt{SELECT * FROM EXAM\_RESULTS WHERE LAST\_NAME LIKE `\%N\%' AND EXAM\_SCORE > 88;}
\item How many records does the following statement returns?\newline 
\texttt{MYSQL> SELECT * FROM EXAM\_RESULTS WHERE STUDENT\_ID <= 12 AND EXAM\_SCORE > 85;}
\end{enumerate}

\end{description}

\end{document}