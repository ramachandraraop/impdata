\documentclass[11pt,a4paper]{article}
\author{TalentSprint}
\date{}
\usepackage{verbatim}
\usepackage{fancyhdr}           % For header and footer
\usepackage{multicol}
\usepackage{colortbl}           % For coloured tables
\usepackage{setspace}           % For line height
\usepackage{seqsplit}           % Splits long words.
\usepackage{amsmath} 
\usepackage{graphicx}
\usepackage{array}
\usepackage{enumitem}
\usepackage{xcolor}
\usepackage[tikz]{bclogo}
\usepackage{textcomp}
\usepackage{listings}
\lstset{language=python,numbers=left,numberstyle=\tiny,numbersep=10pt,showstringspaces=false}

\headheight=14pt
\lhead{\nouppercase{}}
\rhead{\nouppercase{\leftmark}}

\graphicspath{{../Images/} {../ScreenShots/}}
\setcounter{tocdepth}{1}
\setlength\parindent{0pt}
\parskip=4pt
\newcommand{\Code}[1]{\textbf{\texttt{#1}}}
%\thispagestyle{empty}
\begin{comment}
\def\AnswerBox{\fbox{\begin{minipage}{4in}\hfill\vspace{0.5in}\end{minipage}}}


\vspace{1.5pc}
\topskip0pt
\vspace*{\fill}
\centerline{\Huge Modern Programming}
\vspace{2pc}
\centerline{\Huge Practice}
\vspace{2pc}
\centerline{\Large using Python}
\vspace*{\fill}
\centerline{Prepared by TalentSprint WISE Team} 
\setcounter{page}{1}


\end{comment}
%========================================================================

% Lengths and widths
\addtolength{\textwidth}{5cm}
\addtolength{\hoffset}{-1cm}
\setlength{\headsep}{-12pt} % Reduce space between header and content
\setlength{\headheight}{85pt} % If less, LaTeX automatically increases it
\renewcommand{\footrulewidth}{2pt} % Remove footer line
\renewcommand{\headrulewidth}{1pt} % Remove header line
\renewcommand{\seqinsert}{\ifmmode\allowbreak\else\-\fi} % Hyphens in seqsplit
% This two commands together give roughly
% the right line height in the tables
\renewcommand{\arraystretch}{1.3}
\onehalfspacing



% Commands
\newcommand{\SetRowColor}[1]{\noalign{\gdef\RowColorName{#1}}\rowcolor{\RowColorName}} % Shortcut for row colour
\newcommand{\mymulticolumn}[3]{\multicolumn{#1}{>{\columncolor{white}}#2}{#3}} % For coloured multi-cols
\newcolumntype{x}[1]{>{\raggedright}p{#1}} % New column types for ragged-right paragraph columns
\newcommand{\tn}{\tabularnewline} % Required as custom column type in use

% Font and Colours
\definecolor{HeadBackground}{HTML}{333333}
\definecolor{FootBackground}{HTML}{666666}
\definecolor{TextColor}{HTML}{333333}
\definecolor{DarkBackground}{HTML}{6B8E23} %{FD1AA8}
\definecolor{LightBackground}{HTML}{E8FED8} %D3FDC8
\definecolor{tit}{HTML}{FF6600}
\renewcommand{\familydefault}{\sfdefault}
\color{TextColor}
 \headsep = 25pt
% Header and Footer
\pagestyle{fancy}
\usepackage[headheight=110pt]{geometry}
\fancyhf{}% Clear header/footer

\fancyhead[r]{\includegraphics[width = 4cm, height = 2cm]{TS-Logo.png}\hspace{0cm}}

%=================================TITLE=====================================
\fancyhead[l]{\Large{\bf{\textcolor{tit}{\textrm{Strings}}}}}
%===========================================================================

\renewcommand{\headrulewidth}{0.4pt}% Default \headrulewidth is 0.4pt
\renewcommand{\footrulewidth}{0.4pt}% Default \footrulewidth is 0pt

\rfoot{Page \thepage}
\lfoot{COPYRIGHT \textcopyright TALENTSPRINT, 2015. ALL RIGHTS RESERVED.}




\begin{document}
%\chapter*{Strings}
A string in Python is a sequence of characters. For Python to recognize a sequence of characters, like \emph{hello}, as a string, it must be enclosed in quotes. The string can be enclosed in single or double quotes.

Example:
\begin{verbatim}
>>> ``Hello World''
`Hello World'
>>> `Hello World'
`Hello World'
\end{verbatim}

\emph{Note: Python interpreter displays the string with single quotes.}

Having two sorts of quotes can be useful in certain circumstances. If you want text itself to include quotes of one type can define it surrounded by other type.

Example:
\begin{verbatim}
>>> print(``This is `Use of Single Quotes'.'')
This is `Use of Sungle Quotes'.
>>> print(`This is ``Use of Double Quotes''.')
This is ``Use of Double Quotes''.
\end{verbatim}


\textbf{Triple Quoted String Literals}

Strings delimited by single or double quote character are required to lie within a single line. It is sometimes convenient to have a multi-line string, which can be delimited with triple quotes

Example:
\begin{verbatim}
>>> str1 = '''Hello
...  "Good Morning!!!"
...  Have a cup of coffee.'''
>>> print(str1)
Hello
"Good Morning!!!"
Have a cup of coffee.
\end{verbatim}
\section*{String Operations}
\subsection*{Concatenation}

The plus operation with strings means concatenate the strings. Python looks at the type of operands before
deciding what operation is associated with the +.

The plus ( + ) sign is the string \emph{concatenation operator} which is used to combine number of strings and returns the new string.

Example:
\begin{verbatim}
>>> str1 = "Hello World"
>>> print(str1 + " 'Can Be Joined'")   # Prints concatenated string 
Hello World 'Can Be Joined'
\end{verbatim}

\subsection*{Repetition Operator}

The asterisk( * ) sign is the \emph{repetition operator} which is use to repeat the string as many times as specified.

Example:
\begin{verbatim}
>>> str1 = "Hello World"
>>> print(str1 * 3)   # Prints string 3 times
Hello WorldHello WorldHello World
\end{verbatim}

\section*{Accessing Characters In String}
We can access a character from the string by specifying the index of the character. Index starts from '0' indicates the beginning of the string and working their way from -1 at the end.

\subsection*{For Example:}
\begin{verbatim}
>>> str1 = "Strings In Python"
>>> print(str1)       # Prints complete string
Strings In Python
>>> print(str1[0])    # Prints first character of the string
S
>>> print(str1[5])    # Prints sixth character of the string
g
>>> print(str1[-1])   # Prints the last character of the string
n
>>> print(str1[-3])   # Prints the third character from the last
h
>>> print(str1[-8])   # Prints the eigth character from the last
n
\end{verbatim}

\section*{Slicing the String}

We can access the subsets of a string using slice operator ([:]) with indexes starting at 0 in the beginning of the string and working their way from -1 at the end.

\subsection*{For Example: }
\begin{verbatim}
>>> str2 = "Slicing the String"
>>> print(str2[4:])   # Prints string starting from the 5th character
ing the String
>>> print(str2[:4])   # Prints the first four characters
Slic
>>> print(str2[2:8])  # Prints characters starting from 3rd to 7th
icing
>>> printt(str2[4:-3]) # Prints characters from 5th to 3rd character from last
ing the Str
\end{verbatim}

\section*{String Methods}
string methods performs operations on strings.

Strings have their own set of functions. In this section we will go through few of them:

\begin{description}
\item[len()]\

The \emph{len()} function returns the length of a string as an integer.
\texttt{\emph{len}("String")}

\textbf{Example}\\
\begin{verbatim}
>>> len("Hello World") # This returns the value as 11.
>>>
>>> name = "TalentSprint" 
>>> len(name) # 12
\end{verbatim}

\item[lower()]\
Converts all uppercase letters in string to lowercase and return the new string.
\begin{verbatim}
>>> s1 = "PYTHON"
>>> s1.lower()
'python'
\end{verbatim}

\item[upper()]\
Converts lowercase letters in string to uppercase and return the new string.
\begin{verbatim}
>>> s2 = "python"
>>> s2 = name.upper()
>>> print(s2)
'PYTHON'
\end{verbatim}

\item[replace()]\
The function \emph{replace()} returns a copy of the string with all occurrences of substring old replaced by new.

\textbf{Syntax:}
\begin{verbatim}
str.replace(old, new)
old - This is the old substring to be replaced
new - This is new substring, which would replace old substring.
\end{verbatim}
\textbf{Example:}
\begin{verbatim}
>>> str = "This is example for replace function"
>>> str.replace('is', 'was')
'Thwas was example for replace function' 
\end{verbatim}

\item[split()]\
The method \emph{split()} is used to split on the whitespaces (blanks, newline) and returns the list of sub strings as items.

\textbf{Example: }
\begin{verbatim}
>>> str1 = "Engineering Student"
>>> str1.split()
['Engineering', 'Student']
>>> str1.split("i")
['Eng', 'neer', 'ng Student']
\end{verbatim}

\item[strip()]\
By using \emph{strip()} function, It returns a copy of the string with the leading and trailing characters removed.

\textbf{Example: }
As it treats the argument as a set of characters. In this example, we specify all digits, and some punctuation chars.
% Program that uses  strip with argument.%
\begin{verbatim}
>>> value = "543210=Data,123"
# strip all digits
# Also remove equals sign and comma.
>>> result = value.strip("0123456789=,")
>>> print(result)
Data
\end{verbatim}

\end{description}
\end{document}