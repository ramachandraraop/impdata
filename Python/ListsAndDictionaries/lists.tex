\documentclass[11pt,a4paper]{article}
\author{TalentSprint}
\date{}
\usepackage{verbatim}
\usepackage{fancyhdr}           % For header and footer
\usepackage{multicol}
\usepackage{colortbl}           % For coloured tables
\usepackage{setspace}           % For line height
\usepackage{seqsplit}           % Splits long words.
\usepackage{amsmath} 
\usepackage{graphicx}
\usepackage{array}
\usepackage{enumitem}
\usepackage{xcolor}
\usepackage[tikz]{bclogo}
\usepackage{textcomp}
\usepackage{listings}
\lstset{language=python,numbers=left,numberstyle=\tiny,numbersep=10pt,showstringspaces=false}

\headheight=14pt
\lhead{\nouppercase{}}
\rhead{\nouppercase{\leftmark}}

\graphicspath{{../Images/} {../ScreenShots/}}
\setcounter{tocdepth}{1}
\setlength\parindent{0pt}
\parskip=4pt
\newcommand{\Code}[1]{\textbf{\texttt{#1}}}

\begin{comment}
\def\AnswerBox{\fbox{\begin{minipage}{4in}\hfill\vspace{0.5in}\end{minipage}}}

\thispagestyle{empty}
\vspace{1.5pc}
\topskip0pt
\vspace*{\fill}
\centerline{\Huge Modern Programming}
\vspace{2pc}
\centerline{\Huge Practice}
\vspace{2pc}
\centerline{\Large using Python}
\vspace*{\fill}
\centerline{Prepared by TalentSprint WISE Team} 
\setcounter{page}{1}


\end{comment}
%========================================================================

% Lengths and widths
\addtolength{\textwidth}{5cm}
\addtolength{\hoffset}{-1cm}
\setlength{\headsep}{-12pt} % Reduce space between header and content
\setlength{\headheight}{85pt} % If less, LaTeX automatically increases it
\renewcommand{\footrulewidth}{2pt} % Remove footer line
\renewcommand{\headrulewidth}{1pt} % Remove header line
\renewcommand{\seqinsert}{\ifmmode\allowbreak\else\-\fi} % Hyphens in seqsplit
% This two commands together give roughly
% the right line height in the tables
\renewcommand{\arraystretch}{1.3}
\onehalfspacing



% Commands
\newcommand{\SetRowColor}[1]{\noalign{\gdef\RowColorName{#1}}\rowcolor{\RowColorName}} % Shortcut for row colour
\newcommand{\mymulticolumn}[3]{\multicolumn{#1}{>{\columncolor{white}}#2}{#3}} % For coloured multi-cols
\newcolumntype{x}[1]{>{\raggedright}p{#1}} % New column types for ragged-right paragraph columns
\newcommand{\tn}{\tabularnewline} % Required as custom column type in use

% Font and Colours
\definecolor{HeadBackground}{HTML}{333333}
\definecolor{FootBackground}{HTML}{666666}
\definecolor{TextColor}{HTML}{333333}
\definecolor{DarkBackground}{HTML}{6B8E23} %{FD1AA8}
\definecolor{LightBackground}{HTML}{E8FED8} %D3FDC8
\definecolor{tit}{HTML}{FF6600}
\renewcommand{\familydefault}{\sfdefault}
\color{TextColor}
 \headsep = 25pt
% Header and Footer
\pagestyle{fancy}
\usepackage[headheight=110pt]{geometry}
\fancyhf{}% Clear header/footer

\fancyhead[r]{\includegraphics[width = 4cm, height = 2cm]{TS-Logo.png}\hspace{0cm}}

%=================================TITLE=====================================
\fancyhead[l]{\Large{\bf{\textcolor{tit}{\textrm{Lists}}}}}
%===========================================================================

\renewcommand{\headrulewidth}{0.4pt}% Default \headrulewidth is 0.4pt
\renewcommand{\footrulewidth}{0.4pt}% Default \footrulewidth is 0pt

\rfoot{Page \thepage}
\lfoot{COPYRIGHT \textcopyright TALENTSPRINT, 2015. ALL RIGHTS RESERVED.}




\begin{document}

%\chapter*{Lists}
Recall that the main built-in types in python are numerics, sequences, mappings, classes, instances and exceptions. 

The \textbf{sequence} type is characterised by the concept of order. Broadly speaking a sequence has
\begin{itemize}
    \item a first element
    \item a last element
    \item a way of accessing an arbitrary element
    \item a way of systematically accessing all elements \emph{in order}.
\end{itemize}

Lists are one variety of mutable sequence type. Lists are a compound data type, used to group \emph{related} data together.  That is, they are a collection of data items. We enclose the items within square brackets and separate them by commas. Some examples are

\begin{verbatim}
>>> p = [2, 3, 5, 7, 11, 13]
>>> f = [1, 1, 2, 3, 5, 8, 13]
\end{verbatim}

An empty list is denoted by \Code{[]} and a new list is typically initialised to that.

Lists can be heterogenous; that is they can  contain items of different data types. For example, 
\begin{verbatim}
>>> mix = ["A", 12, 13.7, -2]
\end{verbatim}
\texttt{mix} is a list containing 4 elements: the first is a string; the second and fourth are integers and the third is float.

\section*{Accessing an element}
The elements of a list can be accessed by its \emph{index} which is the position of the element in the list. The convention is to use the index number 0 for the first element. Thus a list of n elements is said to have elements in index 0 to index n-1.
\begin{verbatim}
>>> cubes = [1, 8, 27, 64, 125, 216, 343, 512, 729, 1000]
>>> print(cubes[2])
27
>>> print(cubes[9])
1000
>>> i = 3
>>> print(cubes[i])
64
\end{verbatim}
Lists are mutable. So we can modify an element of a list by assigning a new value as in:
\begin{verbatim}
>>> alpha = ["A", "X", "X", "T"]
>>> alpha[2] = "Q"
>>> alpha
["A", "X", "Q", "T"]
\end{verbatim}
As you can see in the above example, a list can have multiple elements with the same value.

But we cannot add an element by trying to access a non-existing index. See \ref{subsec:InvalidIndex}.

\subsection*{Negative indices}
In languages like C, Java you need to know/compute the number of elements in an array, in order to access the last element. But python gives a convenient shorthand: -1 is the index of the last element, -2 is the index of the last but one and so on.
\begin{verbatim}
>>> cubes[-1] == 1000
True
>>> print(cubes[-2])
729
\end{verbatim}
\subsection*{Invalid indices}
It is an error to try to access an elemnent that is not in the list; that is trying to access cubes[10] or cubes[-11] etc., will give \emph{IndexError}.
\label{subsec:InvalidIndex}

\section*{Slicing and striding}
We can \emph{slice}, that is select a part of a list by using the slicing operation. For example,
\begin{verbatim}
>>> cubes[1:5]
[8, 27, 64, 125]
\end{verbatim}

The syntax of a slice is \Code{aList[start:end]}. Note that this operation
    \begin{itemize}
        \item returns a new list
        \item the first element of that new list is \Code{aList[start]}
        \item the last element is \Code{aList[end - 1]}, that is \Code{aList} has (\Code{end} - \Code{start}) elements
        \item aList is unchanged
    \end{itemize}

\begin{verbatim}
>>> p = [2, 3, 5, 7, 11, 13, 17, 19]
>>> p[3:7]
[7, 11, 13, 17]
>>> tp = p[4:]
>>> print(tp)
[11, 13, 17, 19]
>>> print(p[:-2])
[2, 3, 5, 7, 11, 13]
\end{verbatim}

As we can see from the above if either \Code{start} or \Code{end} is not specified it defaults to be the logical end point.

\subsection*{Striding}
 
While creating a slice we can pick every $n^{th}$ element instead of every element. That is, achieved by specifying \Code{aList[start:end:n]}. This will return a new list whose starting elements are \Code{aList[start]}, \Code{aList[start + n]}, \Code{aList[start + 2n]}, \Code{aList[start + 3n]} \ldots till the \Code{aList[end]}. 

\section*{List methods}
All lists support the various operations by \emph{methods}. The general syntax is \Code{list.method([param])}. Some methods mutate the list while others do not. It is important to be aware whether a method is mutating or not.
\subsection*{Adding elements}
We noted that we cannot use a non-existing index to add an element to a list. There are many ways of adding elements to a list:
    \begin{itemize}
        \item Concatenate a list to another: \Code{alpha = alpha + beta}, where \Code{alpha} and \Code{beta} are lists, results in the elements of \Code{beta} being added to the end of existing elements in \Code{alpha}
        \item Extend a list: \Code{alpha.extend(beta)} has the same effect as above.
        \item To add only one element, use \Code{alpha.append(newElement)}
        \item To add an element in the middle, say index $n$, use \Code{alpha.insert(n, newElement)}
    \end{itemize}

\subsection*{Removing Elements}
Again there are different ways:
    \begin{itemize}
        \item Specify the value of the element to be removed:  \Code{alpha.remove(x)} where \Code{x} is an elemnt in \Code{alpha}. If there is more than one element with the value \Code{x} the first is removed.
        \item We can remove a slice by assigning it to the empty list. Thus \Code{alpha[p:q] = []} will remove the elements in indices \Code{p, p+1 ... q-1}. That means the number of elements in the list is now less by \Code{q - p}. Of course a single element at index p can be removed by, \Code{alpha[p:p+1] = []}.
        \item The code \Code{k = alpha.pop()} does two things: One it assigns the value of the first element of \Code{alpha} to \Code{k}; two it removes the first element from the list. 
        \item \Code{alpha.clear()} removes all the elements in alpha, making it the empty list.
    \end{itemize}
    \subsection*{Utility methods}
    \begin{itemize}
        \item \Code{alpha.sort()} sorts the elements of the list \emph{in place}. That is it mutates the list.
        \item \Code{alpha.reverse()} reverses the order of elements in the list. Again mutates the list.
        \item \Code{alpha.count(x)} returns the number of times the value x occurs in the list.
    \end{itemize}
\end{document}